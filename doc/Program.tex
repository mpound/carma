%$Id: Program.tex,v 1.13 2005/03/15 02:20:21 teuben Exp $

\documentclass[preprint]{aastex} % AASTeXv5.0

\begin{document}


\title{The CARMA Program}

\author{N. S. Amarnath, Peter Teuben}
\affil{Document Version: 2005-February-1}
%% List additional authors in the same way.

\begin{abstract}

\begin{verbatim}

Program is a singleton class that gives a (CARMA) program a uniform CLI look and feel.
Every CARMA program should use the Program class in the way described in this document.

1) program keywords, as doxygen-type tags in the comment (//) section. They are compiled by
   the ``keys'' program into C++ code, then compiled and linked into your executable.
   Doxygen tags (proposed)
    @version
    @usage
    @description
    @keys <keyname>	<default-value>	<type>	<one-line-help>

   The amount of program keywords is defined by, and up to the programmer

2) system keywords
    small set of keywords EACH program will listen to
	- daemonize=t|f
	- logger
		loghost=
		logtype=
		logfile=
	- carmaconf=$CARMA/etc/carma.conf    (to be discussed)
	- corbaconf=$CARMA/etc/corba.conf    (to be discussed)

   The amount of system keywords is defined by the Program class, and future 
   extensions to be discussed.

\end{verbatim}

\end{abstract}

\section{Introduction}

This document proposes a generic structure for all CARMA programs,
including specializations for daemon processes. The intent is to
allow CARMA developers to focus on the functionality to be delivered
rather than the mundane tasks associated with program level tasks,
such as generic exception handling, commandline argument parsing,
default policies for program variables and daemonising a UNIX
(Solaris/Linux) process. Most importantly, it ensures a uniform
look and feel for all CARMA programs.

\section{Program}

Every CARMA program will have to do the following:

\begin{itemize}


\item 
initialize all program and system (command line) parameters
%% program variables are global variables initialized by environment - through
%% CLI, file, and/or env variables.

\item 
initialize a logger

\item
initialize program "watch" variables as required

\item
setup exception handling

\end{itemize}

This argues for a style of programming where the user does not write
a {\tt main()}, but will be provided via a library. The main would
look something like:

\footnotesize\begin{verbatim}
int main(int argc, char *argv[]) 
{
  try {
    Program p(argc,argv);
    p.initialize()
    p.run()
    p.terminate();
  } catch (...) {
    ...
  }
}
  
\end{verbatim}\normalsize
Each there should only be one instance of {\tt Program} in a program,
this argues for a {\bf Singleton} style implementation of this class.
If this class is wrapped through a set of global functions, a CARMA
program could be as simple as this:

\footnotesize\begin{verbatim}
#include  "carma/main.h"

void carma_main(void)
{
  x = getdparam("x");
  cout << "x = " << x << endl;
}
  
\end{verbatim}\normalsize

However, in the style of OO we will give the programmer a little more 
responsibility, and use the {\tt Program} class explicitly.

\subsection{main()}

Application programmers only need to write a ``Program::main()''.
This user interface will be controlled by a 
{\bf singleton} class, {\tt Program}. The
{\tt program keywords} are defined via doxygen tags,
and a special ``compiler'' (dubbed {\tt keys} will
compile these doxygen-tags into a static structure
that is used by the {\tt Program} class at 
instantiation. 

\footnotesize
\begin{verbatim}
#include "carma/util/Program.h"
#include <cstdio>

using namespace std;
using namespace carma::util;


//
// @version     $Revision: 1.13 $ $Date: 2005/03/15 02:20:21 $
//
// @usage       make a Plummer model, with a spatial or mass cut-off
//
// @description
//      mkplummer builds a nbody system according to a Plummer model, in virial
//      units (M=G=-4E=1, with E the total energy), and finite spatial extent
//      which can be regulated by specifying mfrac or rfrac or using their default
//      values.
//      litt: S.J. Aarseth, M. Henon and R. Wielen (1974),
//            Astron. and Astrophys. 37, p. 183.
//
//      There could be more description here, but not sure if the parser will find it.
//
// @key nbody   123     i   number of bodies in the Nbody system
// @key seed    0       i   random seed
// @key mfrac   0.999   r   Mass cutoff
// @key rfrac   22.8    r   Radial cutoff (virial units)
// @key comment ""      s   Optional comments added to output data stream
// @key iflag   f       b   an as yet undocumented flag
// @key oflag   t       b   another as yet undocumented flag
//
//


void compute(void)
{
    Program &p = Program::getProgram();
    int n = p.getiParameter("nbody");
    printf("Finishing off computing:   n=%d\n",n);
}
 
int Program::main()
{
    string comment;
    int  n, seed, argc;
    string *argv;
    double  mfrac, rfrac;
    bool  i_flag, o_flag;
    char  seedlog[128];
 
    n = getiParameter("nbody");
    if (hasValue("comment")) comment = getParameter("comment");
    mfrac = getdParameter("mfrac");
    rfrac = getdParameter("rfrac");
 
    //    seed = srandinter(getiparam("seed"));
    seed = getiParameter("seed");
 
    i_flag = getbParameter("iflag");            // undocumented features
    o_flag = getbParameter("oflag");            // undocumented features

    if (o_flag) cerr << "mkplummer: random seed = " << seed << endl;
    sprintf(seedlog, "       random number generator seed = %d",seed);
    cout << "COMMENT:" << seedlog << endl;
    cout << "nbody = " << n << endl;
    cout << "mfrac = " << mfrac <<endl;
    cout << "i/o flag = " << i_flag << " " << o_flag << endl;
 
    compute();
}
 

\end{verbatim}
\normalsize
which can now be compiled and linked as follows (normally a Makefile takes care of this):

\begin{verbatim}
  % keys tProgram1.cc > tProgram1_keys.cc
  % g++ -o tProgram1 `carmaincs` tProgram1.cc tProgram1_keys.cc `carmalibs`
\end{verbatim}

There are some important properties of this approach:
\begin{enumerate}
\item there is a fixed set of programs keywords, 
that must have a name, a default, a type and some one line help. 
\item keywords are not dynamic, or indexed (cf. FITS)
\item parsing order is well defined (see below)
\end{enumerate}

It is also proposed that the parsing order for the values of the keywords
will be in the following decreasing order of relevance:
\begin{enumerate}
\item command line
\item some table with ``global'' or ``program'' defaults (Amar is working on this,
  this could reside in the proposed \verb+$CARMA/etc/carma.conf\+
\item the static table (as shown by the @key entries near {\it Program::main})
\end{enumerate}

The type of a keyword is useful to give at instantiation time, since the program has
not really begun to run, 
and is already able to stop running because of some trivial parsing error .
\begin{itemize}
\item[{\bf r}] real (float or double; we don't distinguish)
\item[{\bf i}] integer
\item[{\bf b}] boolean (true: t,1,y   false: f,0,n)
\item[{\bf hms}] hh:mm:ss.s type thingo, though decimal also allowed?
\item[{\bf dms}] dd:mm:ss.s type thingo, though decimal also allowed?
\item[{\bf t}] time (dates?) 
\item[{\bf s}] string, catch-all for anything that will not be type-checked at all
\end{itemize}

The type can be optionally following by a ``{\tt +}'' symbol, which is needed if multiple items
of the same type are allowed (we could perhaps allow implied loops
for arrays, e.g. 1:10:2  would compute as 1,3,5,7,9). Complex types, such as 
``{\tt s,i,r,r,r}'' are also possible.

\begin{enumerate}
\item {\bf deamonize=} if used, controls how program is to be daemonized (see {\it daemon(3)}
\item {\bf logger=} if used, allows to deviate from the default CARMA way to log (need input from Marc)
  Could also split this up in {\tt loghost=, logtype=, logfile=}.
\item {\bf carmaconf=} override configuration in \verb+$CARMA/etc/carma.conf\+ 
\item {\bf corbaconf=} override configuration in \verb+$CARMA/etc/corba.conf\+ 
\item {\bf help=} helper (for compatibility {\bf --help} is also understood, including a number of
  associated helper flags.



\end{enumerate}


\subsection{Command Line Interface}

A uniform and consistent Command Line Interface (CLI) will be an important
asset to our software. We are proposing a simple static table of a finite
set of ``{\it program keywords}'' (unique to the program, and defined
by the programmer) and
small set of ``{\it system keywords}'' (same meaning to all CARMA programs,
defined by the Program class).
Although the traditional unix style flags (getopts, now modernized by 
long names) may look appealing, we find their usage often confusing
(e.g. sticky vs. non-sticky, multiple use of flags on the command line, 
the ambigues meaning of booleans: does -l mean yes or no longs?).
We do however allow flags like {\tt --help} that are overloaded to their
system keyword equivalents.

\subsection{Exception Handling}

A uniform policy for managing CARMA exceptions will make life easier for 
programmers. Managing an exception may involve one or more of 

\begin{itemize}

\item
reporting/logging the exception

\item
taking recovery/cleanup actions

\item
throwing an exception to the next level.

\end{itemize}

A common exception handling mechanism will be supported through a common 
base class {\tt carma::util::BaseExceptionObj}, that will provide facilities 
for raising an exception, storing a string describing the nature of the 
exception, and will have the ability to log an exception. One other derived 
classes will also be provided - a {\tt carma::util::SystemExceptionObj}
class for managing system errors.

Class {\tt SystemExceptionObj} will provide facilities to obtain system (OS) 
error details, and concatenate any system information string for the 
error to the logged output. Some system errors may also be fatal. 
In such a case, programmers may get a chance to clean up, and ensure that
the program/thread makes a clean exit.


\section{Initialization and termination of a "typical" CARMA process}

Initialization of a CARMA process will involve initialization of the process
wide logger, and, depending on the nature of the process, 

\begin {enumerate}
\item daemonizing the process
\item initializing cleanup utilities (if we decide to have some, that is)
\item initializing CORBA nameserver and notification services, 
\item reading in program configuration information and initializing
the appropriate variables.
\end {enumerate}

Termination of the process will involve

\begin {enumerate}
\item cleaning up memory (heap) resources
\item releasing system resources (files, shared memeory, sockets...)
\item closing and deleting process wide logger, and
\item exiting.
\end {enumerate}

\section{Examples}

\begin{verbatim}

  @key  file  @required     string     Input filename, no default.
  @key  eps   0.01          double     Pointing accuracy [arcsec]
  @key  dra   15,arcsec     unit       Offset in RA

\end{verbatim}

\begin{references}

argtable  http://argtable.sourceforge.net/doc/html/index.html

clig	  http://wsd.iitb.fhg.de/~kir/clighome/

clo++     http://pmade.org/~pjones/software/clo++/

perl      http://www-106.ibm.com/developerworks/linux/library/l-perl-speak.html

gengetopt http://www.gnu.org/software/gengetopt/gengetopt.html

parseargs

NEMO      src/kernel/io/getparam.c

BaseException	CARMA CVS tree - carma/doc/util/html/index.html

\end{references}


\end{document}

