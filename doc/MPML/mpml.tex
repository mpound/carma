\documentclass[letterpaper,12pt,oneside,pdftex]{article}

%% My Commands
\newcommand{\myname}{Ira W. Snyder}
\newcommand{\myemail}{iws@ovro.caltech.edu}
\newcommand{\mytitle}{Monitor Point Markup Language}

\newcommand{\emaillink}[1]{\href{mailto:#1}{$<$#1$>$}}
\newcommand{\sourcecode}[1]{
    % Any parameters can be changed here to get wider tabs, etc.
    \VerbatimInput[frame=lines,numbers=left,tabsize=4,fontsize=\scriptsize]{#1}
}
\newcommand{\plaintext}[1]{
    \VerbatimInput[frame=lines,tabsize=4,fontsize=\scriptsize]{#1}
}
\newcommand{\fullbox}[1]{\fbox{\begin{minipage}{\textwidth}#1\end{minipage}}}
\newcommand{\shadebox}[1]{\fcolorbox{black}{shade}{\begin{minipage}{\textwidth}#1\end{minipage}}}

\newcommand{\mpml}[1]{\texttt{<#1>}}
\newcommand{\attr}[1]{\texttt{#1}}

% Packed itemize (decreased spacing between elements)
\newenvironment{pitemize}{
\begin{itemize}
  \setlength{\itemsep}{1pt}
  \setlength{\parskip}{0pt}
  \setlength{\parsep}{0pt}
}{\end{itemize}}

% Packed enumerate (decreased spacing between elements)
\newenvironment{penumerate}{
\begin{enumerate}
  \setlength{\itemsep}{1pt}
  \setlength{\parskip}{0pt}
  \setlength{\parsep}{0pt}
}{\end{enumerate}}

%% Packages to include

%% Font to use
%\usepackage{pslatex}
%\usepackage{cmbright}
%\usepackage{concrete}
%\usepackage{palatino}
%\usepackage{ae}
%\usepackage{pxfonts}
%\usepackage{txfonts}

% To put verbatim text in boxes
% NOTE: you will need verbatimbox.sty in the same directory
% NOTE: if it is not already installed globally
\usepackage{verbatimbox}

% Colors for use in \shadebox
\usepackage{color}
\definecolor{shade}{gray}{0.85}

% Fix spacing with 12pt docs
\usepackage{exscale}

% Extra spacing options
%\usepackage{setspace}
%\singlespacing
%\onehalfspacing
%\doublespacing

% Remove space at the beginning of paragraphs
\usepackage{parskip}

% Set Margins
\usepackage[left=1in,top=1in,right=1in,nohead,nofoot]{geometry}

% Support for nicer tables
%\usepackage{booktabs}

% Support for source code listings
\usepackage{fancyvrb}

% Support for removing all page numbers
\usepackage{nopageno}

% Support for URLs (unneeded with hyperref)
%\usepackage{url}

% Support PDF HyperText References
\usepackage[pdftex,bookmarks,breaklinks=true,%
            pdftitle={\mytitle},pdfauthor={\myname},%
            colorlinks]{hyperref}

% Force PDF links to hit the top of an image rather than the caption.
% This makes it easier for users: they don't need to scroll up to see
% the image. This MUST come after hyperref.
\usepackage[all]{hypcap}


%% Generate Title Page
\title{\mytitle}
\author{\myname\\\emaillink{\myemail}}
\date{\today}

\begin{document}

% Switch font families
\sffamily

% Title Page
\maketitle

%%%%%%%%%%%%%%%%%%%%%%%%%%%%%%%%%%%%%%%%%%%%%%%%%%%%%%%%%%%%%%%%%%%%%%%%%%%%%%%%
\begin{abstract}
The Monitor Point Markup Language (MPML) is a set of XML tags used to define
monitor point hierarchies.It functionally serves two purposes: to provide data
for configuring the monitor point database and to configure automatically
generated classes interacting with a hierarchy in software. As the single point
of input for both purposes, it also serves as documentation for monitor point
data. This document describes how to use the MPML tags to define a hierarchy and
generate C++ classes for interacting with it. 
\end{abstract}

%%%%%%%%%%%%%%%%%%%%%%%%%%%%%%%%%%%%%%%%%%%%%%%%%%%%%%%%%%%%%%%%%%%%%%%%%%%%%%%%
\tableofcontents

%%%%%%%%%%%%%%%%%%%%%%%%%%%%%%%%%%%%%%%%%%%%%%%%%%%%%%%%%%%%%%%%%%%%%%%%%%%%%%%%
\section{Getting Started}

The easiest way to get started is to look at the annotated example source file
\texttt{TestSubsystem.mpml}. In a compiled CARMA build tree, you can look at the
generated output files \texttt{TestSubsystem.h} and \texttt{TestSubsystem.cc}.

%%%%%%%%%%%%%%%%%%%%%%%%%%%%%%%%%%%%%%%%%%%%%%%%%%%%%%%%%%%%%%%%%%%%%%%%%%%%%%%%
\section{Generating C++ Code}

The preferred way to generate C++ code is to let the CARMA build system handle
it for you. It is also sometimes convenient to run the code generator by hand,
which is described below.

The help output should get you started:

\begin{verbbox}
iws@build1 ~/devel/carma-cvs/build $ ./scripts/mpmlgen/mpmlgen --help
usage: Generator.py [options] <filename.mpml>

options:
  --version             show program's version number and exit
  -h, --help            show this help message and exit
  -t, --skip-translation
                        Do not generate C++ code
  -v, --verbose         Print verbose messages (give twice for more!)
  -e, --extensions      Generate extension classes
  -M, --dependencies    Generate dependency information
  -o DIR, --output-directory=DIR
                        Output directory (default is current)
  -I DIR, --include-directory=DIR
                        Add an extra include path
  -L FILE, --database=FILE
                        Output database-compatible pseudo-MPML
\end{verbbox}
\shadebox{\theverbbox}

The default mode of the \texttt{mpmlgen} tool is to generate C++ code in the
current directory. All other options change this default behavior. Options which
are not self explanatory are described below.

The \texttt{--skip-translation} option inhibits the generator from writing
generated C++ code to disk. It will still validate and parse the source
document.

The \texttt{--verbose} option is used to print some extra information about the
parsing and code generation process. Using this option a single time will
provide extra information. Using this option twice will provide debugging
information.

The \texttt{--extenions} option is used to generate template source files for
extension classes. This is to be used with extended objects. See section
\ref{sec:extensions} for more information.

The \texttt{--dependencies} option is used by the GNU Autoconf/Automake build
system to properly calculate dependencies when an MPML source file is changed.
You will never need to use this option by hand. Please note that this option is
additive: like GNU GCC, it is used to calculate dependencies at the same time as
compilation. This avoids parsing the file twice.

The \texttt{--database} option is used by the CARMA Monitor Point Database
loader to pre-process the file into a pseudo-MPML format with all common
reference information removed. If you are working on the database, it may be
useful to look at the output of this command.

%%%%%%%%%%%%%%%%%%%%%%%%%%%%%%%%%%%%%%%%%%%%%%%%%%%%%%%%%%%%%%%%%%%%%%%%%%%%%%%%
\section{Reusable Types}

When working with generic hardware or software components, you will often find
it very convenient to use multiple instances of the same C++ object to represent
the monitor stream for the component. By using the same C++ type, you can write
generic code once and reuse it multiple times. This includes usage in several
distinct subsystems, as well as multiple instances within a single subsystem.

A good example of this use case is the CARMA Canbus monitor container. Many
parts of the array's hardware are controlled via canbus. Therefore, defining a
common set of monitor points for all canbus hardware makes the system more
consistent as a whole: once you learn one part, you can apply the knowledge
system wide.

To make this usage of the monitor system as convenient as possible, the
\mpml{Common} mechanism has been provided. This generates C++ classes at the
\texttt{carma::monitor} namespace, and allows these classes to be reused in
multiple places.

%%%%%%%%%%%%%%%%%%%%
\subsection{Defining a reusable type (for multiple subsystems)}

This is easy. Create a seperate MPML source file, and make it look something
like this:

\begin{verbbox}
<?xml version="1.0"?>
<!DOCTYPE Common SYSTEM "mpml.dtd">

<Common scope="global">
    <Container name="TestCommonContainer">
        <description>
            A simple container for use in the MPML handbook examples.
        </description>
        <MonitorPoint type="int" name="insideTCC" />
        <!-- more containers and monitor points go here -->
    </Container>
    <MonitorPoint type="int" name="testCommonMP" />
</Common>
\end{verbbox}
\shadebox{\theverbbox}

Please note that you can \emph{only} reference objects at the top-level scope.
This means that both the \texttt{TestCommonContainer} and \texttt{testCommonMP}
objects can be referenced. The \texttt{insideTCC} monitor point
cannot be referenced: it can only be used as part of the container.

%%%%%%%%%%%%%%%%%%%%
\subsection{Using a reusable type}

This is also quite easy. Wherever you decide to use this file, you must include
it, and then reference the object you wish to use. Like this:

\begin{verbbox}
<?xml version="1.0"?>
<!DOCTYPE Subsystem SYSTEM "mpml.dtd">

<Subsystem name="Test">
    <!-- include the TestCommon.mpml source file -->
    <?common TestCommon.mpml?>

    <!-- create an instance of the TestCommonContainer type -->
    <CommonContainer ref="TestCommonContainer" />

    <!-- create an instance of the testCommonMP monitor point -->
    <CommonMonitorPoint ref="testCommonMP" />
</Subsystem>
\end{verbbox}
\shadebox{\theverbbox}

There are additional advanced uses of the \mpml{CommonContainer} and
\mpml{CommonMonitorPoint} which are described in the reference guide, Section
\ref{subsec:CommonReference}. Please read it.

%%%%%%%%%%%%%%%%%%%%
\subsection{Defining a reusable type (single subsystem only)}

This example uses the \mpml{Common scope="local"} tag, described in Section
\ref{subsec:common-tag}. It generates the reusable C++ classes at subsystem
scope, meaning they are only useful within this subsystem.

The reason this feature was implemented was to allow several subsystems to have
reusable objects with the same name. It has not yet been used in the CARMA MPML
source code.

\begin{verbbox}
<?xml version="1.0"?>
<!DOCTYPE Subsystem SYSTEM "mpml.dtd">

<Subsystem name="Test">
    <Common scope="local">
        <Container name="MyLocalContainer">
            <!-- containers and monitor points here -->
        </Container>
    </Common>

    <!-- create two instances of the MyLocalContainer container -->
    <CommonContainer ref="MyLocalContainer" name="InstanceA" />
    <CommonContainer ref="MyLocalContainer" name="InstanceB" />
</Subsystem>
\end{verbbox}
\shadebox{\theverbbox}

You can see that the \texttt{MyLocalContainer} object was used several times.
Please see Section \ref{subsec:CommonReference} for more details on how to use
the \mpml{CommonContainer} tag.

%%%%%%%%%%%%%%%%%%%%%%%%%%%%%%%%%%%%%%%%%%%%%%%%%%%%%%%%%%%%%%%%%%%%%%%%%%%%%%%%
\section{Extending Automatically Generated C++ Classes}
\label{sec:extensions}

Sometimes it is very convenient to provide extra methods for users of a
container or subsystem object. The purpose is to make the object easier to use
by providing extra convenience methods that are specific to the class.

Please see the \texttt{TestSubsystem.mpml} or \texttt{ControlSubsystem.mpml}
source files for examples. You will want to look in both the source and build
trees for the generated files. Extended objects require both generated code
(part of the build tree) and extension code (part of the source tree). The
extension code is named with an \texttt{Ext.h} or \texttt{Ext.cc} suffix.

Extension code templates can be generated with the \texttt{mpmlgen -e} option.
These can then be customized by hand to add any desired functions or member
variables.

%%%%%%%%%%%%%%%%%%%%%%%%%%%%%%%%%%%%%%%%%%%%%%%%%%%%%%%%%%%%%%%%%%%%%%%%%%%%%%%%
\section{MPML Reference Guide}

The best source of information about allowed parameters and their results will
come from reading the DTD and the source code of the MPML code generator itself.
However, I will attempt to document each element and its purpose below.

%%%%%%%%%%%%%%%%%%%%
\subsection{The \mpml{Subsystem} Tag}

This is a top-level tag which is used to describe a subsystem. You should put a
single subsystem into each MPML source file.

The C++ class describing the subsystem will be generated into the
\texttt{carma::monitor} namespace. All children classes will be generated into
the \texttt{carma::monitor::NameSubsystem} namespace.

It can contain all other tags, including:

\begin{pitemize}
\item \mpml{MonitorPoint}
\item \mpml{Container}
\item \mpml{CommonContainer}
\item \mpml{CommonMonitorPoint}
\item \mpml{Common scope="local"}
\item \mpml{description}
\item \mpml{shortName}
\item \mpml{longName}
\end{pitemize}

And supports the following attributes:

\begin{pitemize}
\item \attr{name}
\item \attr{priority}
\item \attr{author}
\item \attr{count}
\item \attr{persistent}
\item \attr{indexName}
\item \attr{extendFrom="Base"}
\end{pitemize}

%%%%%%%%%%%%%%%%%%%%
\subsection{The \mpml{Common} Tag}
\label{subsec:common-tag}

This is a top-level tag which is used to describe a set of reusable monitor
points and containers. Please take special notice of the \attr{scope}\
attribute, described below.

Children will be generated into C++ classes which can be reused in multiple
subsystems, as well as other common source files. The generator will produce C++
classes in the \texttt{carma::monitor} namespace.

It can contain the following tags:

\begin{pitemize}
\item \mpml{MonitorPoint}
\item \mpml{Container}
\item \mpml{CommonContainer}
\item \mpml{CommonMonitorPoint}
\end{pitemize}

And supports the following attributes:

\begin{pitemize}
\item \attr{scope}
\end{pitemize}

The \attr{scope}\ attribute demands special attention. The two allowed values
are \texttt{global} and \texttt{local}. When used as a top-level container (the
most frequent usage), use \texttt{scope="global"}.

The \mpml{Common} tag also supports a special use case by setting the
\texttt{scope="local"} attribute. This use is \emph{only} allowed directly
beneath a \mpml{Subsystem} tag. This will cause the MPML generator to create C++
classes at the Subsystem namespace. These classes can be reused several times
within the subsystem using the normal mechanisms.

It should also be noted that \mpml{Common scope="local"} classes will override
\mpml{Common scope="global"} if both have the same name. This follows the usual
C++ practice of shadowing variables and classes.

%%%%%%%%%%%%%%%%%%%%
\subsection{The \mpml{Container} Tag}

A container is a logical grouping of monitor points. It will usually be used to
describe a collection of monitor points such as those for a single piece of
hardware.

Each container is a specialization of the
\texttt{carma::monitor::MonitorContainer} class.

It can contain the following tags:

\begin{pitemize}
\item \mpml{MonitorPoint}
\item \mpml{Container}
\item \mpml{CommonContainer}
\item \mpml{CommonMonitorPoint}
\item \mpml{description}
\item \mpml{shortName}
\item \mpml{longName}
\end{pitemize}

And supports the following attributes:

\begin{pitemize}
\item \attr{name}
\item \attr{priority}
\item \attr{count}
\item \attr{persistent}
\item \attr{indexName}
\item \attr{extendFrom="Base"}
\end{pitemize}

%%%%%%%%%%%%%%%%%%%%
\subsection{The \mpml{MonitorPoint} Tag}

A MonitorPoint describes a sensor value. This may come from hardware (such as a
temperature sensor) or software (such as a processing time in the pipeline).

It can contain the following tags:

\begin{pitemize}
\item \mpml{description}
\item \mpml{shortName}
\item \mpml{longName}
\item \mpml{units}
\item \mpml{width}
\item \mpml{errHi}
\item \mpml{errLo}
\item \mpml{warnHi}
\item \mpml{warnLo}
\item \mpml{precision}
\item \mpml{integrate}
\item \mpml{enum}
\end{pitemize}

The \mpml{errHi}, \mpml{errLo}, \mpml{warnHi}, and \mpml{warnLo} tags control
the error thresholds used by the Fault System and RTD displays. They can be
given multiple times for \mpml{MonitorPoint type="enum"}. This is useful because
it allows several distinct enumeration values to be error states.

See Section \ref{subsec:enum-tag} for information on the \mpml{enum} tag.

And supports the following attributes:

\begin{pitemize}
\item \attr{name}
\item \attr{type}
\item \attr{priority}
\item \attr{author}
\item \attr{count}
\item \attr{sampling}
\item \attr{persistent}
\item \attr{update}
\item \attr{spectrum}
\end{pitemize}

The \attr{type} attribute (required) is very special. It defines the underlying
class to use for the C++ object. These classes come from the
\texttt{monitorPointSpecializations.h} header file. Valid values are:
\texttt{char}, \texttt{byte}, \texttt{short}, \texttt{int}, \texttt{bool},
\texttt{float}, \texttt{double}, \texttt{complex}, \texttt{string},
\texttt{absTime}, \texttt{serialNo}, and \texttt{enum}.

The \attr{sampling} attribute is used to determine how many samples this monitor
point shall carry. Most monitor points only carry a single sample, however some
monitor points may have several readings per 500ms monitor interval. This
attribute supports such a use case.

The \attr{update} attribute is used to describe how often the monitor point is
updated. The value is interpreted as the number of 500ms monitor frames. When
setting this value to any number greater than one, you \emph{must} also set the
\attr{persistent} attribute to \texttt{true}.

The \attr{spectrum} attribute is used to show that the monitor point is a
spectrum, rather than the default of a time series. Look at the
\texttt{MonitorPoint.h} code for more information.

%%%%%%%%%%%%%%%%%%%%
\subsection{\mpml{CommonContainer} and \mpml{CommonMonitorPoint}}
\label{subsec:CommonReference}

These tags are almost exactly identical, except that they are used to reference
a common Container or MonitorPoint respectively.

All common sources will be checked to make sure that one of them contains the
referenced object. This includes both local and global (external) commons.

This element supports the following attributes:

\begin{pitemize}
\item \attr{ref}
\item \attr{name}
\item \attr{count}
\end{pitemize}

The \attr{ref} attribute (required) must contain the name of the object you wish
to reference. For example, the timestamp monitor point (part of
\texttt{TimestampCommon.mpml}) is referenced like so:

\begin{verbbox}
<CommonMonitorPoint ref="timestamp" />
\end{verbbox}
\shadebox{\theverbbox}

The \attr{name} attribute (optional) must contain the name as you wish the
object to show up in the monitor point hierarchy. This can be used to create
multiple instances of the same common class.

\begin{verbbox}
<CommonMonitorPoint ref="timestamp" name="timestamp1" />
<CommonMonitorPoint ref="timestamp" name="timestamp2" />
\end{verbbox}
\shadebox{\theverbbox}

The \attr{count} attribute (optional) must contain the number of monitor points
you wish to be used. This works exactly as the count attribute does on regular
monitor points. It is provided to give you counted-renamed instance semantics.
The following example will create the monitor points
\texttt{X.myTimestamp[1-4]}.

\begin{verbbox}
<CommonMonitorPoint ref="timestamp" name="myTimestamp" count="4" />
\end{verbbox}
\shadebox{\theverbbox}

%%%%%%%%%%%%%%%%%%%%
\subsection{The \mpml{enum} Tag}
\label{subsec:enum-tag}

This tag is only useful within a \mpml{MonitorPoint type="enum"} tag. It
describes the individual enumeration values that are created for the monitor
point.

It can contain the following tags:

\begin{pitemize}
\item \mpml{description}
\end{pitemize}

And supports the following attributes:

\begin{pitemize}
\item \attr{name}
\end{pitemize}

%%%%%%%%%%%%%%%%%%%%
\subsection{Widely Used Text-Only Elements}

There are many types of elements which only contain text. This text will be
substituted directly into the C++ code, usually as a \texttt{std::string}. It is
not parsed in any way.

The names of the elements should describe their use.

\begin{pitemize}
\item \mpml{description}
\item \mpml{shortName}
\item \mpml{longName}
\item \mpml{errHi}
\item \mpml{errLo}
\item \mpml{warnHi}
\item \mpml{warnLo}
\item \mpml{units}
\item \mpml{width}
\item \mpml{integrate}
\item \mpml{precision}
\end{pitemize}

%%%%%%%%%%%%%%%%%%%%
\subsection{Widely Used Attributes}

There are several attributes which are used for many different MPML elements.
These are all described together since they have identical uses on each
different object.

The \attr{name} attribute is used to give a name to a subsystem, container, or
monitor point. It is used as part of the canonical name and C++ accessor
functions.

The \attr{priority} attribute is used to set the priority for a subsystem,
container, or monitor point. When used on a subsystem or container, it applies
recursively to all objects beneath it in the hierarchy. Acceptable values are:
\texttt{vital}, \texttt{useful}, \texttt{normal}, \texttt{debug},
\texttt{verbose}, \texttt{default}, \texttt{dontarchive}.

The \attr{author} attribute is used to set the author for a subsystem or monitor
point. It is only useful for data description purposes.

The \attr{count} attribute is used to provide array semantics to an object. It
will create several objects, each with the name \texttt{Name[1-N]}.

The \attr{indexName} attribute is used to change the index variable and accessor
name in the generated class. The default is to create a variable named
\texttt{index} and a function \texttt{getIndex()}. It is only useful with the
\attr{count} attribute set to a number greater than one.

The \attr{persistent} attribute is used to make subsystem, container, or monitor
point persistent. This means that it will not be marked bad even if it has not
been updated recently. Valid values are \texttt{true} and \texttt{false}.

The \attr{extendFrom} attribute is used to create extension classes which can be
modified by the user. See Section \ref{sec:extensions}, which describes how to
use this attribute.  The only valid value is \texttt{Base}.

%%%%%%%%%%%%%%%%%%%%
\subsection{Including External Common Source Files}

This is done using XML processing instructions. To include an external source of
common monitor points (a file beginning with \mpml{Common scope="global"}), use
the following syntax:

\begin{verbbox}
<?common Filename.mpml?>
\end{verbbox}
\shadebox{\theverbbox}

%%%%%%%%%%%%%%%%%%%%%%%%%%%%%%%%%%%%%%%%%%%%%%%%%%%%%%%%%%%%%%%%%%%%%%%%%%%%%%%%
\section{Overview of the MPML Generator Code}

This is intended as a very quick guide to hacking the MPML generator code
itself. It is mostly a braindump by the author, as this documentation is being
written. When in doubt, please consult the source code.

%%%%%%%%%%%%%%%%%%%%
\subsection{Cheetah Templates}

The files \texttt{database.tmpl}, \texttt{header.tmpl}, and \texttt{source.tmpl}
are Cheetah templates. These have placeholders which are filled in using Python
variables data structures.

It should be fairly simple to read the code once you have an understanding of
the Cheetah template language. Basically, it is a text substitution language.
Anything not prefixed with a dollar sign is raw text that will appear in the
output.

See the Cheetah documentation for more information:\\
\url{http://cheetahtemplate.sourceforge.net/docs/users_guide_html/users_guide.html}

%%%%%%%%%%%%%%%%%%%%
\subsection{MPML Utilities}

The file \texttt{MPMLUtils.py} is the Python source code used for some extra
utility functions. These are mostly used by the Cheetah templates, but are also
used in a few other places in the code.

%%%%%%%%%%%%%%%%%%%%
\subsection{DTD Validation}

The file \texttt{DTDValidator.py} is the Python source code used for DTD
validation. This requires the use of the XMLProc XML parser. The Python standard
library XML functions use the expat parser, which does not support DTD
validation.

%%%%%%%%%%%%%%%%%%%%
\subsection{SAX Internal Representation}

The file \texttt{SAXIR.py} provides the internal representation (IR) as used by
the XML parser. It also houses the XML parser itself, as well as some other
classes which are closely related to the XML parser.

The \texttt{MPMLHandler} class is what actually handles parsing an MPML source
file. This should be familiar to you if you have ever programmed an XML parser
using the SAX API.

The \texttt{CommonReferenceResolver} class is used to resolve all of the
\mpml{CommonContainer} and \mpml{CommonMonitorPoint} objects for a single
instance of the \texttt{MPMLHandler} class. This means parsing all of the
external common include files, and finding the referenced monitor points
themselves. Care has been taken to allow circular common inclusions.

The \texttt{MPStorageCounter} class is used to calculate the number of monitor
points and samples contained within a subsystem. This information is used in the
generated C++ code.

%%%%%%%%%%%%%%%%%%%%
\subsection{The Main Program}

The file \texttt{Generator.py} contains the main program. This includes the
option parser, as well as the rest of the glue connecting the other pieces
together.

The \texttt{MPMLRenderer} class is used to actually turn an MPML source file
into an output file. This includes all types of output: dependency information,
database pseudo-MPML, and C++ source code.

The \texttt{Generator} class is used to take a file and generate all requested
output (if any). An instance of this class is created for each input file
specified.

The \texttt{main()} function is where the option parser is created and invoked.
Afterwards, the \texttt{Generator} class is invoked as necessary.

\end{document}

% vim: set ts=4 sts=4 sw=4 expandtab nocindent noautoindent textwidth=80:
