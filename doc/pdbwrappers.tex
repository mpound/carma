%$Id: pdbwrappers.tex,v 1.7 2007/08/14 21:35:11 mpound Exp $
\documentclass[preprint]{aastex} % AASTeXv5.0

\begin{document}
\title{Using the Project Database}
\author{Douglas Friedel, Marc Pound, Athol Kemball}
\slugcomment{Version 1.6\ \ August 08, 2007}



\section{What is the Project Database and Why Do We Need it?}

CARMA will observe many projects over its lifetime.  In any given array
configuration, there may be as many as 100 projects assigned observing time.
Keeping track of what has been observed, how many times, for how long,
and at what quality is a major time-consuming task (as observers who have
had to keep multiple spreadsheets up-to-date are well aware).  In order
to maintain easy operations, we must organize and track the progress
of scientific projects through the telescope system in as automated way
as possible.  This principal mechanism for doing so is the Project Database.

Essentially all the information needed to accomplish this tracking
is available in the initial observing proposal and in the observations
themselves. All accepted proposals automatically get entered into the
Project Database, so an unobserved project entry already has lots
of information on obsblocks to be observed, allocated time, source, PI, etc. 
The science data flow pipeline (AstroHeaderWriter, sdpFiller, quality)
will extract post-observation information such as how much time the
observation got, over what hour angle range, how much time remains,
and the automated quality grade.  The observing script itself as well
as the PI source catalog are also
stored in the Project Database as an exact record of what the array was
commanded to do.  The observer then just has to input the final grade
and any relevant comments on the observation.

Once this information is in the database, it is available to the observer
by python queries that can be used to guide which projects should be
observed next, which still need grades, how much more time a project
needs etc.  This simplifies the observer's life.

We have created some custom queries in {\tt pdbWrappers.py} that observers
will use to interact with the project database (and can add more queries
as needs arise).  The project database wrappers are imported into
subarrayCommands, so you have all these available at the ``sac'' prompt.
In the sections below the sac input line (denoted with Sci$>$) is followed
by an example output from the system.


\section{Run a project}

Using run({\tt myscript}) to observe a project is no different from before.
All changes have been encapsulated in the {\tt newProject} command and
in {\tt obsdef*.py}.  
There a limitation now, however, that only projects with project codes
that are known to the project database may be run. The newProject()
command will validate project before observing can begin. The level
of restriction depends on whether it is a science track or commissioning
track: 
\begin{itemize}
\item {\bf TAC-approved science tracks}.  The full obsblockID 
(project.obsblock.subobsblock) must be known to the project database.

\item {\bf Special projects}.  These are projects with one
of the special project codes:
    \begin{itemize}
    \item ``ctNNN" commissioning projects
    \item ``csNNN" -- CARMA Summer school
    \item ``base"  -- baseline
    \item ``fringe" -- fringe test
    \item ``flux"  -- flux observations
    \item ``none" --  unprocessed integrations
    \item ``opnt" --  optical pointing
    \item ``rpnt" --  radio pointing
    \item ``test" --  generic test
    \item ``tilt" --  tilts
    \end{itemize}
\noindent 
(The project code
part of the obsblockID before the first dot, e.g. for c0014.OphB11\_C
the project code is c0014.)
For observations with these project codes you are free
to choose any obsblock or subObsblock name.  If they are previously
unknown to the project database, they will be added.  Note that
{\bf \emph{arbitrary project codes are no longer allowed}}; 
for test observations 
which don't have a commissioning code you can use the code ``test''.
\end{itemize}

Trial numbers are now assigned automatically
by {\tt newProject()}; it will ignore any trial number you pass in 
(that parameter has been kept only so that old scripts won't break).
{\tt newProject()} will also automatically add the running script
to the project database as a record of what was run for that trial.

\section{Check the status of a project}
To check the status (i.e. how much time has been observed) of all obsblocks in a project (e.g. c0014):
\begin{verbatim}
Sci> projectStatus(``c0014'')

Project c0014 has 2 obsblocks 
1: Obsblock OphB11_C is INCOMPLETE
    Observed/Remaining hours: 1.00/3.50
    Trial 1 is COMPLETE. 
    Trial 2 is RUNNING. 
    Trial 3 is RUNNING. 
2: Obsblock OphB11_D is INCOMPLETE
    Observed/Remaining hours: 0.00/3.50
    Trial 1 is INCOMPLETE. 
\end{verbatim}

\section{Grade a trial}
To grade trial 1 of obsblock c0014.OphB11\_C:
\begin{verbatim}
Sci> gradeProject("c0014.OphB11_C",1,"A","Marc","This trial is finished")

Marc gave c0014.OphB11_C.1 a grade of A.
\end{verbatim}
\noindent Arguments are as follows:\\
{\bf obsblockID} - the fully-qualified obsblockId, without the trial number.\\
{\bf trialNo} - trial number\\
{\bf name} - the name of the person making this edit. No default.\\
{\bf grade} - String value indicating the overall grade from the observer. (e.g. 'A','C-', etc.). No default.  This string gets converted to a numeric grade inside of the project database manager.  If you wish to insert a particular numeric grade, then place it in quotes, e.g.  '61.3'.\\
{\bf comment} - any comment from observer about grade. Defaults to empty; above name above name and current date will be prepended to the comment.\\
{\bf length} - the length of time the track took. Default is -1.0 which means the length calculated by quality should be used. This parameter is usually used to pass partial tracks.

\section{Modify comments on a trial}
To show the comments for a given obsblock and trial:
\begin{verbatim}
Sci> getComments("c0014.OphB11_C",1)
Trial 1 : Marc Pound : ( Mon Jul  2 17:42:41 2007 UT ) These data are transplendant! 
\end{verbatim}
\noindent The output shows who wrote the comment and when. Arguments are as follows:\\
{\bf obsblockID} - the fully-qualified obsblockId, without the trial number.\\
{\bf trialNo} - trial number. Default is -1 which means show comments for all trials.

To add comments to the existing comments for the given obsblock and trial:
\begin{verbatim}
Sci> addComments("c0014.OphB11_C",2,"Marc Pound", "I have even more to say about this trial")
\end{verbatim}
\noindent Arguments are as follows:\\
{\bf obsblockID} - the fully-qualified obsblockId, without the trial number.\\
{\bf trialNo} - trial number\\
{\bf name} - the name of the person making this edit. No default.\\
{\bf comment} - The comment to add.  Above name and current date will be prepended to the comment.  

To change existing comments (you should have a very good reason for doing this!):
\begin{verbatim}
Sci> replaceComments("c0014.OphB11_C",2,"Marc Pound", "I made a mistake before")
\end{verbatim}

To delete existing comments (should be done only in very rare circumstances!):
\begin{verbatim}
Sci> deleteComments("c0014.OphB11_C", 2)
\end{verbatim}


\section{Which projects can be run during a given LST time range}
To find out which incomplete 1mm projects can be run between 16:00 and 21:00 LST
\begin{verbatim}
Sci> projectsWithinRange(16,21,band="1mm")

Found a total of 1 projects with 2 obsblocks matching your query.
Sorted by PRIORITY:
      ObsblockID       Band    Hrs_Left    RA        LST_Range    Flex?
          c0089.g19     1MM      8.00     18:27   ( 01:18, 08:21 )  N
    c0089.iras16293     1MM      6.00     16:31   ( 01:59, 06:39 )  N

\end{verbatim}
\noindent  The query returns the available obsblocks, how much more observing
time they require (Hrs\_Left), the LST range over which they can be observed,
and whether or not the PI has indicated flexible hour angle scheduling.
By default the returned list is sorted in order of priority assigned by
the TAC with highest priority listed first.  
The lstStart (16) and lstStop (21) arguments can take either decimal hours (i.e 16.5) or sexagesimal (HH:MM(:SS)) arguments, however sexagesimal arguments must be encased in quotes. Note that LST time ranges can wrap around 24:00 (i.e. lstStart=22.5 lstStop=4.0 is valid)

\noindent Optional arguments to refine the search are:\\
{\bf minHours} - Only show projects with at least this many hours left to observe. Default 2.\\
{\bf flexHA} - True or False indicating whether to include sources with 'flexible hour angle' scheduling or not. Default: True\\
{\bf band} - Only return projects for the given Rx band. One of "3mm", "1mm", "1cm", "any". Case-insensitive. Default: any\\
{\bf arrayConfig} - Only return projects for the specific array configuration.
One of "A", "B", "C", "D", "E", or "any".  Case insensitive. Default: any\\
{\bf sort} - How to sort the output. One of "priority", "lst", "timeleft", "none".  Default is "priority".  Case insensitive.

\section{Which projects can be run starting at a given LST}
To find which incomplete projects can be run starting at 10 hours LST
that have at least 4 hours left to observe
\begin{verbatim}
Sci> projectsStartingAt(10,4)

Using any Rx band 
Found a total of 9 projects with 17 obsblocks matching your query.
Sorted by PRIORITY:
      ObsblockID       Band    Hrs_Left    RA        LST_Range    Flex?
    c0002.n3627_hcn     3MM     20.00     11:19   ( 21:59, 07:56 )  N
   c0002.n3627_hcop     3MM     20.00     11:19   ( 21:59, 07:56 )  N
        c0003.J0927     3MM     40.00     09:26   ( 21:10, 07:46 )  N
     c0022.weiN3032     3MM      7.20     09:51   ( 20:51, 08:18 )  N
     c0022.weiU6570     3MM      7.80     11:34   ( 21:00, 09:03 )  N
     c0022.weiN3773     3MM      6.20     11:37   ( 22:07, 07:57 )  N
     c0022.weiN3011     3MM      7.40     09:48   ( 20:41, 08:26 )  N
     c0022.weiU6003     3MM      6.00     10:52   ( 22:17, 07:23 )  N
     c0022.weiU6805     3MM      8.00     11:49   ( 20:41, 09:29 )  N
        c0033.oj287     3MM     12.00     08:53   ( 21:01, 07:37 )  N
        c0035.n3627     3MM     10.00     11:19   ( 21:59, 07:56 )  N
        c0035.n4736     3MM     13.00     12:49   ( 21:00, 09:42 )  N
  c0037.frayer_gn26     3MM     42.00     12:35   ( 19:13, 11:22 )  Y
         c0048.umon     3MM      5.00     07:29   ( 22:17, 05:38 )  Y
        c0051.sexb1     3MM     16.00     09:56   ( 22:00, 07:12 )  Y
        c0051.sexb2     3MM     16.00     09:56   ( 22:00, 07:12 )  Y
     c0071.COSMOS-1     3MM     31.00     09:59   ( 22:11, 07:02 )  Y


\end{verbatim}
\noindent The lstStart (10) argument can take either decimal hours (i.e 16.5) or sexagesimal (HH:MM(:SS)) arguments, however sexagesimal arguments must be encased in quotes.

\noindent Optional arguments to refine the search are:\\
{\bf minHours} - Only show projects with at least this many hours left to observe. Default 2.\\
{\bf haLimit} - Only show projects with that will be within hour angle $\pm$ haLimit at the LST start time. Default 3.\\
{\bf band} - Only return projects for the given Rx band. One of "3mm", "1mm", "1cm", "any". Case-insensitive. Default: any\\
{\bf arrayConfig} - Only return projects for the specific array configuration.
One of "A", "B", "C", "D", "E", or "any".  Case insensitive. Default: any\\
{\bf sort} - How to sort the output. One of "priority", "lst", "timeleft", "none".  Default is "priority".  Case insensitive.

\section{Validate an Obsblock}
To check that an obsblockID exists in the project database or is a
commissioning/special project (True means valid, False means invalid/not found):  
\begin{verbatim}
Sci> isValidObsblock("c0014.OphB11_C")
True
\end{verbatim}
\noindent This command is called automatically by {\tt newProject()}.

\section{Generate a schedule}
The project database can be used to generate the observing schedule.
This is done by querying the project database and then calling
the {\tt mksched} program with the results.
To generate a 10-day schedule for the 3mm C array:
\begin{verbatim}
Sci1> schedule("C",band="3mm",run=True,days=10)
# Using Rx band 3MM
# Using C array configuration
# Found a total of 41 projects with 127 obsblocks matching your query.
# Sorted by PRIORITY:
mksched INPUT file written to /tmp/mksched.in.2007-08-02_15_49.
Running mksched...
/opt/rt/bin/mksched file=/tmp/mksched.in.2007-08-02_15_49 days=10 config=C start=2007-Aug-2 fixed=none pivot=0.000000 > /tmp/mksched.out.2007-08-02_15_49
mksched OUTPUT file written to /tmp/mksched.out.2007-08-02_15_49.

\end{verbatim}
\noindent The argument {\tt arrayConfig} specifies which array
configuration you are trying to schdule. It is one of "A", "B", "C", "D", "E"
(case insensitive, no default); Note "any" is NOT an option here. 
\noindent Optional arguments to refine the search are:\\
{\bf minHours} - Only show projects with at least this many hours left to observe. Default 4.\\
{\bf band} - Only return projects for the given Rx band. One of "3mm", "1mm", "1cm", "any". Case-insensitive. Default: any\\
{\bf sort} - How to sort the output. One of "priority", "lst", "timeleft", "none".  Default is "priority".  Case insensitive.
The following parameters map directly to inputs to mksched. Type
'mksched --desc' or 'mksched --key' for more info on that program
{\bf run} - Whether or not to run the mksched program on the output file. Default False. \\
{\bf days} -  The maximum number of days to schedule, 0 means no limit. Default: 10 \\
{\bf fixed} -  File name for fixed blocks from say a previous run.  These will be added to the schedule BEFORE any new blocks are scheduled.  Default=none which means no fixed blocks. \\
{\bf pivot} -  Grade below which all projects are scheduled in COMPACT mode. Default 0.



\section{Adding a script to the project database}
To add an observing script and its assocaited catalog to obsblock
c0014.OphB11\_C in the database from file /tmp/myScript (script) and
/tmp/myCatalog (catalog):
\begin{verbatim}
Sci> addScript("c0014.OphB11_C",script="/tmp/myScript",catalog="/tmp/myCatalog")

Successfully loaded /tmp/myScript into database for c0014.OphB11_C 
\end{verbatim}
\noindent Note that the obsblock ID is the full qualified ID project.obsblock[.subobsblock] without the trial number (the script gets put in the last trial). If the file name does not begin with a ``/'' then the same directories searched by run() will be searched. Also note that you can individually add a script or a catalog to the database by using script= or catalog=.
This command is called automatically by newProject if the subarray scriptName variable has been previously (e.g. if {\tt s.getScriptName()} returns something other than ``None''.)

\section{Retrieving a script from the project database}
To retrieve a script and catalog for obsblock c0014.OphB11\_C from the project database and put them into /tmp/newScript (script) and /tmp/newCatalog (catalog):
\begin{verbatim}
Sci> findScript("c0014.OphB11_C",script="/tmp/newScript.obs",catalog="/tmp/newCatalog")

Found script, saving to file /tmp/newScript.obs
Found catalog, saving to file /tmp/newCatalog
\end{verbatim}
\noindent Note that the obsblock ID is the fully-qualified ID project.obsblock[.subobsblock] without the trial number. Also note that, as with addScript, you can individually retrieve a script or a catalog from the database by using script= or catalog=.

\noindent Optional argument:\\
{\bf trialNo} - the trial number for which to get the script. Default is -1 which means return last INCOMPLETE trial number.\\
{\bf verbose} - If True, report on locating/saving script. Default is True.


This output means that there are 2 obsblocks in project c0014. Both
obsblocks (OphB11\_C and OphB11\_D) are incomplete\footnote{See
Table~\ref{tab:status} for the definitions of the status labels.}. Obsblock
OphB11\_C has three trials that have been observed, one of which has received
a grade and is complete, and has been credited with 1 out of 3.5 hours of
observation. The others are marked as RUNNING until the observer has graded
them. The only trial for obsblock OphB11\_D has not been observed yet.

\begin{deluxetable}{ll}
\tablecolumns{2}
\tablewidth{0pt}
\tablecaption{Meanings of Status Labels}
\tablehead{\colhead{Status} & \colhead{Meaning}}
\startdata
INCOMPLETE & For projects, obsblocks, and subobsblocks, this means that they
still have observable\\
&time left in their allocation. For a trial it means it
has not been observed.\\
COMPLETE & This means that the projects, obsblock, subosblock, and/or trial
has been observed\\
&and the total observed time has met the allocation time
(i.e. they are no \\
&longer observable)\\
RUNNING & For trials this means that they have been (or currently are) being
observed and are\\
& awaiting a grade. This has no meaning for projects, obsblocks
or subobsblocks.\\
\enddata
\label{tab:status}
\end{deluxetable}
\end{document}

