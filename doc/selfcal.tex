% Note: This file uses eps graphics. pdfLatex will not work so you must
% do the following:
% latex <filename.tex>
% dvipdf <filename.dvi>

\documentclass[11pt]{article}

\usepackage{color}

\usepackage{graphicx}
\usepackage{lscape}
\usepackage{epsfig}

% PS fonts
\usepackage{mathptmx,courier,pifont,helvet}
\usepackage[T1]{fontenc}

\addtolength{\textheight}{1.75truein}
\addtolength{\topmargin}{-1truein}
\addtolength{\textwidth}{1.5truein}
\addtolength{\oddsidemargin}{-0.75truein}
\addtolength{\evensidemargin}{-0.75truein}

\begin{document}
\title{\LARGE\bf Online SelfCal}
\author{Peter Teuben~(UMD)}
\date {\it $ $Revision: 1.1 $ $ \\ $ $Date: 2006/02/14 03:27:22 $ $}
\maketitle

%
%
\section{Selfcal}

\subsection{Goal}

Complex antenna based gains, derived from a self-cal solution, on a point
source\footnote{in theory any source for which a good model is known}
are a useful diagnostic to monitor the performance and quality
of the interferometric observations. 
The solutions, as well as their error estimates, are published as monitors
points in the CARMA online system.
Instead of having to plot $N(N-1)/2$ phases and amplitudes, only $N$ need to
be summarized.

\smallskip\noindent
They are derived from a {\it meta-channel}, defined by the observer by choosing
any set of channels (typically a wide-band channel). The gains are computed on
a integration timescale, not on the 1/2 second system beat, and these gains
are also not exported to the (Miriad) scientific data sets. 

\smallskip\noindent
However, by virtue of being in the monitor stream, are available to write
sophisticated real-time routines during observations.

%
\subsection{Theory}
%
%

Following e.g. Cornwell et al we can write the outcome
$$
\tilde{V}_{ij} = g_i g^*_j G_{ij} V_{ij}  + \varepsilon_{ij} + \epsilon_{ij} 
$$
but simplify this to
$$
\tilde{V}_{ij} = g_i g^*_j V_{ij}  + \epsilon_{ij} 
$$


and this is where the magic starts...

For a self-calibration we want to minimize

$$
\sum_k \sum_{i,j} w_{ij} \left\| \tilde{V}_{ij} - g_i g^*_j \hat{V}_{ij} \right\|^2
$$

where $k$ is a sum over time, and $i,j$ over antennae baseline pairs.

where at least for a point source the model is well known.

\subsection{Implementation}

Jacobi iteration (miriad::selfcal) vs. least squares solution (hat::xpnt)
The {\tt carma::services::SelfCal} class implements the MIRIAD selfcal
solution.
\bigskip

Minimizing in $L_1$ vs. $L_2$ norm (Schwab)

\end{document}
