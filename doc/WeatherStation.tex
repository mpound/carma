% this will be the CDR
%
%  

\documentclass[11pt]{article}

% PS fonts
\usepackage{mathptmx,courier,pifont,helvet}
\usepackage[T1]{fontenc}

\addtolength{\textheight}{1.75truein}
\addtolength{\topmargin}{-1truein}
\addtolength{\textwidth}{1.5truein}
\addtolength{\oddsidemargin}{-0.75truein}
\addtolength{\evensidemargin}{-0.75truein}

\newcommand{\method}[2]{
\item{\tt #1}\\[0.2em]
  #2
}

\newcommand{\monpoint}[4]{
\item
  \makebox[50ex][l]{\tt #1}(type: {\it #2}; units: {\it #3})\\[0.2em]
  #4
}

\begin{document}
\begin{center}
{\LARGE\bf Weather Station:\\
(--CDR--)}\\
Peter Teuben \\
Version 2.10\\
Updated: January 3, 2007
\end{center}


\section{Functionality and Scope}

\subsection{Overview} 

The weather station provides the observatory with basic weather
information (temperature, pressure, wind speed etc.), and
publishes these (and some derived) variables  to the
standard CARMA monitor system. Several subsystems need this weather
information for proper operation:

\begin{enumerate}

\item hazardous weather conditions, most notably wind speed. The fault
system will be looking at these.

\item system temperature: the ambient temperature is important to
obtain an accurate value for $T_{sys}$.

\item refraction: it is mostly the pressure that 
determines the amount of refraction (several arcmin at low
elevation), and depending on required accuracy the temperature,
humidity level and observing frequency are also needed.

\item delays: need a {\it good} atmospheric model, and again pressure, 
temperature and to some extent humidity are important here.

\item {\it should add something about required accuracy for
some of these items}
\end{enumerate}

Two identical weatherstations (a Campbell Scientific CR10X datalogger)
have been in operation for a few years, one at SZA and one
to sample data at the CARMA site
(the circular buffer of the datalogger can contain many months of
weather info at 5 minute samples).
We plan to use this as our weatherstation (WS).
In addition, there is a separate 
dew point sensor (DPS) for very accurate measurements of the
ambient and dewpoint temperature (most simple weatherstations
have questionable humidity sensors).  Initial comparisons showed
that the Relative Humidity given by the WS and computed from the DSP
differ by a few percent, and the Dew Point Temperatures can be
off by 1-2 degrees Celsius.

% See also: $CARMA/sza/array/code/vxworks/rtc_src/weather.{h,c}    old DASI code
%                  sza/translator/WxCommunicator and WxControl     current SZA

%Both the WS and DPS firmware make the data available on a serial port.


The opacity monitor (tipper) is part of another package.


% peter<->steve
% > maybe a silly question, but why do we need that high accuracy for the 
% > dewpoint?
% 
% no, it is a good question. we have never had a tipper on site, so we have 
% relied on the local water vapor density to choose projects and it works 
% quite well (along with a phase monitor). the tipper is very old (2nd hand 
% from nrao site tests for alma), and if it breaks the time to repair would be 
% very long, so we would have only the surface measurements to guide us.
% 

\subsection{Measured variables}

\subsubsection{WS variables}
The following variables are read from the weather station, currently
available every 5 minutes on station ID 222.  Every 24 hours 
some averaged quantities are available on station ID 333, which
we will ignore.

\footnotesize
\begin{verbatim}
  WX_STN_ID,     /* The station ID character (222/333) */
  WX_YEAR,       /* The year number (eg. 1998) */
  WX_DAY,        /* The day number (eg. 232) */
  WX_TIME,       /* The time-of-day (eg. 1934 [ie. 19 hours, 34 minutes]) GMT */
  WX_INT_TEMP,   /* The internal temperature of the weather station (C) */
  WX_BATTERY,    /* The battery voltage (V) */
  WX_AIR_TEMP,   /* The outside temperature (C) */
  WX_HUMIDITY,   /* The relative humidity (%) */
  WX_WIND_SPEED, /* The wind speed (m/s) */
  WX_WIND_DIR,   /* The wind direction (degrees East of North) */
  WX_PRESSURE,   /* The atmospheric pressure (mb) */
\end{verbatim}
\normalsize

The WS is actually completely programmable. Depending on the modules one buys with it,
and how the internal panel has been properly wired, various instruments can deliver
variables. Thanks to Erik Leitch, some code is now available by which this can be done
via Linux uploading new firmware. We expect to change the 5 minute information cycle
down to whatever we need.

\subsubsection{DPS variables}

The DPS: returns only the ambient and dew point temperature, both in Fahrenheit.


\subsection{A little weather background}

At low altitudes our atmosphere has roughly an exponential scaleheight 
of 8.3km, thus the barometric pressure, to first order, can be given by
as a function of the altitude as follows:
$$
     P(alt) \approx 1000  ~~~ e^{-alt[m]/8300}  [mbar]
$$
This formulae is actually used in the offline ephemeris if no weather
information is available.
\smallskip
The dew point temperature is defined as the temperature to which the air
would have to cool (at constant pressure and constant water vapor
content) in order to reach saturation. 
\smallskip
The relationship between Ambient temperature, Dew Point temperature and
Relative Humidity have various approximations. We currently use the
version by Hoffman \& Welch from the BIMA code:
$$
        {RH \over 100 }= e^{    5370 ({1\over T_a} - {1\over T_d}) }
$$
where temperatures are in Kelvin. Conversion routines using this relationship
can be found in the class
{\tt carma::environment::Atmosphere}, though alternative formulae
(e.g. Clausius-Clapeyron) are also available.
\footnote{see also {\tt http://www.faqs.org/faqs/meteorology/temp-dewpoint/}}

\subsection{Control Input}

The weather station is an autonomous unit and does not need to be
set into an operational mode. Hence no IDL code is needed.

\subsection{Output}

Software output consists of monitor data and occasional logging and tracing.
The monitor points are described in detail in Section~\ref{monitor_points}.
Logging can occur if the weather station looses power (e.g. low voltage battery)
Also, dangerous weather conditions should set an alarm and possible put the
array into a ``safe mode'', however this is the responsibility of the fault
system.

\subsection{Administrative Summary}
\subsubsection{Estimated FTE Effort}
\begin{tabular}{lc}
Design:& 0.25\\
Implementation:& 0.5\\
Testing and integration:& 0.5\\
& ----\\
TOTAL:& 1.25
\end{tabular}
(numbers are in man-months)
\subsubsection{Schedule}

\begin{tabular}{lll}
Package status:& open&\\
Work Package Analysis: & completed:    &  08/27/2002\\
Conceptual design:& completed: & n/a\\
Preliminary design:& completed:& 02/03/2005\\
Critical design:&     & 05/06/2005\\
\hline
\end{tabular}


\section{Design}

Monitoring functions for the weather station will be realized
through a simple RS232 serial port. New code is needed
for publishing the associated CORBA distributed object (DO) that
external clients contact to activate its control functionality. 
An existing flexible serial port access routine (converted to C++) 
from the WASP project will be used to access the serial port.

% The equivalent SZA code, though elegant, was deemed too intrusive and involved in the current build systems

\subsection{Control API}
\label{control}

The weather station has no controls, 
polling occurs on the RS232 port, though the firmware for
both the DPS and WS can be enabled in an asynchronous mode.


\subsection{Monitor Points}
\label{monitor_points}
The following is the complete set of monitor points in the 
{\tt carma.environment.weatherStation} hierarchy.
All monitor points should be published at a 2~Hz rate.

\begin{itemize}
\monpoint{ambientTemperature}{float}{C}{
Outside ambient temperature. Measured with DPS or WS. 
See also weatherMode below.
}
\monpoint{dewpointTemperature}{float}{C}{
Dewpoint temperature. 
Either measured (DPS) or derived (WS).
See also weatherMode below.
}
\monpoint{humidity}{float}{\%}{
Relative Humidity, between 0 and 100\%. 
Either measured (WS) or derived (DPS).
See also weatherMode below.
}
\monpoint{pressure}{float}{mbar}{
Measured Barometric pressure.
}
\monpoint{windSpeed}{float}{m/s}{
Measured Wind speed.
}
\monpoint{waterDensity}{float}{g/m$^3$}{
Water vapor density
}
\monpoint{precipWater}{float}{mm}{
Derived precipitable water vapor. 
}
\monpoint{batteryVoltage}{float}{V}{
Measured Voltage of the weather station battery. 
}
\monpoint{weatherMode}{integer}{n/a}{
Weather station mode. 1=WS only. 2=DPS only. 3=WS+DPS. 
}



\monpoint{DewpointSensor.ambientTemperature}{float}{C}{
Outside ambient temperature. Measured with DPS.
}
\monpoint{DewpointSensor.dewpointTemperature}{float}{C}{
Dewpoint temperature. Measured with DPS.
}
\monpoint{WeatherStation.ambientTemperature}{float}{C}{
Outside ambient temperature. Measured WS.
}
\monpoint{WeatherStation.humidity}{float}{\%}{
Relative Humidity, between 0 and 100\%. 
Measured WS.
}


\end{itemize}

\section{Dew Point Sensor}

The program {\tt wx} runs on {\tt inyo} (a Sun) and connects
to the WinNT (an old P5-75 Gateway) box and its output
looks as follows:

\footnotesize
\begin{verbatim}
Fast "Snapshot" Data
--------------------------------------------------------
Local OVRO Time/Date:       PST Mon Mar 14 19:30:03 2005
Universal Time/Date:        UTC Tue Mar 15 03:30:03 2005
Modified Julian Date:       53444.14587 days

Temperature, Ambient:         9.1 deg C    48.4 deg F
             Dewpoint:      -16.4 deg C     2.4 deg F

Barometeric Pressure:       880.6 millibars
Water Vapor Pressure:         1.7 millibars
            Density:          1.3 g/m^3

Columnar H2O Content*:        2.6 millimeters
Relative Humidity:           15 %
Wind Speed:                  11.7 mph     5.2 m/sec
     Direction:             311.5 deg    (NW)
     North Vector:            7.8 mph     3.5 m/sec
     East Vector:            -8.8 mph    -3.9 m/sec

*From local H2O density assuming 2 km scale height
--------------------------------------------------------
Current Generator Status

Fault              Warning            Status
(   )Low Oil Press (   )Low Oil Press ( X )Array on Util
(   )High Eng Temp (   )High Eng Temp ( X )Ctrl Bld Util
(   )Overspeed     (   )Low Eng Temp  (   )Generator Run
(   )Overcrank     (   )Low Fuel      (   )Array on Gen
(   )Not in Auto   (   )Fuel Tank Rpt (   )Ctrl Bld Gen
\end{verbatim}
\footnotesize

\section{Implementation}



User-level access to the weather system is provided by a C++ class named
Weather, which contain the WS and DPS classes that directly talk to the
serial port.

\subsection{Code Reuse}

SZA is using the same instrument (two identical systems were baught
in year-????). Erik reported that in theory the firmware can be modified
to have the weatherstation asynchronously send weather info at a
pre-programmed rate (via firmware settings, not something we would
want to control via IDL). This was never working according to
advertised interfaces. However it's working fine by polling the
serial port.

\small\noindent
Some of the old code on the old OVRO WindowsNT3.5 based weather
station box probably has some algorithms for the derived quantities 
that can be added to the current ones in {\tt Atmosphere}.

\small\noindent
WS up/downloading firmware code has been made available by Erik Leitch.


\section{Programs}

This package does not currently supply any programs.When the server runs, it
locks up the serial ports and using even one of the test programs would
result in crashing the server and/or producing bogus numbers.

\noindent
{\tt carma/environment/Test/tAtmosphere} : can be used to
excersize the conversion routines between ambient temperature, dewpoint
temperature and humidity.

\section{Notes}
\begin{itemize}


%% \item dewpoint sensor has not been discussed, it should also be part of this package.

\item battery : it can also be monitored, returns battery voltage. does the new weatherstation have it?

%%\item straight RS232 vs. CAN micro (via an XAC3 module?) Firmware needed would argue RS232
%%is earier ( or use serial-to-ethernet converter, SZA uses a Lantronix RS232-TCP/IP converter)

%%\item 2HZ monitoring on RS232 (9600baud) should be ok.

%%\item conversion of RS232 to fiber to avoid lightning damage to the computer
%%(wouldn't you think these are protected against that?)

%%\item rapid variable wind vs. slowly varying temp/press

%%\item averaging here, and publish as a monitor point, or let the monitor system do this?

%%\item wx200d-1.3 (wx200d.sourceforge.net, \$200; open2300.sourceforge.net)

%%\item differential refraction; override if bad weather info?

%%\item at various stages one linux machine has been identified from which we
%%run several serial ports: phone monitor, tipper, weather station and dewpoint
%%sensor.  If so, we may need a muliport solution, or use serial-to-ethernet
%%converters.

\item 
Paul tried a fiber-to-RS232 converter, which works fine for the DPS, but not for
the WS.

\end{itemize}

%
%\section{References}
%
%{\tt http://www.campbellsci.com}

% Serial Programming Guide for POSIX Operating Systems
% http://www.easysw.com/~mike/serial/serial.html

% http://parker.uchicago.edu/sza/peripherals/weather_station.html
% needs a username/password to get access

\section*{Acknowledgements}

The author wishes to thank Curt G, Steve Scott, Paul Daniel and Jim F(?) 
for help during the initial hardware integration and Chul Gwon for
patiently explaining some CORBA internals for the examples.


% \appendix
\newpage
\section{Appendix A: Weather Station}

The weather station is a CRX10 datalogger from
Campbell Scientific\footnote{{\tt http://www.campbellsci.com}}
consisting of the following: a 3 meter tall UT3 tower, a Vaisala HMP35 temp and RH probe with 
solar radiation shield, a Vaisala CS105, PTB101B Barometer (600 to 1060 
mbar), a 05103 RM Young Wind Monitor, a CR10X-1M Datalogger with 1Meg 
memory module and wiring panel, a 12 volt battery power supply with 
charge regulator, an MSX10 10 watt solar panel, and an ENC 16/18 
Enclosure 16" x 18". A SC32B optical isolator is needed to converted
the non-standard 9pin RS232 signal to a regular serial line, to enable
the ``normal'' communication modes used on PCs.

\subsection{Wiring}

Currently a complex change through the required optical isolator to
various forms of serial cables (9 pin, 25 pine, rj11).

\subsection{Debugging}

It is easy on serial lines to get your wires crossed, no pun intended. DCE, DTE, and 
remember RS232 cables are different from telecon RJ11(?) cables, and they've been
used in various locations. See pictures on the 
weatherstation page\footnote{current {\tt http://www.astro.umd.edu/\~ teuben/carma/weather}}

\subsubsection{USB serial}

Once the {\tt environment} has booted,  the modules
{\tt keyspan} and {\tt usbserial} should have been loaded, and via
for example {\tt dmesg} you should see sections such as:


\footnotesize
\begin{verbatim}
...
usbserial.c: USB Serial support registered for Generic
usbserial.c: USB Serial Driver core v1.4
usbserial.c: USB Serial support registered for Keyspan - (without firmware)
usbserial.c: USB Serial support registered for Keyspan 1 port adapter
usbserial.c: USB Serial support registered for Keyspan 2 port adapter
usbserial.c: USB Serial support registered for Keyspan 4 port adapter
usbserial.c: Keyspan 4 port adapter converter detected
usbserial.c: Keyspan 4 port adapter converter now attached to ttyUSB0 (or usb/tts/0 for devfs)
usbserial.c: Keyspan 4 port adapter converter now attached to ttyUSB1 (or usb/tts/1 for devfs)
usbserial.c: Keyspan 4 port adapter converter now attached to ttyUSB2 (or usb/tts/2 for devfs)
usbserial.c: Keyspan 4 port adapter converter now attached to ttyUSB3 (or usb/tts/3 for devfs)
...
keyspan.c: v1.1.4:Keyspan USB to Serial Converter Driver

\end{verbatim}
\footnotesize


\subsubsection{minicom}

\footnotesize
\begin{verbatim}
   Serial line:    9600 8N1: 9600 baud, 8 bit, no parity, 1 stop bit

   Set them manually in the file /etc/minirc.ws

# Machine-generated file - use "minicom -s" to change parameters.
pr port             /dev/ttyUSB0
pr bits             8
pr parity           N
pr stopbits         1
pu baudrate         9600
pu minit            
pu mreset           

   after which the command will be:


   % minicom ws

   assuming of course it has been stuck in the KEYSPAN #1 port!

   You should (perhaps after hitting the RETURN key a few times) see a * prompt.
   For debugging, the most important commands are:
      A   get the current status and memory counters
      B   backup a record
      C   get the time
      D   dump a record
   

   *C
   Y05 D0077  T19:06:18 C1294
   *A
   R+245610. F+573441. V3 A1 L+245610. E99 79 99 M1280   B+3.2132 C3336
   *B
   A1 L+245598 C0806
   *D
   01+0222.  02+2005.  03+0077.  04+1905.  05+17.75  06+18.87  07+0874.  08+2.854  
   09+1.098  10+164.2  11+40.66  12+0.000                                          
                                                                                
   A1 L+245610 C6312 
   *
 

   ??   should we see Online of Offline in minicom's status line at the botttom now ??

   !!   this is an example of a good data record. The 'C' time (19:06:18) is less than
        5 minutes from the 4th data item in the 'D' line  (1905)
\end{verbatim}
\footnotesize

\subsubsection{Command Mode}

The weather station program actually runs every 3 seconds and puts the data into
intermediate storage locations. To get to the intermediate
storage locations, use minicom and do a carriage return or 2 to get the 
familiar * prompt. Then
issue a 7H (enter) to put the weather station into command mode. This
can be verified by the prompt changing to a $>$. Then issue a *6. The
weather station will respond with, MODE 06:0000. Use, Enter, to step
through the intermediate locations or use, \#N, where N is the number of
the desired intermediate location. There are 28 intermediate locations
of which only 19 are used by this program. The ones of interest to you
are: 01 (Battery Volts, weather station power supply), 03 (Wind Speed in
miles per hour), 04 (Wind Direction in degrees), 05 (Barometric Pressure
in mbar), 06 (Air Temperature in degrees C), 07 (Relative Humidity in
percent), 08 (Vapor Pressure of water in mbar), 09 (Saturation Vapor
Pressure of water at the current air temperature in mbar), and 14 (Water
Vapor Density in grams per cubic meter). The other intermediate
locations used by this program are for intermediate calculation storage.

\footnotesize
\begin{verbatim}

*
*
*7H                                                                                                                                        
                                                                                                                                           
>                                                                                                                                          
MODE 06:0000
01:+13.901   voltage
02:+4145.5   
03:+3.5803   wind speed (mph)
04:+299.73   wind dir (deg)
05:+786.43   pressure (mbar)
06:+8.4432   air temp (C)
07:+18.851   rel humidity (%)                                                                                                                  
08:+1.1048   vapor pressure (mbar)
09:+11.048   saturdation vapor pressure
10:+.18851                                                                                                                                 
11:+2.0914
12:+281.61
13:+453.18
14:+1.6092   water vapor density (g/m3)
15:+.52184
16:+2.0000
17:+.37276
18:+.09099
19:+.08613
20:+0.0000
21:+0.0000
...
          
\end{verbatim}
\footnotesize

\subsubsection{tWS}

The test program {\tt carma/environment/Test/tWS} can be used to debug communications:

\footnotesize
\begin{verbatim}

   % cd $CARMA_BUILD
   % carma/environment/Test/tWS repeat=5
WS:   i=5 FC=337314686 Ta=15.66 Td=8.87881 RH=63.95 P=872 wSpeed=0.913 wDir=54.92 Batt=53.52 some=8.53 Rain=0
WS:   i=4 FC=337314692 Ta=15.66 Td=8.87881 RH=63.95 P=872 wSpeed=0.913 wDir=54.92 Batt=53.52 some=8.53 Rain=0
WS:   i=3 FC=337314697 Ta=15.66 Td=8.87881 RH=63.95 P=872 wSpeed=0.913 wDir=54.92 Batt=53.52 some=8.53 Rain=0
WS:   i=2 FC=337314702 Ta=15.66 Td=8.87881 RH=63.95 P=872 wSpeed=0.913 wDir=54.92 Batt=53.52 some=8.53 Rain=0
WS:   i=1 FC=337314708 Ta=15.66 Td=8.87881 RH=63.95 P=872 wSpeed=0.913 wDir=54.92 Batt=53.52 some=8.53 Rain=0

   % carma/environment/Test/tWS traceLevel=1
{...  A: A R+400462. F+573441. V3 A1 L+400462. E99 99 99 M1280   B+3.2485 C3345  *}
{...  D: D A1 L+400462 C0814  *}
{...  B: B A1 L+400450 C0786  *}
{...  C: C Y05 D0126  T01:13:48 C1281  *}
{...  D: D01+0222.  02+2005.  03+0126.  04+0110.  05+15.75  06+63.58  07+0872.  08+08.53 09+0.888  10+34.47  11+077.0  12+0.000  A1 L+400462 C6292  *}

    !! this is an example of a good (not ``flatlined'') data sample. The WS time is 01:13:48, and
       the last sample time is less then a 5 minutes from this (0110).
       
\end{verbatim}
\normalsize

\newpage
\section{Appendix B: Dew Point Sensor}

% 2005: current model appears to be model 1088

The Dew Point Hygrothermometer is a Model 1063 
(aquired 1991 from Technical Services 
Laboratory\footnote{{\tt http://www.tslinc.com/}}, 
and returns the ambient temperature 
and a direct measurement of the dewpoint temperature 
from a chilled mirror hygrothermometer.  It currently
is configured in a mode where it returns the two
temperatures (in 23 bytes) every 2 seconds accross
a 600 baud serial line. This theoretically
takes 0.345 seconds to transmit.

\subsection{Wiring}

Serial line at the DPS (currently on the roof of the 
OVRO control building) is converted to fiber, braught
into the building and converted back to a serial line.

\subsection{Debugging}


\subsubsection{minicom}

\footnotesize
\begin{verbatim}
   Serial line:    600 8N1: 600 baud, 8 bit, no parity, 1 stop bit

   Note that minicom doesn't support this speed from the menus, but you can
   edit them manually in the file /etc/minirc.dps

# Machine-generated file - use "minicom -s" to change parameters.
pr port             /dev/ttyUSB1
pr bits             8
pr parity           N
pr stopbits         1
pu baudrate         600
pu minit            
pu mreset           


   after which the command will be:


   % minicom dps

   assuming of course it has been stuck in the KEYSPAN #2 port!

   Since the DPS is an autonomous machine, you can't talk to it, it will
   spit out numbers every 2 seconds:

.T+ 58.3/D+ 18.6 P68.
.T+ 58.5/D+ 18.6 P6A.
.T+ 58.6/D+ 18.5 P6A.
.T+ 58.7/D+ 18.6 P6C.
.T+ 58.8/D+ 18.6 P6D.

   The 58.* values are the ambient temperature in Fahrenheit, where 
   the 18.* values are the dew point temperature.  The P6* values are
   a checksum (which we ignore) to validate the two preceding numbers 
   are ok. The T+ and D+ are equally so important, they indicate
   good values....

\end{verbatim}
\footnotesize

\subsubsection{tDPS}

The test program {\tt carma/environment/Test/tDPS} can be used to debug
communications:

\footnotesize
\begin{verbatim}

   % cd $CARMA_BUILD
   % carma/environment/Test/tDPS repeat=5
DPS:  i=5 FC=335733070 dFC=1 Ta=22.6667 Td=-4.44444 RH=16.0163
DPS:  i=4 FC=335733074 dFC=4 Ta=22.6111 Td=-4.44444 RH=16.071
DPS:  i=3 FC=335733078 dFC=4 Ta=22.6111 Td=-4.38889 RH=16.1375
DPS:  i=2 FC=335733082 dFC=4 Ta=22.5556 Td=-4.38889 RH=16.1926
DPS:  i=1 FC=335733086 dFC=4 Ta=22.5    Td=-4.38889 RH=16.248

    where it should be noted that RH has been derived from Ta and Td, not measured.
    Note these are not the raw measured values, the instrument returns
    two temperatures in Fahrenheit, and have been converted to Celcius
    for readability here. Increasing traceLevel will show the raw measured
    values in the mostly readable output from the serial port:

    % carma/environment/Test/tDPS traceLevel=2
{.... buffer[23]="T+ 58.0/D+ 28.5 P65"}
{.... T+ offset by 1 bytes}
DPS:  i=1 FC=337314227 dFC=1 Ta=14.4444 Td=-1.94444 RH=32.3566

\end{verbatim}
\normalsize
% $

\newpage
\section{Appendix C: Linux cautionary notes}

\subsection{Setup}

Given a standard CARMAlized linux workstation, the follow items need to be in place
for the weather station:

\begin{itemize}

\item wider permission on {\tt /dev/ttyUSB*}. We used {\tt chmod 666 /dev/ttyUSB*}
The permissions on a serial device may be reset upon a reboot, which is related
to the new udev filesystem.  The official udev way is to set those
permissions in the file {\tt /etc/udev/permissions.d/50-udev.permissions}.


\item sudo more mortal users to use minicom: 
add the line {\tt Cmnd\_Alias   MINICOM = /usr/bin/minicom}
to the file {\tt /etc/sudoers}. 
This will allow you to make {\tt /etc/minirc.ws} and
{\tt /etc/minirc.dps} if you desire to debug with {\tt minicom}.

\item {\it anything else Paul?}

\end{itemize}

Since we are using the KEYSPAN 4port serial-USB converted a few cautionary
words for systems that don't auto(un)load their modules
so well. Once the KEYSPAN is plugged
into the USB port, a {\bf red light} should show up on the base right under the USB
(open circle) symbol. This means the drivers were loaded. Once you
connect a serial device, the moment the software opens the port
a {\bf green light} will be on
at the port (1,2,3 or 4) that is being used. 
{\bf Never disconnect the USB cable at this stage}, 
though it's safe to disconnect the serial device itself. If you would, the
device would most likely lock up and a reboot is needed to clear the status.

\subsection{Runtime}

The weather station normally runs on {\tt environment.carma.pvt} as the task
{\tt weatherStation}. Since it keeps the serial port to both DPS and WS open,
running test programs like {\tt tWS} and {\tt tDPS} will either crash
{\tt weatherStation} (for {\tt tWS}) or
return nonsense (for {\tt tDPS}), depending how the hardware deals with
multiple attempts to access.

\footnotesize
\begin{verbatim}

WS:  if a 2nd client comes in, 1st one dies and 2nd takes over
DPS: 2nd client just returns no data, need to stop the server

## on the server (normally acc.carma.pvt)


% /opt/rt/bin/imradmin --imr corba --get-server-info WeatherHost
% /opt/rt/bin/dumpMonitor frames=1 component=Weather


% /opt/rt/bin/imradmin --imr corba --stop-server WeatherHost
% /opt/rt/bin/imradmin --imr corba --reset-server WeatherHost

% /opt/rt/bin/imradmin --imr corba --start-server WeatherHost


## on the hostmachine (normally environment.carma.pvt):

% /opt/rt/bin/dumpMonitor frames=1 subsystem=Weather


% /opt/rt/bin/imradmin --imr corba --set-server WeatherHost args "imr=corba.carma.pvt:20000 traceFile=syslog traceLevel=0 useDBMS=true emulate=false sleep=1 dps=false ws=true --"


## tinkering with the args or executable, good for debugging via your own 

% /opt/rt/bin/imradmin --imr corba --get-server-info WeatherHost
% /opt/rt/bin/imradmin --imr corba --set-server WeatherHost args "imr=corba.carma.pvt:20000 traceFile=syslog traceLevel=0 useDBMS=true emulate=false sleep=1 dps=false ws=true --"

% /opt/rt/bin/imradmin --imr corba --set-server WeatherHost exec /home/teuben/carma/build/bin/weatherStation



Server WeatherHost:
    ID:                       15
    Status:                   stopped
    Name:                     WeatherHost
    Host:                     environment.carma.pvt
    Path:                     /opt/rt/bin/weatherStation
    RunDir:                   /home/control
    Arguments:                imr=corba.carma.pvt:20000 traceFile=syslog traceLevel=1 useDBMS=true imrpoa=true emulate=false sleep=1 dps=false ws=true --
    Activation Mode:          shared
    POA Activation:           true
    Update timeout (ms):      500
    Failure timeout (secs):   15
    Maximum spawn count:      2
    Started manually:         no
    Number of times spawned:  0

\end{verbatim}
\footnotesize

\subsection{Testing on a localhost}

Just for completeness, here's an example annoted shell session of 
running a weather station server and monitor on an isolate localhost.

\footnotesize
\begin{verbatim}

       # rc.carma will take a few seconds until the shell prompt returns
1% scripts/rc.carma --nosu --imr localhost --file imr/controlTestLocal.xml --dir /tmp/carma --noapps start

       # this will start carma_tools/bin/{imr,nameserv,notserv}
2% bin/frameScriberPublisher imr=localhost subsystem=weather &  ==  (client)
3% bin/frameCollator imr=localhost &    === acc (server)
4% bin/weatherStation imr=localhost sleep=1 &

       # subsystem= on the client, component= on the server
5% bin/dumpMonitor subsystem=Weather stats=t
6% bin/dumpMonitor subsystem=Weather frames=2

       # if ready to stop
7% scripts/rc.carma --nosu --imr localhost --file imr/controlTestLocal.xml --dir /tmp/carma --noapps stop

\end{verbatim}
\footnotesize

\subsection{Testing on environment}

Here's an example annoted shell session of running the weather station server on {\tt environment},
and being careful about not interfering with the system itself 


\footnotesize
\begin{verbatim}

       # rc.carma will take a few seconds until the shell prompt returns
       # notice the new non-conflicting port number (carma uses 20000) and FQN usage
       # if you do want to post monitor points to the system just use 20000 instead
1% scripts/rc.carma --nosu --imr environment.carma.pvt:30000 --file imr/controlTestLocal.xml --dir /tmp/carma --noapps start
1% scripts/rc.carma --nosu --imr corba.carma.pvt:20000       --file imr/wx.xml --dir /tmp/carma --noapps start
       # this will have started carma_tools/bin/{imr,nameserv,notserv}

2% bin/frameScriberPublisher imr=environment.carma.pvt:30000 subsystem=weather &
3% bin/frameCollator imr=environment.carma.pvt:30000 &
4% bin/weatherStation imr=environment.carma.pvt:30000 sleep=1 emulate=false ws=true dps=false &
       # this will oddly (?) also publish them as monitor points, so you can monitor them with e.g. rtd

       # subsystem= on the client, component= on the server
5% bin/dumpMonitor subsystem=Weather stats=t
6% bin/dumpMonitor subsystem=Weather frames=2

       # if ready to stop
7% scripts/rc.carma --nosu --imr environment.carma.pvt:30000 --file imr/controlTestLocal.xml --dir /tmp/carma --noapps stop

\end{verbatim}
\footnotesize

\section{Appendix D: Monitor Points}

There is no good recipe for adding a new monitor point, but here are some points that certainly would
have to be taken care of. Examples are taken from the weather station(s) code, a relatively simple
CARMA component in itself:

\begin{enumerate}

\item
Add the monitor point in the appropriate spot into {\tt carma/monitor/WeatherSubsystem.mpml}

\item
Add the appropriate item in e.g. {\tt carma/environment/WS.h}, and grab it from the instrument
in {\tt carma/environment/WS.cc}

\item
Add the item to {\tt carma/environment/Weather.h} and grab it from WS in
{\tt carma/environment/Weather.cc}

\item
grab and publish the value in the DO {\tt  carma/environment/weatherStation.cc} in
something like
{\tt  wss.weatherStation().lastTime().setValue(w.getWSLastTime());}

\item
If needed, modify {\tt carma/ui/rtd/windows/rtdweather.cc}

\end{enumerate}

\end{document}

station 222 and 333 are different styles of output, e.g.

D01+0222.  02+2005.  03+0125.  04+2355.  05+18.73  06+45.89  07+0872.  08+07.36 09+1.546  10+204.4  11+29.15  12
D01+0333.  02+2005.  03+0125.  04+2400.  05+09.54  06+3328.  07+17.52  08+20.01 09+49.97  10+07.45  11+24.51  12+13.64  13+13.52  14+245.9  15+0.000  16+227.1 17+076.5  18+16.58  19+0876.  20-69
D01+0222.  02+2005.  03+0126.  04+0015.  05+17.39  06+54.78  07+0872.  08+08.11 09+0.408  10+139.9  11+15.97  12
D01+0222.  02+2005.  03+0126.  04+0030.  05+17.58  06+44.62  07+599.6  08+6.678 09+0.746  10+182.3  11+24.05  12


control  22010 95.0  2.3 57372 24376 ?       Rsl  09:25   0:00 /opt/rt/bin/weatherStation imr=corba:20000 traceFile=syslog traceLevel=0 useDBMS=true imrpoa=true emulate=false dps=false ws=true -- -ORBServerId WeatherHost -ORBserver_instance 1128011145

/opt/rt/bin/frameScriberPublisher 
   is now running twice....


Sep 29 09:25:45 environment.carma.pvt weatherStation:  {2005} {NOTICE} {carma.environment.weatherStation} {} {Program starting, pid=22010}  
Sep 29 09:25:45 environment.carma.pvt /opt/rt/bin/weatherStation:  {2005} {WARN} {carma.environment.weatherStation} {} {Database cannot be contacted, so using the canonical name to tagId map in conf file /home/control/rt/UPDATE-2005-09-28-1706-2-6-9-11-ELsmp/conf/monitor/canonicalNamesToTagIDs.conf. It had better match that in the database or db integrity will be compromised}  
Sep 29 09:25:46 environment.carma.pvt /opt/rt/bin/weatherStation:  {2005} {INFO} {carma.environment.weatherStation} {[carma::util::CarmaThreadGroup::getDefaultCarmaThreadGroup]} {carma::util::CarmaThread::start: starting thread CarmaThread.defaultSignalHandler with id 3061857200 }  
Sep 29 09:25:46 environment.carma.pvt /opt/rt/bin/weatherStation:  {2005} {INFO} {carma.environment.weatherStation} {} {WS:open /dev/ttyUSB0 -> 12}  
Sep 29 09:25:47 environment.carma.pvt /opt/rt/bin/weatherStation:  {2005} {INFO} {carma.environment.weatherStation} {} {Ta/Td/RH/Ta/Td/RH/P/wS,wD,h2o= 0 0 100 13.99 -3.94567 28.51 1.705 21.65  7.23962 210 0}  
Sep 29 09:25:48 environment.carma.pvt /opt/rt/bin/weatherStation:  {2005} {ERROR} {carma.environment.weatherStation} {Exception Handlers Watchdog} { Program received signal - SIGHUP - hangup signal with number 1 and is being terminated  }  
Sep 29 09:25:48 environment.carma.pvt /opt/rt/bin/weatherStation:  {2005} {NOTICE} {carma.environment.weatherStation} {Exception Handlers Watchdog} { Program weatherStation(22010) terminating...  }  



---


**A
 R+205061. F+205060. V3 A1 L+205061. E99 99 99 M1280   B+3.2713 C3324  
**B
 A1 L+205049 C0793  
****C
 Y05 D0322  T01:43:27 C1279  
**C
 Y05 D0322  T01:43:28 C1280  
**D
01+0222.  02+2005.  03+0321.  04+1040.  05-1.042  06+39.42  07+0788.  08+1.749 
09+4.755  10+327.7  11+5.423  12+0.000 
                                                                                                                                                                                                                                  
 A1 L+205061 C6286                                                            

---

*A                                                                                                                                                                                                                                
 R+205061. F+205060. V3 A1 L+205061. E99 99 99 M1280   B+3.2713 C3324                                                                                                                                                             
*B                                                                                                                                                                                                                                
 A1 L+205049 C0793                                                                                                                                                                                                                
*C                                                                                                                                                                                                                                
 Y05 D0322  T01:45:28 C1282                                                                                                                                                                                                       
*D
01+0222.  02+2005.  03+0321.  04+1040.  05-1.042  06+39.42  07+0788.  08+1.749
09+4.755  10+327.7  11+5.423  12+0.000
                                                                                                                                                                                                                                  
 A1 L+205061 C6286                                     



-----

Eric algorithm:

send A:
    R+205061. F+205060. V3 A1 L+205061. E99 99 99 M1280   B+3.2713 C3324

find 'C' getLastRecordNo ->  between L+ and .  -> 205061

send '205061G':
    A1 L+205061 C1094

--- recover from long flatline
*1D                                                                                                                           
01+0222.  02+2005.  03+0322.  04+1440.  05-5.459  06+42.53  07+0.000  08+1.321                                                
09+6.211  10+333.8  11+0.000  12+0.000                                                                                        
                                                                                                                              
 A1 L+205073 C6291                                                                                                            
*B                                                                                                                            
 A1 L+205061 C0787                                                                                                            
*B                                                                                                                            
 A1 L+205049 C0793                                                                                                            
*1D                                                                                                                           
01+0222.  02+2005.  03+0321.  04+1040.  05-1.042  06+39.42  07+0788.  08+1.749                                                
09+4.755  10+327.7  11+5.423  12+0.000                                                                                        
                                                                                                                              
 A1 L+205061 C6367                                   


i see an error code?

E10 04
intermediate storage full?



 04+1455.  A1 L+205109 C6305


E10 96  
E10 04                                                                                                                        
*                                                                                                                             
*                                                                                                                             
E10 96 A                                                                                                                      
 R+205289. F+205288. V3 A1 L+205289. E99 99 99 M1280   B+16.748 C3734                                                         
*                                                                                                                             
E10 04 B                                                                                                                      
 A1 L+205277 C1149                                                                                                            
*C                                                                                                                            
 Y05 D0322  T16:14:01 C1275                                                                                                   
*D                                                                                                                            
E10 04                                                                                                                        
01+0222.  02+2005.  03+0322.  04+1610.  05+09.36  06+15.08  07-6999.  08+1.360                                                
09+1.827  10+123.7  11+0.000  12+0.000                                                                                        
                                                                                                                              
 A1 L+205289 C6638                                                                                                            
*                                                                                                                             
E10 04                                                              

E10 04                                                                                                                        
E10 10                                                                                                                        
E10 04                                                                                                                        
E10 04 

e10 11

*A
 R+2917387. F+2917386. V3 A1 L+2090621. E99 00 31 M0000   B+3.2352 C3476  
*B
 A1 L+2090591 C0847  
*C
 Y05 D0322  T16:53:35 C1285  
*D
01+0172.  02+54.62  03+0344.  04-453.9  05+205.9  06-3904.  07+3934.  08-30.61 
09-573.0  10+6.742  11+169.8  12+2.936  13+69.58  14-604.5  15-4.738  16-6.001 
17-64.36  18+035.8  19-3.393  20+3.871  21-345.0  22-044.5  23+62723  24+3357. 
25+19.63  26+.H5126 

 A1 L+2090621 C4773  

====


on non-corba mahines imr's are also running in OAD mode (the -s flag)

/opt/carmaTools/bin/imr -a -s -ORBconfig /var/carma/imr/bima1.carma.pvt_20000/oad.conf


on corba:

/opt/carmaTools/bin/imr -a -ORBconfig /var/carma/imr/corba.carma.pvt_20000/imr.conf
/opt/carmaTools/bin/nameserv -ORBDefaultInitRef .....
/opt/carmaTools/bin/notserv -ORBDefaultInitRef ....


building:


make lib/libcarmaenvironment.so
make AR:= RANLIB:= install 
