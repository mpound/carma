%$Id: auxservices.tex,v 1.26 2013/11/19 22:24:39 teuben Exp $
\documentclass[preprint]{aastex}
\usepackage{mathrsfs}
\usepackage{epsf}
%\usepackage{makeidx}
%\usepackage{fancyhdr}
%\usepackage{graphicx}
%\usepackage{float}
%\usepackage{alltt}
%\usepackage{doxygen}
% --------------------
% ----------------------DEFINITIONS USED IN THIS PAPER --------------------
\newcommand{\xlam}{
   \ifmmode{X_\lambda}
   \else{$X_\lambda$}
   \fi}
\newcommand{\ylam}{
   \ifmmode{Y_\lambda}
   \else{$Y_\lambda$}
   \fi}
\newcommand{\zlam}{
   \ifmmode{Z_\lambda}
   \else{$Z_\lambda$}
   \fi}
\newcommand{\rearth}{  % Earth radius
   \ifmmode{r_\oplus}
   \else{$r_\oplus$}
   \fi}
\newcommand{\scrL}{ % script L for path length
   \ifmmode{{\mathscr L}}
   \else{${\mathscr L}$\/}
   \fi}
\def\degree{\ifmmode{^\circ} \else{$^\circ$}\fi}
\def\nexpo#1#2{\ifmmode{#1 \times 10^{#2}}\else{$#1 \times 10^{#2}$}\fi}
\def\expo#1{\ifmmode{10^{#1}}\else{$10^{#1}$}\fi}
\def\msun{\ifmmode{\rm M_\odot}\else{M$_\odot$}\fi}
\def\esat{\ifmmode{\rm e_{sat}}\else{e$_{sat}$}\fi}
\def\etal           {{\rm et~al}.\ }
\def\eg             {{\rm e.g.}\/}
\def\ie             {{\rm i.e.}\/}
\def\uv{{\it uv}}
\def\cmm {\ifmmode{{\rm cm^{-3}}}\else{{${\rm cm^{-3}}$}}\fi}
\def\cmtwo {\ifmmode{{\rm cm^{-2}}}\else{{${\rm cm^{-2}}$}}\fi}
\def\invpc{\ifmmode{\,{\rm pc^{-1}}}\else{\thinspace {\rm pc$^{-1}$}}\fi}
\def\invsec{\ifmmode{\,{\rm s^{-1}}}\else{\thinspace {\rm s$^{-1}$}}\fi}
\def\invyr{\ifmmode{\,{\rm yr^{-1}}}\else{\thinspace {\rm yr$^{-1}$}}\fi}
\def\psec {\ifmmode{\,{\rm .\negthinspace^{s}}}\else{{$.\negthinspace^{s}$}}\fi}
\def\pmin           {$.\negthinspace^{m}$}
\def\pasec           {$.\negthinspace^{\prime\prime}$}
\def\pdeg           {$.\kern-.25em ^{^\circ}$}
\def\kms{\ifmmode{\,km\,s^{-1}}\else{\thinspace km\thinspace s$^{-1}$}\fi}
%
% Greater than or approximately equal to and less than or
% approximately equal to signs, using macros defined in PLAIN.TEX
%
\catcode`\@=11 \def\gapprox{\mathrel{\mathpalette\@versim>}}
\def\lapprox{\mathrel{\mathpalette\@versim<}}
\def\@versim#1#2{\lower2.9truept\vbox{\baselineskip0pt\lineskip0.5truept
    \ialign{$\m@th#1\hfil##\hfil$\crcr#2\crcr\sim\crcr}}}
\catcode`\@=12
% --------------------------------------------------------------------------
\begin{document}
\title{CARMA Auxiliary Services (WP\#163)}
\author{Peter Teuben, Chul Gwon, Marc Pound}
\slugcomment{
Version 3.0\ \ \today
}

\section{FUNCTIONALITY}

Auxiliary Services is intended to provide routines commonly needed
for array control and observational planning.  These routines are
mostly, but not exclusively, astronomical in nature.  Auxiliary 
Services will contain a mixture of new code, reused code, and
wrappers for external libraries. It will involve both library
routines and standalone programs (depending on the nature of the
service).  We describe the services below.
Observational planning tools that use auxiliary services must
work off-site (e.g., no RDBMS) and  ideally without a network
connection. 

\subsection{Tables and Catalogs}

Table representation and I/O is needed by control procedures
and for observational planning.  Examples are source catalogs
(a catalog is just a kind of table), spectral line catalogs,
phase switch tables, and flux calibrator history.  This package
will provide support for reading and writing of tables, including
both public and private catalogs.

Catalog support is needed both on-site and off-site.  On-site,
source catalogs provide the connection between a source name and its
coordinates (including, e.g. VLSR velocity, but also proper motion)
which ultimately get passed to the observing system. Similarly,
line catalogs provide the connection between a spectral line name
and its rest frequency.  Off-site, these catalogs are needed  by
the user planning observations perhaps disconnected from internet
(laptop on the airplane paradigm).  Flux calibrator history is
needed for (obviously) flux calibration, planning observations,
and is also useful for system checks.

Source catalogs should contain catalog name and source specifications,
where the specification would consist of source name(s), coordinates 
and equinox, velocity, velocity reference frame and definition, proper motion, 
source size and flux estimates as a function of frequency
(Requirement 2.6.14 R1-3).  The requirement that source size and flux
be part of the source catalog (2.6.14-R2) will make a non-RDBMS 
implementation more difficult, as it makes the size of a single catalog
entry variable. It may also put an unnecessary burden on observers when
creating their observing program. Its intention is essentially
covered by the calibrator flux catalog; we suggest it be dropped.

Line Catalogs in general are obtained externally, e.g., the Lovas and
Poynter-Pickett catalogs. However, if we want to create a separate
site catalog (say, frequently observed lines), then the minimum columns
needed are line identification and rest frequency.

It is a requirement  (2.7.5-R1) that the ``system automatically
harvest calibrator visibilities and make them available as a function
of frequency and time.''\footnote{We believe this requirement is
poorly specified.  We suggest it be re-written as ``make their fluxes
available as a function of frequency and time.''} Most user access
to the flux measurements will be the measurements themselves
rather than the visibilities, so a separate flux calibrator catalog
seems appropriate where some quality control is possible during
the transfer.  Flux calibrator history should be available to data
calibration software.  For MIRIAD, this is accomplished by the
ASCII file {\tt \$MIRCAT/cals.fluxes}, see Section~\ref{fluxtable}
for an example.

Previous standard source and flux catalogs from BIMA and OVRO should
be transferred for testing and first light.

\subsection{Ephemerides and Planetary Brightness}

Ephemerides are needed by the on-site observing system to get RA,
DEC for moving objects like planets and comets. Off-site, again
they are needed for observational planning.  This package will
provide object interfaces to read astronomical ephemerides for
control and observational planning tools.  It will also provide a
program to periodically download updated ephemerides from JPL.
In addition to apparent size, which is obtained via the ephemeris routines,
we need planet brightness calculations for flux calibration.

\subsection{Astronomical Timekeeping}

There are two widely used time standards. One is the rotation of
the earth, and the other is the frequency of atomic oscillations.
The earth's rotation is not uniform; its rate exhibits both
periodic changes and long term drifts on the order of a second per
year. Atomic standards are the closest approximations currently
available to a uniform time with accuracies on the order of
microseconds per year.

For internal time representation, CARMA uses Modified Julian Day
(Scott \& Woody 2002; Requirement 2.6.5-R5). However, for
source tracking and observational planning, universal and/or
sidereal time are needed.  Auxiliary services will provide
appropriate conversion between the following timescales as
needed\footnote{A good review of astronomical timekeeping,
from which much of this discussion is copped, can be found at
{\tt http://www.cv.nrao.edu/\~ rfisher/Ephemerides/times.html}}

\subsubsection{Atomic Times}

\subsubsubsection{International Atomic Time: TAI}

The SI second is defined as ``the duration of 9,192,631,770 cycles of
radiation corresponding to the transition between two hyperfine
levels of the ground state of Cesium 133.'' The International
Atomic Time scale (Temps Atomique Internationale, TAI) is the 
primary world time standard. It is a statistical
timescale based on a large number of atomic clocks.  

\subsubsubsection{Coordinated Universal Time: UTC}

Coordinated Universal Time (UTC) is the basis of civil timekeeping.
By definition, UTC and TAI have the same rate, but UTC stays close
to Mean Solar Time (and thus slows down) 
by adding integer numbers of seconds, called
leap seconds, from time to time.

\begin{equation}
UTC = TAI - N_{leap}
\end{equation}

\noindent where $N_{leap}$ is the number of leap seconds for the
date in question.  The leap second table is available from 
the International Earth Rotation Service (IERS).
The CARMA master clock is synched to UTC (and thus TAI) via GPS.  
The number of leap seconds (now at 32) has not changed since January 1, 1999,
and there is discussion of abandoning it.

\subsubsubsection{Julian Day}
The Julian Day (JD) of a particular instant of time is the number
of days and fractions of a day since 12 hours UTC on
January 1 of the year -4712, where the year is given in the Julian
proleptic calendar.  The Modified Julian Day (MJD) was introduced by
space scientists in the late 1950's. It is defined as
$MJD = JD - 2400000.5.$
The half day is subtracted so that the day starts at midnight in
conformance with civil time reckoning in Greenwich.
As mentioned above, CARMA expresses UTC as MJD (see carma::util::Time).

\subsubsubsection{Dynamical Time: TT, TDB}

Dynamical time is the independent variable in the theories which
describe the motions of bodies in the solar system. When looking up
the position of a solar system body in a precomputed ephemeris,
the date and time used must be in terms of one of the dynamical
timescales.

Terrestrial Time (TT)\footnote{You might also find references to this
as ET}, with unit of duration 86400 SI seconds on the geoid, 
is the independent argument of apparent geocentric ephemerides.  It is
linked to TAI through the exact formula 
\begin{eqnarray}
TT &=& TAI + 32.184~s \\
   &=& UTC + N_{leap} + 32.184~s
\end{eqnarray}

Barycentric Dynamical Time (TDB) is the independent argument of 
ephemerides and dynamical theories that are referred to the 
solar system barycenter.  TDB differs from TT by an amount
which cycles back and forth by between 1 and 2 milliseconds due to
relativistic effects.  Planetary motions are computed using TDB.


\subsubsection{Earth Rotation Times}

\subsubsubsection{Universal Time: UT1}

Universal Time (UT1 = UT) measures the actual rotation of the earth,
essentially mean solar time.  It is the observed rotation of the
earth with respect to the mean sun corrected for the observer's
longitude with respect to the Greenwich Meridian and for the
observer's small shift in longitude due to polar motion.  Since the
earth's rotation is not uniform, the rate of UT1 is not constant,
and its offset from atomic time is continually changing in a not
completely predictable way.

UT1 is needed for computing the sidereal time, an essential part of
pointing a telescope at a celestial source. To obtain UT1, one must
look up the value of UT1-UTC (aka DUT1) for the date concerned in tables
published weekly by the IERS.

\begin{equation}
UT1 = UTC + DUT1
\end{equation}

The quantity UT1-UTC typically changes by 1 or 2 ms per day, and
can only be obtained by observation, though seasonal trends are
known and the IERS listings are able to predict some way into the
future with adequate accuracy for pointing telescopes. The value
as of this writing is about -600 ms.

\subsubsubsection{Sidereal Time: GMST, GAST, LMST, LAST}

Greenwich Mean Sidereal Time (GMST) is linked to UT1 by a
numerical formula described in 
{\it The Astronomical Almanac}.  There are
no leap seconds in GMST, but the second changes in length along with
the UT1 second, and also varies over long periods of time because
of slow changes in the Earth's orbit. This makes the timescale
unsuitable for everything except predicting the apparent directions
of celestial sources.

Greenwich Apparent Sidereal Time is Greenwich Mean Sidereal Time
(GMST) corrected for the shift in the position of the vernal equinox
due to the ellipticity of the earth's orbit and nutation. The right 
ascension component is called the "equation of the equinoxes":

\begin{equation}
GAST = GMST + EE.
\end{equation}

The Local Mean Sidereal Time (LMST) is just GMST plus the observer's
east longitude, where longitude is measured positive to the east of
Greenwich (so LMST is GMST {\it minus} what we traditionally think of as our
longitude).

Local Apparent Sidereal Time (LAST or simply LST) is the apparent right
ascension of the local meridian, from which the hour angle of any
celestial source can be determined knowing its RA. It is obtained from 
GMST by adding the east longitude (corrected for polar motion, $\Delta\lambda$)
and the equation of the equinoxes. 
\begin{eqnarray}
LAST &=& GMST + \scrL + \Delta\lambda + EE \\
     &=& GAST + \scrL + \Delta\lambda 
\end{eqnarray}


\subsection{Source Coordinates and Velocities}

Sky coordinates are used for source tracking and observational
planning.  CARMA will support Equatorial J2000, Galactic, and AzEl
sky coordinate systems (Requirement 2.6.4-R2).  Full conversion from
equinox coordinates to current apparent coordinates is, of course,
a requirement for tracking a source. Source apparent coordinates must
include a correction for diurnal aberration due to the displacement
of the observer from the center of the Earth, as well as annual
aberration.  Table 1 contains the velocity reference frames that
will be supported by CARMA.

\begin{table}[!h]
\label{t-velframe}
\caption{CARMA Supported Velocity Reference Frames}
\begin{tabular}{lll}
\hline
Frame & Description &  Definition \\
\hline
Topocentric & The LO is fixed, no tracking & Antenna station coordinates\\
Barycentric &  Solar System Barycenter &  JPL Ephemeris DE405 \\
Heliocentric &  Center of Sun & JPL Ephemeris DE405 \\
Kinematic LSR &  Conventional Local Standard of Rest &  Solar motion towards (18h +30�) at epoch 1900.0 \\
\hline
\end{tabular}
\end{table}

A full source specification must include a radial velocity to allow LO
tracking.  As with sky coordinates, there are a number of ways of
specifying velocity. One must specify a {\it velocity reference
frame} and a {\it velocity definition}.\footnote{Requirements
2.6.2-R3 states that CARMA will support the "LSR, Heliocentric,
Planetary, Topocentric and CZ velocity reference frames", confusing
velocity frame and velocity definition. This needs to be corrected.}
Table 2 contains common radial velocity definitions.
The radio and optical (equivalent to redshift) definitions are 
both approximations to the relativistic formula valid for $V<<c$.

\begin{table}[!h]
\label{t-veldef}
\caption{Radial Velocity Definitions} 
\begin{tabular}{lll}
\hline
Definition  & Formula &  Conversion to Frequency \\
\hline
Radio & $V = c (\nu_0 - \nu)/\nu_0$ & $\nu(V) = \nu_0 (1-V/c)$\\
Optical & $V = c (\nu_0 - \nu)/\nu$ & $\nu(V) = \nu_0 (1+V/c)^{-1}$\\
Redshift & $z = (\nu_0 - \nu)/\nu$ & $\nu(V) = \nu_0 (1+z)^{-1}$\\
Relativistic & $V = c (\nu_{0}^2 -\nu^2)/(\nu_{0}^2 +\nu^2)$ & 
$\nu(V) = \nu_0 \sqrt{1 - (V/c)^2}~(1 + V/c)^{-1}$\\
\hline
\end{tabular}
\end{table}

To calculate Doppler corrections, we need to place the source at rest with
respect to the observer at the time of interest. 
We can write the Doppler velocity as
\begin{equation}
V_{Doppler} = \vec{V}_{\earth}\cdot\vec{r_s}
            + V_{rad,frame} - \vec{V}_{frame}\cdot\vec{r_s},
\end{equation}
\noindent where the first term is the observer's motion toward
the source, the second term is the radial velocity in the relevant
frame, and the last term is the motion of the frame toward the
source.
For Doppler tracking of the LO, the Doppler velocity must be converted to
observed frequency $\nu$, given the desired rest frequency $\nu_0$.

\subsection{Unit Conversion}
The CARMA Standard Units have been defined for all public APIs\footnote
{{\tt http://www.mmarray.org/workinggroups/computing/CarmaStdUnits.html}}.
Internal representation of quantities is often in units convenient
for the programmer or the associated hardware.  Auxiliary Services
provides a comprehensive unit conversion class.

{\it insert results of discussion on the use of units during R3 integration week here}

\subsection{Antenna Location and Pointing Solutions}

There are a number of ways to represent the antenna locations.  It has
been decided that CARMA will present antenna coordinates to the observer
in the convenient (East, North, Up) system, which is also used in the array
configuration memo. However, other processes, such as the Delay Engine
and baseline determination require a different antenna coordinate system
(e.g. equatorial XYZ, which can be either geocentric or topocentric).
Therefore conversion utilities are needed and will provided by this package.

After array reconfiguration, antenna positions must be determined
to $\sim\lambda/10$.  A baseline error $\Delta b = (\Delta X, \Delta
Y, \Delta Z)$  will translate into a phase error $\Delta\phi  =
2\pi/\lambda\Delta \vec{b}\cdot\vec{s}$, where $s$ is the unit vector in
the source direction.  Observations of quasars with well-known
positions are used to determine $\Delta b$. (Wright 1990;
Thompson, Moran \& Swenson 1986 [TSW]).
Routines will be provided to determine antenna position (baseline)
from such observations.  (The Program to acquire such observations
is part of Atomic Commands).

The pointing model is used to remove effects such as encoder
zero offsets, axis misalignments, tilts,  etc.   The azimuth and
elevation correction terms, $\Delta AZ$  and  $\Delta EL$, are
described by equations with 7 to 9 terms (Wright 1990; Mangum 2001). The
specification of the pointing model is covered in the Antenna API.
This package will provide the program(s) to reduce pointing data
and derive the pointing model coefficients.  It also will provide
the atmospheric refraction correction term needed for source tracking.

\subsection{Observing Modes}

Observing Modes are described in HLCR\&S Section 2.4 and in the Singledish
work package.  A recap: Interferometric modes are Single pointing, Multiple
pointing, and Mosaic.  Singledish modes are Spectral line Position-switch,
Spectral line On-The-Fly, and Continuum beam-switch.  In discussions 
about the Singledish package, Pound and Welch have ruled out spectral
line frequency-switching and continuum OTF.

These modes are specified by the observer, so must be passed to the
Control system by the Observing Blocks, for instance in a get/setObsMode() in
SubarrayControl.  We can create a class in Auxiliary Services that has
enums/string representations of the available modes.

\section{IMPLEMENTATION}
The solutions used by OVRO and BIMA to the above aspects of
observing and control are described in the Work Package analysis 
\footnote{{\tt http://www.mmarray.org/project/WP/AuxiliaryServices/auxservices2.pdf}}
and will not be repeated here, except where this package 
is reusing previous code or algorithms.

\subsection{Tables and Catalogs}
{\em Intelligent ASCII Tables} will be used for a number of
tables/catalogs described here.  ASCII makes the most sense as data
format for the end-user.  It is portable and easy to edit without
special tools.  (It also opens up the possibility of using CVS to
keep on-site and off-site catalogs in sync).  By intelligent we mean
it has a header that would appear to most table readers and plotters
as comment lines, but define the names of the columns, their data types, their units etc. 
An example of this could look as follows\footnote{units and types are generally
not used or enforced in our table I/O}:

\footnotesize
\begin{verbatim}
   #|name|ra      | dec   | flux | error | date|
   #| s  | r      |  r    |  r   |  r    |  r  |
   #|    | deg    |deg    | Jy   | Jy    | MJD |
   #
   3c555 6.234366 -12.2244   3.2 0.1  342.45623349    # 2004-02-03T12:33:22.2
   ...
\end{verbatim}
\normalsize

Although in some sense these tables perform a function that could
also be take up by our RDBMS (e.g. mySQL), it is not meant to
approach that level, and the simple API we are proposing will
reflect that. It is meant to make reading and writing such (ASCII)
tables quick and easy.   For very large tables, there may be performance
hits for searching.  Lovas 3 mm and 1 mm catalogs together are 3000
lines; BIMA calibrator fluxes has 4800 entries; Poynter-Pickett is
huge but unlikely to be used much.

A well-defined programmatic interface for this to read and write 
such tables  is provided by 
carma::services::Table API\footnote{see also {\tt carma::dbms::Table} for a
simular interface to our RDBMS}
A Table contains an ASCII ``spreadsheet''  
row-based representation of the file, and member functions exist to
extract usefully typed (double, HMS, DMS, ...) columns or rows
from this table.


\footnotesize
\begin{verbatim}
  Table t("flux.tab");
  vector<double> flux, mjd;

  mjd  = t.getDoubleColumn(5);       // 6th column (columns are 0-based)
  flux = t.getDoubleColumn("flux");  // column named ``flux''

  // getHmsColumn is overloaded to allow sexagesimal or
  // space-separated H,M,S.  Ditto for DMS.
  ra   = t.getHmsColumn(2);           // H:M:S.S in column 2
  ra   = t.getHmsColumn(2,3,4);       // H M S.S in columns 2,3,4 resp.

\end{verbatim}\normalsize

A pure virtual class Catalog is implemented, from which various
specific catalogs can be derived.

\footnotesize
\begin{verbatim}
class Catalog
        // All Catalogs have a method to lookup an entry (row)
	virtual CatalogEntry& lookup(string entryName);

class CatalogEntry
        virtual string getName();

class Source : public CatalogEntry
class SpectralLine : public CatalogEntry
class Flux: public CatalogEntry

class SourceCatalog : public Catalog
class SpectralLineCatalog : public Catalog
class FluxCatalog : public Catalog

\end{verbatim}
\normalsize

\begin{itemize}
\item {\em Source Catalogs} 

Initially, ASCII source catalogs created by observers will be used directly 
by Control via the SourceCatalog API.  Later, uploading or downloading of
ASCII tables to/from the system RDBMS could be implemented.  
An example catalog use would be as follows:

\footnotesize
\begin{verbatim}
  // open the  CARMA source catalog on disk
  SourceCatalog cat(Program::getConfFile("catalogs/SystemSource.cat");
  // find a source, case insensitive
  Source source = cat.lookup("3c273");
  // Point the antenna at it.
  drive.setRaDec(mjd, source.getXCoordinate(), source.getYCoordinate(), true);
\end{verbatim}
\normalsize

with the following an example of a source entry:

\footnotesize
\begin{verbatim}
#| Source|   RA       |   DEC      | Parallax| Velocity | VelFrame | VelDef |  PMRA | PMDEC | ID |Comments|
#|  s    |   hms      |   dms      |    r    |    r     |    r     |   r    |   r   |   r   |  i |   s    |
#|       |            |            |         |          |          |        |       |       |    |        |
0005+544  00:05:04.363 54:28:24.938    0.0       0.0         LSR      RADIO    0.000  0.000   1     -rla

\end{verbatim}
\normalsize


\item {\em Line Catalogs} 

These are provided in ASCII by catalog
authors. It is rare for a user to add a new line, but some mechanism
could be put in place for user line catalogs (low priority).
Automatic line frequency lookup as at OVRO is a nice feature and
should be kept.  This can be provided with aliases in
the system spectral line catalog for the most commonly observed
lines or with static methods of SpectralLineCatalog.

%% TODO
%% PJT: does CO1-0 really work, i.e. does it klnow how to
%% concat the  Line  and Transition   column
%% e.g. tSpectralLineCatalog line=CO trans=2-1  fails!!

\footnotesize
\begin{verbatim}
  // open and read the Lovas line catalog from disk
  SpectralLineCatalog cat(Program::getConfFile("catalogs/SpectralLine.cat");

  // find the CO J=1-0 line
  SpectralLine line = cat.lookup("CO1-0");

  // get its rest frequency in GHz
  Units u;
  double frequency = line.getFrequency().gigahertz();

  // possible shortcut (currently unimplemented)
  double frequency = cat.getFrequency("CO1-0").gigahertz();

  // get all CO lines in a given frequency range
  vector<Spectraline> lines = cat.lookup("CO",
                                          Frequency(100,"GHz"),
                                          Frequency(231,"GHz")
                                         );
\end{verbatim}
\normalsize

\item {\em Flux Calibrator History}  

A Flux contains the source name, date of observation, flux value in Janskys,
error on the flux, frequency of the measurement, and a descriptive
string (which may be empty). An example of how it might work:

\footnotesize
\begin{verbatim}

class Flux (name,date,value,error,frequency,description);
// read in the flux catalog from disk
FluxCatalag fc("carma.fluxes");

// Get a vector containing all measurements of 3c84
vector<Flux> fluxlist = fc.lookup("3c84");

// Get the measurement closest to a specific data
Flux flux = fc.lookup("3c84",mjd);

// Get only the most recent measurement,
// alias for fc.lookup("3c84",Time::MJD());
Flux flux = fc.lookupLatest("3c84"); 

\end{verbatim}
\normalsize


\end{itemize}

\subsection{Ephemerides and Planetary Brightness}

The JPL Solar System Ephemeris specifies the past and future
positions of the Sun, Moon, and nine planets in three-dimensional
space. Many versions of this ephemeris have been produced to include
improved measurements of the positions of the Moon and planets and
to conform to new and improved coordinate system definitions. The
current state of the art is JPL DE405 (Standish 1998, 
see also \\ {\tt http://ssd.jpl.nasa.gov/iau-comm4}). 
Typically
a binary file (about 4.5MB) containing the tables from which the 3D 
positions of solar system objects are calculated is spans
50 years. Querying a solar system object can be done many
times per ms CPU. The majority of CPU is related to opening
the binary file, and/or post-processing the 3D positions
via the NOVAS routines.

Ephemerides could also be queried/downloaded at the
JPL Horizons website; a simple scripting interface 
to this (e.g. via
email) should be sufficient.  NOVAS requires JPL fortran code
though recently several C versions have also become available. 
We are using the {\tt ephcom} code from ephemeris.com to
access the JPL binary tables.
BIMA has lots of
code for planet brightness calculations (Fortran+C), including
the most recent  Mars model by Mark Gurwell.  These will be
wrapped/translated.  

\footnotesize
\begin{verbatim}
  Ephemeris e;
  double mjd = Time::MJD();

  // set time
  e.setMJD(mjd);
  // set a souorce
  e.setSource("mars");

  // obtain current sky positions
  double ra  = e.getRA();
  double dec = e.getDEC();

  double az = e.getAz();
  double el = e.getEl);

\end{verbatim}
\normalsize

\subsection{Astronomical Timekeeping}
We will use the Naval Observatory Vector Astrometry Subroutines
(NOVAS; Kaplan et al. 1989) library where appropriate.  NOVAS
supplies routines to compute sidereal times and the equation of the
equinoxes given a Julian Day and to convert TDB to TT.  For conversion to 
other timescales, we will write our own routines (probably cut and 
paste from SLALIB).
A crontab script will fetch tables of UT1-UTC from IERS at regular
intervals (once a week should suffice); an API to read these will
be provided by using Table. A class IERSTable is available for this:

\footnotesize
\begin{verbatim}
  IERSTable iers(Program::getConfFile("catalogs/IERS.tab"));
  
  double dut1 =  iers.dut1(mjd);
  double xpol =  iers.xpolar(mjd);
  double ypol =  iers.ypolar(mjd);

\end{verbatim}
\normalsize

\subsection{Source Coordinates and Velocities}
Conversion of J2000 sky coordinates
to current epoch coordinates is required including all corrections
(diurnal aberration, polar motion, etc.). NOVAS provides this functionality 
({\it topo\_star, topo\_planet}).
Full conversion of (HA, DEC) to (AZ,EL) is provided by 
NOVAS ({\it equ2hor}), but not the reverse.
The Delay Engine has a method to 
do the reverse, which will be migrated to this package.

We will allow users to specify any of the four radial velocity
definitions, Redshift is essentially the same as Optical and
would be stored internally as Optical.  We will provide a method to
convert to frequency for a fully-specified velocity.

We need the components of the observer's motion toward the source
in the relevant velocity frame.  NOVAS has routines to compute
observer velocities (e.g., {\it get\_earth} returns the earth's
motion vectors in heliocentric and barycentric coordinates).
We will need to supply the velocity of the standard frames in
an earth- or sun-centered coordinate system.
For instance, the components of the LSR solar motion in J2000.0 in km/s are 
(--.29000685, 17.31726682, --10.00140835).

%% clarification of the system and assumptions made here:
%% this is in APSTAR.F, these are the ra,dec,radial velocities, not galactic UVW
%% it is based on a (0,10*sqrt(3),10) vector in 1900, and thus an abs.vel.
%% of 20 km/s at an angle of 30 degrees between the Y (dec) and Z (radial?) vector
%% ???

%Convert bary to geo, add rotational motion, dot with source vector, add frame motion dotted with source vector.
%From BIMA ephemeris.c (see also slalib rvlsrk.c)
%DOUBLE PRECISION VLSR(3) /.29000685d0,-17.31726682d0,10.00140835d0/

\subsection{Unit Conversion and Conformable Quantities}
The GNU Units library has been
adopted for unit conversion and has been integrated into CARMA
via carma::services::Units.  The file {\tt \$CARMA\_TOOLS/share/units.dat}
contains the possible conversions and is fairly comprehensive;
for instance the default installation already has  Janskys.
Any additional conversions needed are added by editting units.dat,
no recompilation is necessary.

Two quantities are said to be conformable if they can be converted
between each other by a scale change.  We have implemented 
the ConformableQuantity class and derived classes (e.g., Angle, HourAngle,
Length, Distance, Temperature, Velocity) for use in interfaces
to reduce the chance of errors due to units (see Mars Observer).

\subsection{Antenna Location and Pointing Solutions}

Antenna locations will be determined from interferometric data
acquired by Atomic Commands.   Since we have plenty of antennas for
self-calibration,  we will adopt the BIMA method of antenna-based
solutions.  The quickest way to accomplish this is to export as
MIRIAD and use the MIRIAD task BEE.  BEE is a sophisticated tool
for fitting antenna positions on baselines to 2 km.  Shell scripts
and WIP graphical plotting package have helped automate the process.
{\em This requires the MIRIAD filler be available
before well before first light.}

This package will provide routines to convert between commonly used
antenna coordinate systems.  The class {\tt sza::util::Coordinates} 
was migrated to {\tt carma::services::AntennaCoordinates}.

The OVRO, BIMA, and SZA antennas will retain their current pointing
models. The Antenna API specifies a PointingModel that encompasses
all three. The coefficients for each pointing model equation
will then be determined  with observations as described above.
It will not be necessary to create new fitting software; the
fitting software of each will be reused. For BIMA radio pointing,
this is the MIRIAD task PNT and some associated shell scripts.
The input to PNT is an ASCII file containing results of gaussian
fits to 5-point interferometric observations of planets and point
sources (the output of XPNT).  PNT will be modified to remove the
fitted refraction term, since that will be determined separately.
At OVRO, radio and optical data were reduced by a Fortran program
that runs on the VaxStation; this code must be ported to Linux.
An alternative is to modify PNT to handle both BIMA and OVRO
pointing models.  The SZA has written its own fitting program.

The radio and optical refraction corrections will be computed
using the ALMA prescription (Mangum 2001).  These formulae have
subarcsecond accuracy down to 2 degrees elevation.  They have not
yet been coded up by ALMA (Mangum, private communication), so no
reuse is possible here.  The class {\tt carma::environment::Atmosphere}
already contains the refractivity calculation, since it was needed
for the Delay Engine.  Code has been implemented in Atmosphere to calculate
the refraction corrections (refractivity times the elevation mapping
function), with the mapping function as described by Yan (1996)
and Mangum (2001).  See the Interferometry design document for
detailed discussion of refraction.

\section{MAJOR CLASSES}

The following classes exist in the {\tt carma::services} namespace:
\begin{itemize}

\item {\tt Table} - Base class for all ASCII table operations;
already implemented.

\item Catalogs - {\tt Catalog, SourceCatalog, FluxCatalog, SpectralLineCatalog},
as described above.

\item {\tt AntennaCoordinates} - 
Converts between various antenna location coordinate systems; 
actual migrated from of {\tt sza::util::Coordinates}.

\item {\tt Ephemeris} - load the JPL ephemeris tables for
selected sources and calls NOVAS routines directly.


\item {\tt AstroTime} - Astronomical Timekeeping.
The most important methods are those to compute sidereal times from UTC.
The inputs are the MJD on the UTC timescale; UT1-UTC will be internally
supplied via IERS tables.  Example usage:

\footnotesize
\begin{verbatim}
// instantiate a location from Observatory.cat
Location carma("CARMA"); 
// instantiate an AstroTime for this location
AstroTime time(carma);

// get the current LST
double lst = time.localSiderealTime();

// get the current hour angle for 3c273
SourceCatalog cat; 
cat.open(SourceCatalog::defaultCatalog());
Source source = cat.lookup("3c273");
HourAngle ha  = time.hourAngle(Time::MJD(),source.getXCoordinate())
\end{verbatim}
\normalsize

%NOTE: SiteInfo is unused: suggest we delete it from CVS.
%carma::services::Location is used in Ephemeris and elsewhere
%
%The SiteInfo class will be a wrapper for the NOVAS {\tt site\_info}
%structure.
%\begin{verbatim}
%SiteInfo(double latitude, double longitude, double altitude, 
%         string name = "CARMA");
%\end{verbatim}

\item {\tt carma::environment::Atmosphere} -- The Atmosphere class is
already well-developed in support of the Delay Engine. We refer
readers to the CARMA API web page for details. Here we highlight
the suggested public method for the refraction correction,

\footnotesize
\begin{verbatim}
double computeRefractionCorrection(double elevation,
                                   double airTemp, 
                                   double atmPressure,
                                   double relHumid,
                                   double frequency);
\end{verbatim}
\normalsize

This method will return the refraction pointing correction in
radians, to be added to the computed {\em in vacuo} elevation.
The correction will be appropriate for optical or radio depending
on the value of frequency ($\nu > 3$~THz = $100 \mu$m is considered
``optical'').  Additional private methods that compute various
portions of the refraction correction are also implemented.

Atmosphere also implements the conversion routines between
Ambient Temperature, Dewpoint Temperature and Humidity:

\footnotesize
\begin{verbatim}

double computeHumidity(double airTemp, double dewTemp, int method=0);

double computeDewpoint(double airTemp, double Humidity, int method=0);

\end{verbatim}
\normalsize

\item {\tt Source} 
The Source class is derived from CatalogEntry and represents
a source from the CARMA catalog.
Units of Source member variables will follow the FK5 conventions.
It is instantiated primarily using Conformable Quantities.

\footnotesize
\begin{verbatim}
class Source ( 
            const std::string&  name,
	    const Angle&        xCoord,
	    const Angle&        yCoord,
	    const Velocity&     velocity,
	    const Distance&     distance,
	    coordSysType        coordSys = RADEC,
	    double              xProperMotion = 0.0,
	    double              yProperMotion = 0.0,
	    const std::string&  catalogFormat = "CARMA",
	    unsigned long  	idNo = 0,
	    const std::string&  comments = "This Space For Rent"
	     )
\end{verbatim}
\normalsize
An alternative constructor substitutes the Conformable Quantity
{\tt carma::services::Parallax} for Distance.
To support velocity specifications a separate Velocity class has
been created.  It is intended that this class will handle
conversions between various velocity reference frames
and definitions.

\item {\tt Units} -- 
The Units class is simple and easy to use.  The method
{\tt double convert(value, char *convertFrom, char *convertTo)} returns
a converted value. For instance:

\footnotesize
\begin{verbatim}
     Units u ;
     double valDegrees = 123.4;
     double valRadians = u.convert(valDegress,"degrees","radians");
\end{verbatim}
\normalsize

\noindent There are also methods for standard scale factors 
e.g., {\tt milli(), micro(), giga()}.

\item {\tt Conformable Quantities}
Angle, Delay, Distance, Frequency, HourAngle, Length, Temperature, 
[Pressure], Velocity.
These wrapper classes handle commonly used quantities.
They are handy in constructors and at top-level interfaces, 
but should be used with caution if nested deeply in your
classes if performance is an issue.
Best is to advertise the units to use and let the clients deal with it.

\item {\tt Vector, Matrix} -- mathematical vector and matrix operations.


\end {itemize}

\section{PROGRAMS}

Here we briefly list the programs that Auxilliary Services provides. Some
of the {\tt CARMA/carma/services/Test/} programs are also worth mentioning
as they are able to help with debugging or performance issues.

\subsection{checksource}

This program simply shows where the source is in topocentric Ra/Dec and Az/El,
it can produce this for a number of specific times. Example:

\footnotesize
\begin{verbatim}

  % checksource source=mars mjd=+0.5 observatory=ovro nsteps=10 step=0.1 
    Time              Ra (topo)    Dec        Az        El
2005-05-03T05:36:08.6 22:16:36.33 -12:28:28.20 58:29:08.24 -51:37:35.63
2005-05-03T08:00:08.6 22:16:53.25 -12:27:01.03 87:23:19.04 -24:15:56.71
2005-05-03T10:24:08.6 22:17:10.04 -12:25:33.99 108:53:38.78 04:09:29.18
2005-05-03T12:48:08.6 22:17:26.71 -12:24:06.80 135:38:00.79 28:17:37.26
2005-05-03T15:12:08.6 22:17:43.27 -12:22:39.24 176:04:30.41 40:19:23.28
2005-05-03T17:36:08.6 22:17:59.82 -12:21:11.21 218:35:04.82 31:33:47.32
2005-05-03T20:00:08.6 22:18:16.44 -12:19:42.77 247:16:49.21 08:41:52.28
2005-05-03T22:24:08.6 22:18:33.17 -12:18:14.11 269:00:12.10 -19:18:16.36
2005-05-04T00:48:08.6 22:18:50.03 -12:16:45.49 295:18:36.94 -47:11:58.12
2005-05-04T03:12:08.6 22:19:06.98 -12:15:17.12 353:15:57.31 -64:53:05.00

\end{verbatim}
\normalsize


\subsection{distance}

Useful to compute the angular distance between two sources, of which the default 
reference source is the sun, therefore giving us the phase of a planet

\footnotesize
\begin{verbatim}
  % distance source=mars nsteps=10 step=7
# Time                  sun :  az/el           mars :  az/el          angular distance
2005-05-02T17:39:56.3 118.2070044 54.00587420 219.0904067 31.02387906 71.22194157
2005-05-09T17:39:56.3 115.8390133 55.41543736 222.1412729 31.43829553 72.93436185
2005-05-16T17:39:56.3 113.5080883 56.53793565 225.3370556 31.76112806 74.64436707
2005-05-23T17:39:56.3 111.3156542 57.36960494 228.6558067 31.97323827 76.35719420
2005-05-30T17:39:56.3 109.3636946 57.91589638 232.0718082 32.05976503 78.08348411
2005-06-06T17:39:56.3 107.7531986 58.19142634 235.5591583 32.00345956 79.84070620
2005-06-13T17:39:56.3 106.5787506 58.22334772 239.0908102 31.78852910 81.64249957
2005-06-20T17:39:56.3 105.9096966 58.04375769 242.6375573 31.40593144 83.49912267
2005-06-27T17:39:56.3 105.7793693 57.68134499 246.1716099 30.85125322 85.42517596
2005-07-04T17:39:56.3 106.1936268 57.15913529 249.6699786 30.11521849 87.44527736

\end{verbatim}
\normalsize



\subsection{Test/tEphemeris}

This programs does much like {\tt checksource} except much more verbatim. It also have
a silent mode, e.g. to check the influence of I/O accross NFS  if ``large'' ephemeris need
to be read.

\footnotesize
\begin{verbatim}
 % carma/services/Test/tEphemeris
\end{verbatim}
\normalsize

\subsection{Test/tIERSTable}

Helpful to debug the current IERS table. Prints min and max valid times, and check 
interpolation into the table. This program needs a direct link to the IERS table
file, not relative to some CARMA, e.g.

\footnotesize
\begin{verbatim}
  % carma/services/Test/tIERSTable in=$CARMA/conf/catalogs/IERS.tab mjd=53491
53491 -0.59399 -0.0695 0.2727
\end{verbatim}
\normalsize

prints out the MJD in days, dut1 in seconds, X-polar and Y-polar motion both in arcsec.

\section{EXAMPLE TABLES}

In this section we show some example tables that have already been in use 
within BIMA/OVRO/SZA. The equivalent CARMA versions have been placed
in {\tt CARMA/conf/catalog}, where {\tt CARMA} is the top of the
source, build or install tree. The {\tt Program::getConfFile(const string\& name)}
provides a convenient interface to get access to these tables.

\begin{table}[!h]
\label{t-catalogs}
\caption{CARMA catalogs in CARMA/conf/catalog} 
\begin{tabular}{lll}
\hline
Filename  & Function & Interface\\
\hline
IERS.tab         &  IERS table of dut1, and polar motion    & IERSTable\\
Observatory.cat  &  list of stations and their lat,lon,elev & Location \\
SpectralLine.cat &  list of spectial lines and their frequency  & SpectralLineCatalog \\
SystemSource.cat &  list of sources, sky positions and slow motions & SourceCatalog \\
\hline
\end{tabular}
\end{table}


\subsection{Lovas line catalog}

The Lovas spectral line 
catalog\footnote{See also {\tt http://physics.nist.gov/cgi-bin/micro/table5/start.pl}}
comes with MIRIAD.
It has a somewhat complicated structure and 
contains a list of all identified and un-identified interstellar molecular
spectral lines between about 10 MHz and 800 GHz.

\footnotesize
\begin{verbatim}
 Frequency   Unc. Formula      Quantum                      Tr(K)  Source          Telescope    Astr.  Lab.
 (MHz)                                                      /Ta(K)                              ref.   ref.
 
 
  200809.32 *( 3) SO2          16(1,15)-16(0,16)             4.87  OriMC-1         NRAO     12m Jew89
  200888.30 *(10) SO2          13(5,9)-14(4,10) v2=1         0.28  OriMC-1         NRAO     12m Jew89
  200913.79 *( 4) HC3N         22-21 v7=1 l= 1f              0.73  OriMC-1         NRAO     12m Jew89
U 200936.         unidentified    (U204070?)                 0.50  OriMC-1         NRAO     12m Jew89

   68972.154*( 4) SO2          6(1,5)-6(0,6)                 0.8   OriMC-1         NRAO     11m Joh76
U  69460.         unidentified                               0.18  OriMC-1         NRAO     11m Tur89
   69464.094*( 9) SO2          14(4,10)-15(3,13)             0.70  OriMC-1         OSO      20m Sch83

\end{verbatim}
\normalsize

The table has been reformmatted slightly and made available as
{\tt conf/catalogs/SpectralLine.cat} for automated access using 
{\tt carma::services::SpectralLine} and 
{\tt carma::services:SpectralLineCatalog}.
\newline
{\it This is still the 3mm catalog, we don't have the 1mm catalog in here yet}


\subsection{BIMA flux catalog}
\label{fluxtable}
The BIMA flux catalog, also delivered via MIRIAD, was maintained
by astronomers at Hat Creek (mostly Mel Wright and Rick Forster)
from carefully controlled observations.

\footnotesize
\begin{verbatim}
! SOURCES
! sorted by time and RA
!
!Source         Day.UT          Freq (GHz)      Flux(Jy)        Rms(Jy) Ref.
'nmaobs'        92MAY20.0       87.0             1.0            0.2
!
!! 0005+544
'0005+544'      98APR27.0       86.2             0.9            0.4    9
'0005+544'      98OCT22.0       86.2             0.9            0.2    9
'0005+544'      99MAR14.0       86.2             0.4            0.2    9
..
'0005+544'      03JUL01.0       86.2             0.3            0.1    9
'0005+544'      03SEP01.0       86.2             0.4            0.3    9
!
!! 0006-063
'0006-063'      95JUN18.0       86.2             1.4            0.9    7
...

\end{verbatim}
\normalsize

\subsection{sza ephem files}

SZA ephem files (only shown here for illustrative purposes)
are ASCII ephemeris, one per source,
obtained via the JPL/HORIZON procedure, and look roughly as follows (only
the first 4 columns count)

\footnotesize
\begin{verbatim}
#
# MJD (TT)   Right Ascen    Declination   Distance (au)      Calendar (TT)
#---------  -------------  -------------  ------------     -----------------
53005.0000  00:33:35.2643  +03:41:19.071  1.1112017627  #  2004-Jan-01 00:00
53006.0000  00:35:47.3868  +03:56:55.925  1.1200841033  #  2004-Jan-02 00:00
53007.0000  00:37:59.9236  +04:12:32.155  1.1289840401  #  2004-Jan-03 00:00
...
53369.0000  16:03:50.9833  -20:34:42.059  2.2622996871  #  2004-Dec-30 00:00
53370.0000  16:06:43.6583  -20:43:12.186  2.2558736835  #  2004-Dec-31 00:00
53371.0000  16:09:36.7800  -20:51:32.042  2.2494140942  #  2005-Jan-01 00:00
\end{verbatim}
\normalsize

Again, this table format is not supported in CARMA.

\subsection{sza source catalog files}

SZA source catalogs (only shown here for illustrative purposes)
contain a frame of reference (e.g. {\tt FIXED}
(meaning AZEL) and {\tt J2000}), source name, ra, dec and
optional proper motions in ra and dec;


\footnotesize
\begin{verbatim}

#----------------------------------------------------------------------------
#     Name       RA          DEC         PMRA (hrs)   PMDEC (deg)   Magnitude
#----------------------------------------------------------------------------
J2000 TheOct    00:01:35.71 -77:03:56.9 -0:0:00.0042 -0:0:00.177 # 4.78
J2000 30Psc     00:01:57.60 -06:00:51.1 +0:0:00.0034 -0:0:00.041 # 4.41
J2000 2Cet      00:03:44.40 -17:20:10.0 +0:0:00.0017 -0:0:00.009 # 4.55
...
alias sirius   AlpCMa
...
FIXED zenith   *  90.0 *
FIXED nadir    * -90.0 *
FIXED horizon  *   0.0 *
J2000 npole2000   00:00:00.0     90:00:00
...
EPHEM sun      $SZA_DIR/ephem/sun.ephem


\end{verbatim}
\normalsize
%$
Again, these conventions are not supported in CARMA.

\section{NOTES}

% used to be unresolved issues, now they are notes :-)
% {\em PJT: NOTE FOR CDR WE ARE NOT ALLOWED TO HAVE ANY UNRESOLVED ISSUES.}

\begin{itemize}
%\item Deployment: Where will catalogs, ephemerides be located on disk,
%and how will programs find them?  Except for test tables, we don't want 
%these in CVS. DE405 are binary tables. If we would use HORIZON based
%tables, they can be ASCII but probably need more frequent updates

\item procedures for new solar system objects (comets etc.). We can
create them as DE405-compatible binary files, or through their orbital
elements. Either way, a new service routine {\tt solsysN} 
in NOVAS (the way this is normally done) will be needed for this.

%\item should the ephemerides be a client/server, as we do at HatCreek
%(the big NAIF ephemeris files put quite a latency on the system
%as they need to be loaded and interpreted each time an observing
%command is given). The current lookup time is around 0.1ms, once
%the ephemeris is loaded, and any subsequent NOVAS transformations
%to get it to AzEl will supersede this.

\item the transmitter should be a special source, it has (E,N,U)
antenna coordinates and on a per-antenna basis will have the proper
local AZEL computed.

%\item separation of off-site and on-site services needs to be made
%clearer and prioritized.  
% NOT THE JOB OF THIS PACKAGE: OBSERVER TOOLS
% HANDLES THIS USING THE INTERFACES WE PROVIDE.

%\item which RDBMS is meant for the fluxcal database: doing it at
%the site with the pipeline, or at NCSA in the real pipeline?
%Also: the automatically harvested database (cf. 2.7.5-R1) should
%be turned into an offsite database with quality control (i.e.
%an astronomer needs to look at the data and decide on proper
%error bars - this is the way BIMA has always distributed their
%evolving flux catalog, and is maintained with MIRIAD).
% NOT THE JOB OF THIS PACKAGE: OBSERVER TOOLS
% HANDLES THIS USING THE INTERFACES WE PROVIDE.

\item NOVAS 3.0 coming out ``soon'', we use the C library (2.0), the
current Fortran library is 2.9. So there is always a lag between 
Fortran (which is what Kaplan writes) and 
and C (written by Bangart). Bangert reported the C 3.0 version to
be available early 2005.

%\item proper motion now done in mas/year, from the old style arcsec/year
%(However FK5 is arcsec/Julian century.)

\item IERS may discontinue use of leap seconds in the future. 
% Cross that bridge when we come to it.

%\item (How) Do we deal with name aliases in ASCII tables?  Not a problem for
%RDBMS.  sza has special entries 
%such as {\tt alias 3C286 J1331+305} in their source table.

%\item Whose responsibility is it to provide higher-level interpolator 
%classes?  Quad interpolator good enough for e.g. IERS table lookup?
%YES: SZA ALREADY USES QUAD INTERP FOR DUT. IT CHANGES QUITE SLOWLY.

\end{itemize}

\section{FTE ESTIMATE AND MILESTONES}
The functionality of this package has expanded since the conceptual
design review (addition of antenna coordinates, sky coordinates, 
time, and units). The schedule for completion and
FTE estimate is as follows:
   
\begin{tabular}{lrcll}
\hline
Stage & & Months & Responsible & Release (Completion Date) \\
\hline
Work Package Analysis     & & & & 08/27/2002\\
Conceptual Design Review  & & & & 09/06/2003 \\
Preliminary Design Review & & & & 09/02/2004 \\
Critical Design Review    & & & & 05/05/2005 \\
\hline
Design  & & 2.0 & all \\
Implementation 
& Antenna Coordinates & 0.5 & Teuben & R1\\
& Intelligent Tables  & 0.5 & Teuben & R1 \\
& Refraction          & 0.25 & Pound  & R1 \\
& Time Conversion     & 0.5 & Teuben & R1 \\
& Units               & 0.5 & Gwon   & R1 \\
& Ephemerides         & 0.5 & Teuben & R2 \\
& Source Catalog      & 0.25 & Gwon  & R3 \\
& Line Catalog        & 0.25 & Gwon  & R3 \\
% -- 
& Doppler Correction  & 0.25 & Pound & R4 \\
& Baselines           & 0.5 & Teuben & R4 \\
& Pointing            & 0.5 & Pound  & R4\\
& Planet Brightness   & 0.25 & Teuben & R4\\
& Flux History        & 0.25 & Gwon  & R5\\
& Observing Modes      & 0.25 & Pound & R5\\
& Total             & 7.5 & \\
Testing & & 1.0 & all\\
Integration & &  1.0 & all\\
\hline
Package Total & & 9.5 & FTE months  \\
\end{tabular}

\section{HISTORY}

\begin{itemize}
\item v2.8 - last active version during initial development (May 2005)
\item v3.0 - various doppler and optical/radio restframe clarifications
  and implementions.  Also coinciding wiht this, the conversion to the 
  NOVAS V3 API (RADVL) that implements their new radial velocities
  implementation) - August 2012.

\end{itemize}

%%%%
% this style with natbib gives problems with the newer version (2009) of natbib.sty
% as you can find it on e.g. ubuntu 12.04
% the 2007/02/05 8.0 (PWD) version works ok
%     2009/07/16 8.31 (PWD, AO) - has a problem

\begin{thebibliography}{}

\bibitem[]{} Astronomical Almanac - USNO. See URL=
\bibitem[]{} Kaplan, G.H. et al. 1989, AJ, 97, 4
\bibitem[]{} Lovas, F.J. - {\tt http://physics.nist.gov/cgi-bin/micro/table5/start.pl}
\bibitem[]{} Mangum, J. 2001, ALMA Memo 366
\bibitem[]{} Newhall, X. X.; Standish, E. M.; Williams, J. G. 1983, Astron. Astrophys. 125, 150-167. (DE102)
\bibitem[]{} Scott, S. \& Woody, D. 2002, {\sl CARMA Time Distribution}, \\
{\tt http://www.mmarray.org/project/WP/MasterClock/hw/TimeDistribution.pdf}
\bibitem[]{} Standish, E.M..: 1998, "JPL Planetary and Lunar Ephemerides, DE405/DE406", 
JPL IOM 312.F-98-048.
\bibitem[]{} Thompson, Moran, \& Swenson, 1986 (1st edition), publ. Wiley
\bibitem[]{} Wright, M.C.H.W 1990, BIMA Memo 2
\bibitem[]{} Yan, H. 1996, AJ, 112, 1312

\end{thebibliography}



\end{document}
