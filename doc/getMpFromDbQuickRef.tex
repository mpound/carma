\documentclass[preprint]{aastex}

%
% Greater than or approximately equal to and less than or
% approximately equal to signs, using macros defined in PLAIN.TEX
%
\catcode`\@=11 \def\gapprox{\mathrel{\mathpalette\@versim>}}
\def\lapprox{\mathrel{\mathpalette\@versim<}}
\def\@versim#1#2{\lower2.9truept\vbox{\baselineskip0pt\lineskip0.5truept
    \ialign{$\m@th#1\hfil##\hfil$\crcr#2\crcr\sim\crcr}}}
\catcode`\@=12

\begin{document}
\title{Quick Reference for {\tt getMpFromDb}}
\author{Chul Gwon}
\slugcomment{Version \today}

\section{Overview}
{\tt getMpFromDb} (formerly known as {\tt carmaMonitorDataExtractor})
is a command line utility to query monitor data that has been stored
in the DBMS and print it to standard output. To use getMpFromDb on the
production system, the keywords that will be essential will be {\tt
mp}, {\tt start}, {\tt end}, and {\tt singletable}:

\noindent {\tt mp} is used to set the search criteria for mointor
points.  It can handle both wildcards and comma-separated monitor
points.

{\tt mp="Ovro*Xac*, Bima*, MasterClock.Host.*"}

\noindent The keywords{\tt start} and {\tt end} specify the start and
end time over which the search should occur.  They can be in any
format that is allowed by the unix {\tt date} command.  These are
essential for controlling the number of monitor points that are
retrieved.  Not specifying a start and end time will try to retrieve
everything, which will throw an exception on the production system
because there are just too many entries.  {\tt start} and {\tt end}
consist of a date and an time.  The format is defaulted to {\tt
\%Y-\%b-\%d} for the date and HH:MM:SS for the time.

{\tt start="2005-Feb-04 23:42:42" end="2005-Feb-04 23:42:44"}

{\tt getMpFromDb} currently allows retrieved monitor data to be
displayed two different ways: listing monitor points individually in a
table, including all attributes, or listing monitor points together in
a single table for a specified attribute.  Switching between the two
displays is controlled by the {\tt singletable} keyword.  The default for
this keyword is {\tt true}, all monitor points in a single table for a
specified attribute (controlled by the {\tt attr} keyword and
defaulted to {\tt "value"}). 

\begin{verbatim}
\%~>  bin/getMpFromDb mp="Master*V" start="2005-Feb-04 23:42:41" colwidth=10
Time to execute set up queries 0.0300921
Time to execute wide set up queries 2.89327e-06
Time to execute wide  query 0.0024799


Comparing the following Monitor Point value: 
            Monitor Point Canonical Name     TagID
       MasterClock.MasterClock.digital5V   1835074
           MasterClock.MasterClock.ps24V   1835071
           MasterClock.MasterClock.ps12V   1835072
        MasterClock.MasterClock.analog5V   1835073

     Date/Time (UTC)   1835074   1835071   1835072   1835073
2005-Feb-04 23:42:42         5        24        12         5
2005-Feb-04 23:42:42         5        24        12         5
2005-Feb-04 23:42:42         5        24        12         5
2005-Feb-04 23:42:43         5        24        12         5

\end{verbatim}

A user may also obtain separate tables for each individual monitor
point by setting {\tt singletable} to {\tt false}

\begin{verbatim}
\%~> bin/getMpFromDb mp="Master*V" start="2005-Feb-04 23:42:41" singletable=false
Time to execute set up queries 0.0295561
Time to execute narrow queries 0.00333605
Monitor Point: MasterClock.MasterClock.ps24V
     Date/Time (UTC)     value        blankingFlag            validity    iSample
2005-Feb-04 23:42:42        24        UNDETERMINED   VALID_NOT_CHECKED          0
2005-Feb-04 23:42:42        24        UNDETERMINED   VALID_NOT_CHECKED          0
2005-Feb-04 23:42:42        24        UNDETERMINED   VALID_NOT_CHECKED          0
2005-Feb-04 23:42:43        24        UNDETERMINED   VALID_NOT_CHECKED          0

Monitor Point: MasterClock.MasterClock.ps12V
     Date/Time (UTC)     value        blankingFlag            validity    iSample
2005-Feb-04 23:42:42        12        UNDETERMINED   VALID_NOT_CHECKED          0
2005-Feb-04 23:42:42        12        UNDETERMINED   VALID_NOT_CHECKED          0
2005-Feb-04 23:42:42        12        UNDETERMINED   VALID_NOT_CHECKED          0
2005-Feb-04 23:42:43        12        UNDETERMINED   VALID_NOT_CHECKED          0

Monitor Point: MasterClock.MasterClock.analog5V
     Date/Time (UTC)     value        blankingFlag            validity    iSample
2005-Feb-04 23:42:42         5        UNDETERMINED   VALID_NOT_CHECKED          0
2005-Feb-04 23:42:42         5        UNDETERMINED   VALID_NOT_CHECKED          0
2005-Feb-04 23:42:42         5        UNDETERMINED   VALID_NOT_CHECKED          0
2005-Feb-04 23:42:43         5        UNDETERMINED   VALID_NOT_CHECKED          0

Monitor Point: MasterClock.MasterClock.digital5V
     Date/Time (UTC)     value        blankingFlag            validity    iSample
2005-Feb-04 23:42:42         5        UNDETERMINED   VALID_NOT_CHECKED          0
2005-Feb-04 23:42:42         5        UNDETERMINED   VALID_NOT_CHECKED          0
2005-Feb-04 23:42:42         5        UNDETERMINED   VALID_NOT_CHECKED          0
2005-Feb-04 23:42:43         5        UNDETERMINED   VALID_NOT_CHECKED          0
\end{verbatim}

A summary of all keywords that are available to the user are as follows:
\begin{itemize}
 \item {{\tt conffile} ({\tt string}) an alternative DB configuration
       file if you are running with a DBMS on your local machine}
 \item{{\tt mp} ({\tt string}) search string for monitor points of
        interest}
 \item{{\tt start} ({\tt string}) start time for monitor point query}
 \item{{\tt end} ({\tt string}) end time for monitor point query}
 \item{{\tt format} ({\tt string}) format of the date to use in query}
 \item{{\tt singletable} ({\tt boolean}) format for printing out monitor points}
 \item{{\tt attr} ({\tt string}) attribute of the monitor points to display}
 \item{{\tt multiwild} ({\tt string}) desired wildcard for
        multi-character match}
 \item{{\tt singlewild} ({\tt string}) desired wildcard for
        single-character match}
 \item{{\tt colwidth} ({\tt int}) width of columns}
\end{itemize}

In a local sandbox or other developmental environment, it may be
necessary to specify the {\tt conffile} keyword, which points to the
DBMS on your local system.  Column width may be controlled by using
the {\tt colwidth} keyword.  This will change the widths of all
columns except the date.  It is also possible (but not recommended) to
change the characters used for multiple-character and single-character
wildcards.  The defaults are currently {\tt *} and {\tt ?}
respectively, but can be modified using the {\tt multiwild} and {\tt
singlewild} keywords.

\appendix
\clearpage
\end{document}