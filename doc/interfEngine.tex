%$Id: interfEngine.tex,v 1.11 2005/01/12 13:56:25 mpound Exp $
\documentclass[preprint]{aastex}
\usepackage{mathrsfs}
\usepackage{epsf}
%\usepackage{makeidx}
%\usepackage{fancyhdr}
%\usepackage{graphicx}
%\usepackage{float}
%\usepackage{alltt}
%\usepackage{doxygen}
% ----------------------DEFINITIONS USED IN THIS PAPER --------------------
\newcommand{\xlam}{
   \ifmmode{X_\lambda}
   \else{$X_\lambda$}
   \fi}
\newcommand{\ylam}{
   \ifmmode{Y_\lambda}
   \else{$Y_\lambda$}
   \fi}
\newcommand{\zlam}{
   \ifmmode{Z_\lambda}
   \else{$Z_\lambda$}
   \fi}
\newcommand{\rearth}{  % Earth radius
   \ifmmode{r_\oplus}
   \else{$r_\oplus$}
   \fi}
\newcommand{\scrL}{ % script L for path length
   \ifmmode{{\mathscr L}}
   \else{${\mathscr L}$\/}
   \fi}
\def\degree{\ifmmode{^\circ} \else{$^\circ$}\fi}
\def\nexpo#1#2{\ifmmode{#1 \times 10^{#2}}\else{$#1 \times 10^{#2}$}\fi}
\def\expo#1{\ifmmode{10^{#1}}\else{$10^{#1}$}\fi}
\def\msun{\ifmmode{\rm M_\odot}\else{M$_\odot$}\fi}
\def\esat{\ifmmode{\rm e_{sat}}\else{e$_{sat}$}\fi}
\def\etal           {{\rm et~al}.\ }
\def\eg             {{\rm e.g.}\/}
\def\ie             {{\rm i.e.}\/}
\def\uv{{\it uv}}
\def\cmm {\ifmmode{{\rm cm^{-3}}}\else{{${\rm cm^{-3}}$}}\fi}
\def\cmtwo {\ifmmode{{\rm cm^{-2}}}\else{{${\rm cm^{-2}}$}}\fi}
\def\invpc{\ifmmode{\,{\rm pc^{-1}}}\else{\thinspace {\rm pc$^{-1}$}}\fi}
\def\invsec{\ifmmode{\,{\rm s^{-1}}}\else{\thinspace {\rm s$^{-1}$}}\fi}
\def\invyr{\ifmmode{\,{\rm yr^{-1}}}\else{\thinspace {\rm yr$^{-1}$}}\fi}
\def\psec           {$.\negthinspace^{s}$}
\def\pmin           {$.\negthinspace^{m}$}
\def\pasec           {$.\negthinspace^{\prime\prime}$}
\def\pdeg           {$.\kern-.25em ^{^\circ}$}
\def\kms{\ifmmode{\,km\,s^{-1}}\else{\thinspace km\thinspace s$^{-1}$}\fi}
%
% Greater than or approximately equal to and less than or
% approximately equal to signs, using macros defined in PLAIN.TEX
%
\catcode`\@=11 \def\gapprox{\mathrel{\mathpalette\@versim>}}
\def\lapprox{\mathrel{\mathpalette\@versim<}}
\def\@versim#1#2{\lower2.9truept\vbox{\baselineskip0pt\lineskip0.5truept
    \ialign{$\m@th#1\hfil##\hfil$\crcr#2\crcr\sim\crcr}}}
\catcode`\@=12
% --------------------------------------------------------------------------
\begin{document}
\title{CARMA Interferometry Subsystem Design}
\author{Marc Pound, Colby Kraybill}
\slugcomment{Version 1.6 (\today)}


\section{Introduction}
\subsection{Functionality and Scope}

The primary purpose of the Interferometry subsystem (aka Delay  
Engine) is the calculation of delays, which are sent on to the correlator
crates and lobe rotator controller.  The delays consist of the fixed
(per array configuration) delay line offset plus corrections for geometry,
thermal expansion, and atmospheric propogation.  The Delay Engine
requires for each antenna source position, observation frequency, station
coordinates, latitude, elevation, yoke height, structural temperature(s),
and axis misalignment.  These are provided from the Control SubArray
Tracker (or perhaps in the future, a Configuration Database).  In addition local
pressure, temperature, and relative humidity are needed from the weather
station via Control.  Also needed is LST to compute the hour angle 
from RA.  To realize the 30 femtosecond delay accuracy needed by CARMA,
all input to, outputs from, and calculations within the delay 
engine will be double precision.

This package has been descoped somewhat since the WP analysis.
The selection of Walsh sequences and communication of them to the
Lobe Rotator is done by the Control Subsystem.  The water line monitor
phase correction is between the WLM DO and the Correlator Pipeline--the
Delay Engine is not involved. Similary, phase corrections for
changing subreflector focus are done in the Correlator Pipeline rather
than focus delays calculated by the Delay Engine.

This document uses a lot of mathematical symbols; they are defined
in Table \ref{t-symbols}.


\subsection{Antenna Coordinate Systems}
There are a number of ways to represent the antenna locations.  It has
been decided that CARMA will present antenna coordinates to the user in
the convenient (East,North,Up) system, which is also used in the array
configuration memo, with conversion routines provided by the Auxiliary
Services system.  However, in order to conform to standard equations in TMS,
in this document and in the Delay Engine Control API (and of course in the
code itself), 
we will use Equatorial coordinates in meters: (\xlam,
\ylam, \zlam) with \xlam in the plane defined by the terrestrial poles
and the reference point of the array, \ylam toward the east and \zlam
toward the NCP.  A future enhancement will allow input of ENU coordinates
in the API, which will be converted to XYZ internally 
(SZA conversion code will be migrated to Auxiliary Services).

\section{Refraction, Refractivity, and Path Length\label{s-refraction}}
Astronomical refraction is the difference between the true and apparent
elevation of an object in the sky. Because of the changing index of
refraction in the earth's atmosphere, the object appears higher than
its true elevation.  This must be taken into account when pointing an
antenna. The correction is not small: the typical value at mid-elevations
is about an arcminute and increases to many arcminutes near the horizon.
At radio wavelengths, a large component of the refraction is due to
water vapor, where it is some 20 times greater than in the optical.
An excellent summary of the refraction pointing correction as it is used
in radio astronomy is given by Mangum (2001).

Because refraction increases the signal travel time, it also comes in as
an additive propagation delay correction.  This is commonly referred to
as ``differential refraction'' since the antennas at different locations
will not observe the source at identical elevation and hence experience
differential delays.  It should not be confused with the commonly
used term for refraction that varies at a function of wavelength
(an important effect in the optical regime).  To keep things clear,
in this document, we will use the term atmospheric propagation delay.
Atmospheric propagation delay is derived in Section \ref{s-atmosdelay}.
First, we turn our attention to astronomical refraction, which will
lay the groundwork for deriving the delay.

Formally, the astronomical refraction is the integral of the
index of refraction, $n$, over the signal path:

\begin{equation}
R = \int_{1}^{n_0} dn \frac{\cot E(n)}{n}
\label{e-Rdef}
\end{equation}

\noindent where $E$ is the source elevation.  Using Snell's Law a general
formula for refraction in a spherically symmetric atmosphere can be obtained
(Smart 1972)

\begin{equation}
R = \int_{1}^{n_0} dn \frac{n_0 r_0 \cos E_0}
{n \sqrt{n^2 r^2 - n_0^2 r_0^2 \cos^2 E_0} }
\label{e-smart}
\end{equation} 

\noindent where $r$ is the geocentric distance to the layer with index of
refraction $n$, and $n_0$, $r_0$, and $E_0$ are the measured index of
refraction, geocentric distance, and apparent elevation at the antenna 
location.  The general prescription
for treating refraction involves an analytical form (usually a series expansion)
of $R$ with separable
terms that depend on elevation and local atmospheric conditions.

\begin{equation}
 R(P,T,RH,E) = R_0(P,T,RH) f(P,T,RH,E_0), 
\label{e-Rsep}
\end{equation}

\noindent where $P$, $T$, and $RH$ are the local atmospheric pressure,
temperature, and relative humidity.  At some telescopes, this is
simplified even further:
\begin{equation}
 R(P,T,RH,E) = R_0(P,T,RH) f(E_0).
\label{e-Rsepsimple}
\end{equation}
\noindent Since departures of $n$ from
unity are small, the refractivity $N_0$ 
(again at the antenna location) is typically used instead:

\begin{eqnarray}
N_0 &\equiv& 10^6~(n - 1)\\
  &=& 10^6~R_0 (radians)\\
  &=& \frac{R_0 (arcsec)}{0.206265}.
\label{e-Ndef}
\end{eqnarray}

\noindent The refractivity is an important component in calculation of
atmospheric propogation delay; the delay is directly proportional to the
refractivity integrated through the atmosphere (i.e., the pathlength \scrL). It may be broken into
two parts: the optical refractivity due to induced dipole ultraviolet
transitions of dry air and water vapor, and the infrared refractivity
due to permanent dipole infrared rotational transitions of water vapor.
The weather-dependent formula given by Smith \& Weintraub (1953) and Thayer
(1974) is used by both BIMA and OVRO (see TMS equations 13.6 and 13.70):

\begin{eqnarray}
N_{opt} &=& N_{opt,D} + N_{opt,V} \\
	&=& 77.6 \frac{p_D}{T} +~64.8 \frac{p_V}{T} \\
\label{e-dry}
N_{IR} &=& 3.776\times10^5 \frac{p_V}{T^2}\\
\label{e-wet}
N_0 &=& N_{opt} + N_{IR}
\label{e-sw}
\end{eqnarray}

\noindent where $p_D$ is the partial pressure of the dry air and $p_V$
is the partial pressure of water vapor, both in millibars and $T$ is
in Kelvin.  BIMA ignores the second optical term $N_{opt,V}$ (by
design or accident?), which is small compared with the IR refractivity.
For CARMA, we will keep all terms.\footnote{The International Association of Geodesy
website suggests coefficients for equation \ref{e-sw} which include
a contribution from CO$_2$ and differ by a few tenths of a percent from
those given here. Of course, our code will be written such that changes
in the formula for $N_0$ due to new research can be accomodated.}
It is not uncommon to see the components of $N_0$ split up with
the terms containing $p_V$ being together called the ``wet refractivity''
($N_V$) and the $p_D$ term call the ``dry refractivity'' ($N_D$).

The atmospheric pressure measured by the on-site barometer is the sum
of the partial pressures $P = p_D + p_V$, and $p_V$ is calculated from
the relative humidity in percent and the surface saturated water vapor
pressure (millibars):

\begin{equation}
p_V = RH \frac{\esat}{100}.
\label{e-pv}
\end{equation}

\noindent The formula for \esat\/ used by both BIMA and OVRO is that given
by Crane (1976) (see TMS eqn 13.15):

\begin{equation}
\esat = 6.105~\exp \Bigl[ 25.22~\Bigl({T~-~273.2 \over T }\Bigr)~-~5.31 
~\ln \Bigl(\frac{T}{273.2}\Bigr) \Bigr] .
\label{e-esat}
\end{equation}
\noindent (OVRO writes it as a function of dewpoint temperature, 
but the result is the same). Substituting into 
equation \ref{e-sw} and rewriting in terms of total pressure:
\begin{equation}
N_0 = 77.6~\frac{P}{T}~-~0.128~\frac{RH~\esat}{T}~+~3.776\times10^3~\frac{RH~\esat}{T^2}
\label{e-fullN}
\end{equation}

Note that $N_0$ is the refractivity at the earth's surface and looking
towards the zenith.  To compute the refractivity integrated through the earth's
atmosphere and towards the source (that is, the path length \scrL) 
we must take into account the differing scale heights of
the neutral atmosphere and the water vapor (i.e., of  the wet and dry
refractivities) as well as the fact that the atmosphere is spherical.
Below, we take the neutral scale height to be $h_D$ and water vapor scale
height to be $h_V$.

At both BIMA and OVRO the sphericity is approximated by assuming
the atmosphere consists of spherical layers with differing indices
of refraction.  OVRO does an iterative spherical shell approximation to 
a full ray-tracing solution at the source elevation assuming a
scale height that is the weighted sum of the scale heights
\begin{equation}
h_{OVRO} = \frac{h_D~N_D + h_V~N_V + h_{IR}~N_{IR}}{N_D + N_V + N_{IR} } 
\end{equation}
\noindent 
OVRO takes the infrared water vapor scale height $h_{IR} \approx 1.05 h_V$.
BIMA assumes $h_{IR} = h_V$ and 
opts for the approximation, good for 
short baselines ($D<<\rearth$)
and moderate zenith angles ($E_0 \gapprox 20\degree$; TMS equation 13.33):

{\bf FIX: missing a length term -- check jack's notes }
\begin{equation} 
\Delta\scrL_{BIMA} =  \expo{-6} \left( N_{opt,D}~h_D + N_{IR}~h_V \right)~
\left(\frac{1}{\rearth~\sin^2~E_0}\right) .
\label{e-BIMApathlength}
\end{equation}
\noindent In this case $\Delta\scrL$ is the pathlength difference between the 
antenna location and the array reference position.
For CARMA, we will follow the prescription given by TMS equation 13.38--13.41:
\begin{equation}
\scrL = 10^{-6} 
\frac{N_0~h_0}{\sin E_0} \left(1 - \frac{h_0}{\rearth} \cot^2 E_0 \right),
\end{equation}

\noindent where $h_0$ is the atmospheric scale height. Substituting in from equations \ref{e-sw} and \ref{e-fullN}:

\begin{equation}
\scrL  = 10^{-6} 
\left\{
\frac{N_D~h_D}{\sin E_0} \left(1 - \frac{h_D}{\rearth} \cot^2 E_0 \right) 
 +~\frac{N_{opt,V}~h_V}{\sin E_0} \left(1 - \frac{h_V}{\rearth} \cot^2 E_0 \right) 
 +~\frac{N_{IR}~h_{IR}}{\sin E_0} \left(1 - \frac{h_{IR}}{\rearth} \cot^2 E_0 \right)
\right\}
 \label{e-pathlength}
\end{equation}

\begin{eqnarray*}
{\mathscr L} &=& \left(\frac{10^{-6}}{T \sin E_0}\right) 
  \left\{ 
    77.6~P~h_D \left[ 1 - \frac{h_D}{\rearth} \cot^2 E_0 \right]
 -~0.128~RH~\esat~h_V \left[ 1 - \frac{h_V}{\rearth} \cot^2 E_0 \right]
\right. \\
 & & \left. 
 +~\nexpo{3.776}{3}~RH~\esat~h_{IR}~T^{-1} \left[ 1 - \frac{h_{IR}}{\rearth} \cot^2 E_0 \right]
  \right\}. 
\end{eqnarray*}

\noindent This formula is accurate to better than 0.4\% at
$E_0 \ge 10\degree$ and better than 2\% at $ 7\degree \le E_0 < 10\degree$.
Since equation \ref{e-pathlength} is singular at $E_0 = 0$, we must have
a safeguard at low elevations. We suggest that for $E_0 < E_{min}$, we
set $\scrL = \scrL_{max}$ as defined in TMS equation 13.40.
The interferometry DO will also need to keep reasonable default values
for atmospheric pressure, air temperature, and relative humidity in case these
are unavailable or otherwise corrupted (e.g. weather station inoperative).
Note $E_0$ correponds to the pointing center of the antenna, which
may be different than the phase center (although likely not too different).

\subsubsection{Antennas at Differing Heights\label{s-zheight}}
Implicit in equation \ref{e-pathlength} is the assumption that all antennas
in the array are coplanar.  This is certainly not the case for Cedar Flat,
where the station heights differ by as much as 53 m.  So equation 
\ref{e-pathlength} needs a correction to take this into account. 
The correction is negative, so that antennas at higher stations have
a shorter pathlength (less delay).
With the installation of the A-array stations, which are off the ``T'',
BIMA solved this by adding a term to equation 
\ref{e-BIMApathlength}:
\begin{equation}
\delta\scrL_z = -\frac{R_0~\Delta h}{\sin~E_0}
\label{e-zbima}
\end{equation}
\noindent where $\Delta h$ is the
additional height of the antenna above the reference plane of the array.
Equation \ref{e-zbima} implicitly assumes the bottom of the atmosphere
is a uniform slab.  A more correct prescription would be simply to
replacing \rearth\ with $(\rearth + \Delta h)$ in equation 
\ref{e-pathlength}; this is what we will do for CARMA.
%In equatorial antenna coordinates, this becomes
%\begin{eqnarray}
%\delta\scrL_z & = & -\frac{R_0}{\sin E_0} \left( \xlam \cos L + \zlam \sin L \right) \\
%% & = & -\frac{10^{-6}}{\sin E_0} 
%\left\{ 77.6~\frac{P}{T}~-~0.128~\frac{RH~\esat}{T}~+~3.776\times10^3~\frac{RH~\esat}{T^2} \right\}~
%\left( \xlam \cos L + \zlam \sin L \right).
%\end{eqnarray}

\subsection{Frequency Dependence of the Refractivity}

The IR refractivity has a frequency dependence due to the effects of the
wings of the infrared emission that is not accounted for in equation
\ref{e-sw} (Clifford \& Hill 1981). Because of this, the accuracy of
equation \ref{e-sw} is limited to about 1.5\% at 100 GHz and 3\% at
230 GHz.  The accuracy can be improved by adding a frequency dependence
to the IR refractivity.

An example plot the $N$ as a function of frequency taken from Clifford
\& Hill in shown in Figure \ref{f-hcrefrac}.  Note this curve is for
weather conditions significantly worse than we would expect at Cedar Flat.
What is currently unknown (but knowable) is how this curve changes with
weather conditions.  But the general features of the curve are nonetheless
relevant.  As the figure shows, $N$ is roughly constant up to about 50 GHz,
then slowly increases with frequency, except for a dip at 183 GHz due to
the atmospheric water line. (There are larger dips above 320 GHz, but we
have not shown them since they are outside CARMA's nominal operating range).

\subsubsection{A Possible Implementation for CARMA}

The data in Figure \ref{f-hcrefrac} can be fit well up to 300 GHz by
a second-order polynomial (shown as a dashed line).  The deviation of
the fit from the data in the 3 mm operating range is negligible and in
the 1 mm range is less than 0.3\%.  (A fourth-order polynomial does even
better above 200 GHz).  The largest deviation is at the 183 GHz dip and
is only 0.8\%.  However, CARMA does not have receivers in this operating
range so it is not a problem.  Therefore, given a refractivity curve that
represents average weather conditions at Cedar Flat, an adequate expression 
for the refractivity that meets CARMA's requirements is

\begin{eqnarray}
N(\nu,E_0) & = & N_0 ~ f(\nu) \\
  & = & N_0 ~ (1 + A~\nu + B\nu^2)
\end{eqnarray}
\noindent with $N_0$ as determined from Equation \ref{e-fullN} and 
\[
  A,B = \left\{ \begin{array}{ll}
                   0 & \mbox{$\nu \leq 50$ GHz} \\
                   \mbox{from fit} & \mbox{$\nu > 50$ GHz}
                 \end{array} 
          \right. 
\]
\noindent 
Thus, \scrL and therefore the tropospheric delay would 
have to be multiplied by $f(\nu)$.
If the average Cedar Flat conditions do not provide sufficient accuracy
for $f(\nu)$, then the Liebe atmospheric code (which UMD has) could be
used to improve the correction.   This investigation would fall to
Auxiliary Services (see below).

\subsection{Refraction Pointing Correction}

While the refraction pointing correction $R$ is formally part of Auxiliary
Services, it is related to the refractivity $N_0$ and so we comment upon
it here.  The Requirements document states the commanded antenna positions
need to be accurate to 1$\arcsec$, which translates to knowing $N$ to
1.7\% (assuming typical $R_0$ is 60\arcsec).  Since CARMA will operate
primarily at 230 GHz, we would need to include this frequency dependence
to stay within the required tolerance.  Since we have a valid equation
for $N_0$ and thus $R_0$, determining the refraction pointing correction
$R$ comes down to finding a good formula for the function $f$ in equation
\ref{e-Rsep} or \ref{e-Rsepsimple}.  There is a lot of literature on the
subject (e.g. Marini 1972; Davis et al. 1985; Yan 1996). A review of what
different observatories do is provided by Mangum (2001).

Neither BIMA nor OVRO use an analytical function $f(E_0)$ for the elevation
dependence of $R$.  When fitting pointing data, BIMA includes an
weather-independent elevation term in the pointing model of the form
$\cot E_0$ to account for refraction (Wright 1990).  OVRO does a ray-tracing
approximation like the one it uses for \scrL (for optical pointing 
the IR terms are removed).

For CARMA, we can calculate $R$ from $N_0$ and some to-be-determined
function $f(E_0)$, or use someone else's routines.  For instance,
The SLALIB library has refraction routines (e.g., SLA\_REFRO), to
determine the apparent elevation given true elevation and weather
information.  These use an expansion formula for the elevation
dependence of $R$ and an IR refractivity that is perhaps out of
date, but claims to give accurate results to $E_0 \ge 15\degree$.
However, CARMA needs $N_0$ for two separate calculations, the
refraction angle and the tropospheric propagation delay and there
is no SLALIB routine to calculate the $N_0$ separately from $R$.
So, if we use a canned routine we wind up essentially having two
different methods of determining $N_0$.

In conjunction with the Auxiliary Services package authors,
we have decided that the class \\
{\tt carma::environment::Atmosphere} will calculate
refraction-related quantities \scrL and $N_0$, and Auxiliary Services
will provide $R$ for antenna pointing correction using the ALMA
prescription (Mangum 2001).

Finally, we note once the water line monitors are fully integrated,
\scrL and $R$ could in principal be calculated from WLM data using
an appropriate mapping function (see e.g. Davis et al. 1985)

\section{Interferometric Delay Corrections\label{s-delay}}
The total delay correction is the sum of several components 
\begin{equation}
\delta\tau_{total} = \delta\tau_g + \delta\tau_{atmos} + \delta\tau_{therm} 
%%+ \delta\tau_{focus} 
+ \delta\tau_{adj}
\label{e-delay}
\end{equation}
\noindent and for a given antenna station added to its fixed delay 
line offset, $\tau_0$, to arrive at the total delay
\begin{equation}
\tau_{total} = \tau_0 + \delta\tau_{total}. 
\label{e-totaldelay}
\end{equation}
\noindent The quantity $\tau_0$ is determined through observations after
each array reconfiguration.  The last term in Equation \ref{e-delay},
$\delta\tau_{adj}$ is an ``adjustable'' delay that can be used for
``peaking up'' the delays after an antenna move or as an artificial delay
for engineering tests.  We derive the rest of the terms below.
In practice, we difference the derived delay term
for a given antenna and for the array reference position, so that
the applied delay term is $\delta\tau_{ant} - \delta\tau_{ref}$.


\subsection{Geometric\label{s-geometric}}
The geometric delay correction is the simply the $w$ coordinate times $-1$,
\begin{equation}
 \delta\tau_g = - \frac{1}{c} \left( \xlam~\cos\delta~\cos H - 
                    \ylam~\cos\delta~\sin H +
                    \zlam~\sin\delta \right) +
                    \frac{D_a}{c}~\cos E_0
\end{equation}
\noindent where \xlam, \ylam, and \zlam\ are the antenna position
in equatorial coordinates, $H$ is the phase center hour angle (which
may be different than the pointing center hour angle), and 
$D_a$ is the distance between the antenna azimuth and elevation axes, 
measured perpendicular to the azimuth axis and in the same units 
as the antenna positions (see TMS equation 4.15). This last term takes 
into account axis misalignment (Wade 1970; TMS Section 4.7).

\subsection{Atmospheric Propagation\label{s-atmosdelay}}
During its journey from the radio source to the antennas, the
radio wave must of course pass through the Earth's atmosphere.  Both the
ionosphere and the troposphere modify the propagation speed, resulting in
an overall delay relative to propagation in vacuum.  The parts of the wave
front pass through different atmospheric regions, causing a differential
delay in arrival at two antennas.  This differential delay must
be removed on a per antenna basis.

\subsubsection{Ionosphere\label{s-ionosphere}}
Composed of charged particles, the ionosphere has a plasmatic 
refractive index at frequency $\nu$
\begin{equation}
n = {\left[ 1 - {\left( \frac{\nu_p}{\nu} \right)}^2
           {\left(1 \pm \frac{\nu_g}{\nu}~\cos~\Theta \right)}^{-1}
    \right]}^{\frac{1}{2}}
\end{equation}
\noindent where $\nu_p$ is the plasma frequency, $\nu_g$ is the
electron gyrofrequency, and $\Theta$ is the angle between the
magnetic field and the propagation direction (See Spitzer 1978).  
The associated delay
is the integral of the refractive index over the path through
the ionosphere
\begin{equation}
\delta\tau_{ion} = \frac{1}{c} \int dS (n - 1).
\end{equation}
\noindent Various approximations (Sovers et al. 1998; TMS Section 13.2)
allow the delay to be expressed as a function of frequency and 
zenith distance, $\zeta = 90 - E_0$. 
\begin{eqnarray}
\delta\tau_{ion} &=& -\left(\frac{q}{\nu^2}\right) \sec \zeta_i \\
           q &=& \frac{c r_e}{2 \pi} N_e\\
           \zeta_i &=& \sin^{-1} \left[ \frac{\rearth}{\rearth+h_i} \sin \zeta \right] 
\end{eqnarray}
\noindent where $r_e = \nexpo{2.818}{-13}~{\rm cm}$ is the classical electron 
radius, $N_e \approx \nexpo{6}{13}~\cmtwo$ is the total column
density of electrons at zenith, and $h_i \approx 350$ km is the height of the
ionospheric layer.  Note this additional delay is negative.
At millimeter wavelengths, $\delta\tau_{ion}$ is small due to the inverse-square
frequency dependence; for instance
at 100 GHz it is \nexpo{8}{-3} ns at zenith.  Although $\delta\tau_{ion}$ 
is traditionally ignored at mm wavelengths, this value is comparable to the
antenna misalignment term for the BIMA antennas.  The problem is that
$N_e$ is varies diurnally and with the Solar cycle, making $\delta\tau_{ion}$
difficult to determine.  Point source observations
at widely separated frequencies can be used to remove $\delta\tau_{ion}$
(Sovers et al. 1998; TMS Section 13.2) and we might consider a
regular program for CARMA+SZA to do so (\eg, simultaneous quasar measurements
at 230 GHz and 30 GHz at various times of the day and year could
be used to construct a model for $\delta\tau_{ion}$), but
for first light, we will follow convention and set $\delta\tau_{ion} \equiv 0$.

\subsubsection{Troposphere\label{s-troposphere}}

Unlike the ionosphere, the troposphere--the first few tens of kilometers of
the Earth's atmosphere--is electrically neutral. Because the tropospheric
index of refraction is not the vacuum value, radio signals will experience
delay, about 8\thinspace ns at zenith and 80\thinspace ns at $E_0=6\degree$.
The tropospheric delay is just 
\begin{equation}
\delta\tau_{tr} = \frac{\scrL}{c},
\end{equation}
\noindent 
with \scrL as determined from equation \ref{e-pathlength} (including
modification for antennas at differing heights). The full 
atmospheric propogation delay is then 
\begin{equation}
\delta\tau_{atmos} = \delta\tau_{tr}+\delta\tau_{ion}.
\end{equation}

\subsection{Thermal Delay\label{s-thermal}}
A further geometric delay is due to temperature variations which cause
a vertical displacement of the antenna reference point.  The magnitude
of the displacement at a height $h$ above the ground is proportional to
the change in temperature $\Delta r = \alpha\Delta T~h$ with $\alpha \sim 12
{\rm ppm/K}$ (Sovers et al. 1998). Note $\Delta T$ here is the
temperature of the antenna structure, not the air temperature.
Diurnal variations in temperature of 10 to 20 K are not uncommon,
so the total displacement could be as much as 0.36 mm for $h = 3$m, 
corresponding to a delay

\begin{equation}
\delta\tau_{therm} = \frac{\Delta r}{c} = \nexpo{1.2}{-3}~{\rm ns}.
\label{e-thermal}
\end{equation}

For a homogeneous array, the thermal delay is not an issue, since
all antennas should respond similarly to temperature changes.
Since CARMA is a heterogeneous array however, this correction becomes
important. What is required to implement this correction is 
a good determination of $\alpha$ and $h(T_0)$ for each antenna type and
temperature sensors on the antenna structure to measure $\Delta T$ (which
might be some average of several sensors).
The BIMA antennas have sensors on the feed legs, the OVRO antennas currently
have no sensors on the antenna structure.  We will stub out the method
for the thermal delay correction, but it will be turned off until
the above requirements are met.

\section{Hardware Limits}

The digitizer board FPGAs provide for delays up to 20~$\mu$s
(a baseline of about 6 km).  The delay engine will make
a best effort to ensure the delays it passes to the lobe rotator
are between 0 and 20~$\mu$s inclusive.  The delays will
always be non-negative, as the delay engine will subtract the
smallest delay (positive or negative) from all delays. So one
antenna should always have a zero delay.  If the resultant
maximum delay is larger than 20~$\mu$s, the delay engine will
reset the ``overflowed'' delay to 20~$\mu$s, log the error, and
set the {\tt outOfRange} monitor point to true.  The fault system should
track this monitor point.


\section{Update Rates}
As outlined in Steve Scott's document ``Phase and Delay Update
Rates,'' phase and delay will be expressed in units of time.
All quantities will be antenna-based. This makes for straightforward
computational loops over antennas, since all subarrays will
share the same array reference position.  The interferometry
engine will calculate a new delay value every 30 seconds and
distribute that information to the lobe rotator controller and
correlator subsystems.  These values are time tagged (as defined
in the Control Subsystem design) with a discontinuity flag for
source change, frequency change, etc. The receiver of the delays
will do quadratic interpolation over the most recent three values.
The update rates have been chosen to track the fastest astronomical
sources (comets; see Pound's ``Suggested Tracking Update Rate for
CARMA''), but not artificial satellites.

\section{Code Testing}
Correctness of the delay values is, of course, very important. We will
test the Delay Engine against the codes of both BIMA and OVRO.
Test harness will be written around the current code bases to produce
tables of UT, sky position, frequency, pressure, temperature, and delay
for several sources around the sky.  The output of the Interferometry
Engine will be compared against these tables with the requirement that
the phase difference at the highest frequency be less than 1\degree.
We will also test simple array configurations, such as a cross, where
the (geometric) delays are easily hand-calculable.  Performance
testing will also be done.

\section{Design Details \protect \footnote{The Delay Engine API is contained
in a separate doxygen-generated document: {\sl The Delay Engine Reference
Manual}} }

In order to keep the Delay Engine up to date on the latest positions,
the Subarray Tracker will call {\tt setAntennaRaDec()} followed
by {\tt computeDelays()} at 30-second intervals.  This interval eliminates
the need for the Delay Engine to do any interpolation on positions.
Upon completion of {\em all} parameter setting for the current
cycle, the SAT must call the Delay Engine {\tt computeDelays()} method to
initiate the delay calculation.
The {\tt computeDelays()} method allows the SAT to make all the
changes it needs before signalling the Delay Engine that a self-consistent
set of delays may now be calculated.
In response to the {\tt computeDelays()} call, the Delay Engine will 
calculate either one value or three, depending on the previous state 
of the system.  The Delay Engine always keeps the 3 most recent
delay states in memory as {\tt DelayInfo} objects. 

For observing the transmitter or drift scans of a planet, an alternative
method for setting positions, {\tt setAntennaAzEl()} is available.

\subsection{Case 1: No state change (``steady-state'')}
If the state has not changed since the last
{\tt computeDelays()} call, the Delay Engine will calculate delays for
the timestamp indicated by the {\tt mjd} given, {\em not} the 
current time.  The timestamp provided by the SAT should be 60
seconds in the future.  Individual antennas do not 
need to use the same timestamp, all though in practice all
antennas in the same subarray probably will.
Example SAT usage:

\begin{verbatim}
#include "carma/interferometry/DelayEngineControl.h"
#include "carma/services/AstroTime.h"

// get the current time
double mjd = Time::MJD(); 

// add one minute. (1 Day)/1440 = one minute.
mjd += ( 1.0/AstroTime::MINUTES_PER_DAY );   

// Imaginary magic call to calculate the correct positions at 
// the given time
calculatePositions(mjd);

// loop over all antennas in the selected subarray and
// set their positions and other parameters
for(int antNo = subarray1.firstAnt; antNo < subarray1.lastAnt; i++ ) {
    delayEngine.setAntennaRaDec(
                                     antNo, mjd,
                                     pntRa[antNo], pntDec[antNo],
                                     phsRa[antNo], phsDec[antNo],
                                     frequency, 
                                     pntDistance, phsDistance, 
                                     false
                                    );
}

// Since weather generally varies smoothly, it should not
// be necessary to set a discontinuity flag when updating the
// weather.   The set of weather parameters from the most
// recent call to setWeatherParameters() is always used
// in the tropospheric delay calculation.
delayEngine.setWeatherParameters(airTemp, atmPressure, dewPoint, relHumid);

// Now signal the Delay Engine to do its thing.
delayEngine.computeDelays();
\end{verbatim}

In this case, the Delay Engine calculates the values the delays 
would have at the timestamp {\tt mjd}, about 60 seconds from ``now''
(``now+30'', ``now'',``now-30'' will be
on the client subsystems' quadratic interpolator stacks; 
``now-30'' gets pushed off).
The new values are passed along to client subsystems.
In steady-state, none of the three DelayInfo objects should have
discontinuity set to true.

\subsection{Case 2: State change}
If some fundamental parameter has changed, as indicated by either
the {\tt setAntennaRaDec} discontinuity flag or a change in
any of the delay status flags, the previous delays become invalid
and a new triplet must be calculated.  If a state change occurs,
the delay engine will set its {\tt NeedsInitialization} monitor
point until it has the three DelayInfo objects required for
client interpolation.
Example SAT usage:

\begin{verbatim}
#include "carma/interferometry/DelayEngineControl.h"
#include "carma/services/AstroTime.h"

bool discontinuity;
double mjd = Time::MJD(); // get the current time

for(int i = 0; i < 3; i++)  {
    // add i * 30 seconds to generate next
    // timestamp. (1 Day)/2880 = 30 seconds.
    mjd += ( (float)i * 0.5 / AstroTime::MINUTES_PER_DAY );  

    // discontinuity is set first time through only.
    discontinuity = ( i == 0 ? true : false ); 

    // Imaginary magic call to calculate the correct positions at 
    // the given time
    calculatePositions(mjd);

    // loop over all antennas in the selected subarray and
    // set their positions and other parameters
    for(int antNo = subarray1.firstAnt; antNo < subarray1.lastAnt; i++ ) {
         delayEngine.setAntennaRaDec(
                                     antNo, mjd,
                                     pntRa[antNo], pntDec[antNo],
                                     phsRa[antNo], phsDec[antNo],
                                     frequency, 
                                     pntDistance, phsDistance, 
                                     false
                                    );
    }
}


// Now signal the Delay Engine to do its thing.
delayEngine.computeDelays();
\end{verbatim}


In this case, all previous delays become invalid.  Therefore the
Delay Engine must calculate three sets of delay values at the
given timestamps (``now'', ``now+30'', ''now+60'') to be passed
along in rapid succession to the client subsystems with the
appropriate discontinuity flags.  When the first discontinuity
flag is encountered the three DelayInfo objects are cleared and
{\tt NeedsInitialization} is set to true.  Once the three DelayInfo
objects are reset, {\tt NeedsInitialization} is set to false, and a
call to {\tt computeDelays()} will initiate the delay calculation.
A call to {\tt computeDelays()} when {\tt NeedsInitialization}
is true will do nothing and return immediately.

Finally, it is possible to change delay status flags or
fixed values without changing the antenna parameters. In
this case the Delay Engine will recompute the delays for
the 3 existing DelayInfo objects and resend them. Example SAT usage:

\begin{verbatim}
#include "carma/interferometry/DelayEngineControl.h"

delayEngine.setAdjustableDelay(subarray.firstAnt, 0.123);
delayEngine.useAdjustableDelay(subarray.firstAnt, true);
delayEngine.computeDelays();
\end{verbatim}


\subsection{Consequences of the Design}

Because we are not distinguishing between subarrays, a call to {\tt
computeDelays()} will cause new delay values to be calculated for
all subarrays regardless of which one just changed source.  This is
a little added overhead for the Delay Engine (calculating new delays
for antennas that might not need it) and for the client subsystems
(flushing their quadratic interpolator buffers and reinitializing
their interpolations).  However, we regard this as a small price
to pay for the simplicity gained by ignoring subarray membership.

The SAT must wait for {\tt computeDelays()} to return.  Timing tests
indicate this is not a problem.  On a 1GHz/PIII/512MB laptop running the
IMR, notification and name services, monitor system, lobe rotator simulator,
and the correlator simulator with 23 antennas and 16 correlator bands,
{\tt computeDelays()} takes about 2.5 seconds. Most of this time is
spent communicating with the correlator bands.

\subsection{Class List}
\begin{tabular}{ll}
\hline
Class & Description \\
\hline
carma.interferometry.DelayEngineControl & The IDL class. \\
carma.interferometry.DelayEngineImpl & The DO implementation, 
subclass of DelayEngineControl \\
carma.interferometry.DelayInfo       & Holds self-consistent delay information.\\
carma.interferometry.DelayEngine     & The Program class. \\
carma.environment.Atmosphere         & Contains refraction, weather related
routines. (Hides gory details).\\
carma.services.constants.Physical    & Physical constants \\
carma.services.constants.Astro       & Astronomical constants\\
carma.services.Units                 & Unit conversion utilities \\
\hline
\end{tabular}


\section{Monitor Points}
The interferometry engine has the following monitor points:

\begin{tabular}{lcrclcc}
\hline
Monitor Point & \# samples & Per Antenna & Type & Units & Persistent? & Fault Node?\\
\hline
adjustableDelay       & 3  & yes & Double  & nanosec & yes & no\\
adjustableDelayStatus & 3  & yes & Boolean &         & yes & no\\
axisDelay             & 3  & yes & Double  & nanosec & yes & no\\
calculatedAt          & 3  & yes & Double  & MJD     & yes & yes\\
calculatedFor         & 3  & yes & Double  & MJD     & yes & no\\
delayOffset           & 3  & yes & Double  & nanosec & yes & no\\
discontinuity         & 3  & no  & Boolean &         & yes & no \\
emulate               & 1  & no  & Boolean &         & yes & yes\\
geometricDelay        & 3  & yes & Double  & nanosec & yes & no\\
geometricDelayStatus  & 3  & yes & Boolean &         & yes & no\\
ionosphericDelay      & 3  & yes & Double  & nanosec & yes & no\\
ionosphericDelayStatus& 3  & yes & Boolean &         & yes & no\\
needsInitialization   & 1  & yes & Boolean &         & yes & yes\\
outOfRange            & 3  & no  & Boolean &         & yes & yes\\
pathLength            & 3  & yes & Double  & meter   & yes & no\\
refractivity (zenith) & 3  & yes & Double  & radian  & yes & no\\
thermalDelay          & 3  & yes & Double  & nanosec & yes & no\\
thermalDelayStatus    & 3  & yes & Boolean &         & yes & no\\
totalDelay            & 3  & yes & Double  & nanosec & yes & no\\
troposphericDelay     & 3  & yes & Double  & nanosec & yes & no\\
troposphericDelayStatus& 3 & yes & Boolean &         & yes & no\\
validUntil            & 3  & yes & Double  & MJD     & yes & yes\\
\hline 
\end{tabular}

\noindent Monitor points with 3 sample points correspond to the three
time-tagged values for interpolation.  The delay for the most recent
time is the first sample. To provide smooth interpolation, old
values are recomputed in the next cycle if a status variable has changed.
If Emulate is true, the delay engine is in emulation mode and is calculating
delays but is {\em not} communicating with the lobe rotator and correlator
bands.  

The monitor point names should follow the hierarchy
{\tt carma.delayengine.antennaN.[name]} for per-antenna points
and {\tt carma.delayengine.[name]} for others.
The delay status monitor points show if a particular delay
correction is or is not being used for calculations made by the
interferometry engine. Note there is no TotalDelayStatus. The
total delay is controlled only through the individual delay
corrections.  The MPs will be produced once every 30 seconds.
The {\tt CalculatedAt} is the time the {\em at} which delays were
calculated; {\tt CalculatedFor} is the time {\em for} which they
were calculated; {\tt ValidUntil} is the time when they expire; 
{\tt NeedsInitialization} indicates the Delay Engine does not have
enough information to calculate the current delays (e.g. doesn't know the
source RA and DEC).

\subsection{Fault Nodes}
There are a few monitor points that may be used as nodes for the fault
system.  Because the Delay Engine does not run on the standard half-second
cycle, there are two timestamp monitor points to indicate data validity
( {\tt calculatedAt, validUntil}), one to indicate the Delay Engine is
in a happy state ({\tt needsInitialization}), and one that indicates the
delay value is within hardware limits.  These are all ``per antenna''
monitor points, so for each category there will be as many as there are
active antennas.  A tru value for the {\tt emulate} monitor point indicates
that the delay engine is in emulation mode and not communicating with
its client DOs.  This may or may not signal a fault.

\section{FTE Estimate and Milestones}
The FTE estimate is as follows:

\begin{tabular}{lrcl}
\hline
Stage & & Months & Responsible \\
\hline
Design  & & 2.0 & both\\
Implementation& & & \\
  & IDL interface & 0.25 & Kraybill \\
  & refractivity  & 0.5 & Pound \\
  & delays        & 0.5 & Pound \\
  & Monitor system interface & 0.5 & Pound \\
  & Program interface & 0.25 & Pound \\
  &       \cline{1-2}
  & Total             & 2.0  & \\
Testing & & 1.0 & both\\
Integration & &  1.0 & both\\
\hline
Package Total & & 6.0 & FTE months  \\
\end{tabular}

\clearpage
The schedule for completion is as follows:

\begin{tabular}{ll}
\hline
Task & Completion Date\\
\hline
Conceptual Design Review  & 08/20/2003\\
Preliminary Design Review & 11/06/2003\\
Critical Design Review    & 01/26/2004\\
API defined                          & completed \\
IDL interface coded                  & completed \\
Program interface roughed out        & completed \\
Delay methods coded                  & completed \\
Monitor system mpml coded            & completed \\
Refractivity and pathlength coded    & completed\\
Program interface completed          & completed \\
Monitor system interface completed   & completed \\
Unit Tests coded                     & completed \\
Integration                          & completed \\
\end{tabular}

\begin{thebibliography}{}
\bibitem[]{} Crane, R.K. 1976, Methods of Experimental Physics, 12B, 186
\bibitem[]{} Davis, J.L., Herring, T.A., Shapiro, I.I., Rogers, A.E.E., \& Elgered, G. 1985, Radio Science, 20, 1593
\bibitem[]{} Hill, R.J. \& Clifford, S.F.  1981, Radio Science, 16, 77
\bibitem[]{} Mangum, J. 2001, ALMA Memo 366
\bibitem[]{} Marini, J.W. 1972, Radio Science, 7, 223
\bibitem[]{} Ulich, B.L.  1981, Int. J. Infrared and Millimeter Waves, 2, 293
\bibitem[]{} Smart, W. M. 1977, \underline{Textbook on Spherical Astronomy} 
\bibitem[]{} Smith, E.K. \& Weintraub, S. 1953, Proc. IRE, 41, 1035
\bibitem[]{} Spitzer, L. 1978, \underline{Physical Processes in the Interstellar Medium}, 61{\it ff}
\bibitem[]{} Thayer, G.D. 1974, Radio Science, 9, 803
\bibitem[]{} Thompson, Moran, \& Swenson, 1st edition, chapters 4 and 13
\bibitem[]{} Wright, M.C.H.W 1990, BIMA Memo 2
\bibitem[]{} Yan, H. 1996, AJ, 112, 1312
\end{thebibliography}

\begin{deluxetable}{cl}
\tablecaption{Symbol definitions}
\tablehead{
\colhead{Symbol} &
\colhead{Definition} 
}
\startdata
$c$ & speed of light \\
$D_a$ & antenna axis misalignment \\
$\Delta h$ & antenna altitude displacement from array reference plane\\ 
$\delta$ & declination\\
$\delta\tau_{0}$ & fixed delay line offset\\
$\delta\tau_{adj}$ & adjustable delay\\
$\delta\tau_{atmos}$ & total atmospheric delay \\
$\delta\tau_{ion}$ & ionospheric delay\\
$\delta\tau_{tr}$ & tropospheric delay \\
$\delta\tau_g$ & geometric delay \\
$\delta\tau_{therm}$ & thermal delay due to structural expansion\\
$e_{sat}$ & saturated water vapor pressure \\
$E_0$ & antenna elevation \\
$E_{min}$ & minimum elevation at which to compute full refractivity \\ 
$h_D$ & atmospheric scale height of dry air\\ 
$h_V$ & atmophseric scale height of water vapor\\
$h_{IR}$ & infrared scale height \\
$H$ & hour angle\\
\scrL & pathlength (integrated refractivity towards source) \\
$n$ & index of refraction \\
$N_0$ & total zenith refractivity \\
$N_{IR}$ & infrared zenith refractivity \\
$N_{opt}$ & optical zenith refractivity \\
$N_{opt,D}$ & optical zenith refractivity due to dry air\\
$N_{opt,V}$ & optical zenith refractivity due to water vapor\\
$\nu$ & observing frequency, band center \\
$\nu_g$ & ionospheric electron gyrofrequency\\
$\nu_p$ & ionospheric plasma frequency \\
$P$   & atmospheric pressure\\
$p_D$ & partial pressure of dry air \\
$p_V$ & partial pressure of water vapor \\
\rearth & radius of Earth \\
$R$     & refraction \\
$R_0$ & total zenith refraction \\
$RH$  & relative humidity, measured at ground \\
$T$   & ambient temperature, measured at ground \\
$X_{\lambda}$   & antenna $X$ position, equatorial coordinates \\
$Y_{\lambda}$   & antenna $Y$ position, equatorial coordinates \\
$Z_{\lambda}$   & antenna $Z$ position, equatorial coordinates \\
\enddata
\label{t-symbols}
\end{deluxetable}

\begin{deluxetable}{cc}
\tablecaption{Values To Be Used for Constants}
\tablehead{
\colhead{Symbol} &
\colhead{Value} 
}
\startdata
$E_{min}$ & 1$^{\circ}$\\ 
$h_D$ & 9.48 km \\ 
%(ovro) 7 km (bima)\\
$h_V$ & 2.103 km \\
%(ovro) \dots (bima)\\
$h_{IR}$ & 2.216 km \\
\rearth & 6378.14 km \\
%(ovro) 2.5 km (bima)\\
%$A$   & first-order coefficient of refractivity frequency dependence\tablenotemark{1} & -2.84E-5 \\
%$B$   & second-order coefficient of refractivity frequency dependence\tablenotemark{1} & 5.40E-7\\
\enddata
%\tablenotetext{a}{Values for for frequency in GHz.}
\end{deluxetable}


\begin{figure}[!ht]
\epsscale{0.8}
\plotone{refractivity.ps}
\caption{Frequency dependence of refractivity from Hill \& Clifford (1981),
calculated for $T = 300$ K, $P = 1$ atm, $RH = 80\%$, and $p_V = 28.2$ mbar
(solid line). We have normalized the refractivity by its value at 
50 GHz. Dashed line shows a quadratic fit to the data that should be
adequate for CARMA's purposes.
}\label{f-hcrefrac}
\end{figure}
\clearpage

\end{document}


