%$Id: loberotator.tex,v 1.7 2003/12/10 02:03:06 colby Exp $
\documentclass[preprint]{aastex}
\usepackage{mathrsfs}
\usepackage{epsf}
% ----------------------DEFINITIONS USED IN THIS PAPER --------------------
\newcommand{\xlam}{
   \ifmmode{X_\lambda}
   \else{$X_\lambda$}
   \fi}
\newcommand{\ylam}{
   \ifmmode{Y_\lambda}
   \else{$Y_\lambda$}
   \fi}
\newcommand{\zlam}{
   \ifmmode{Z_\lambda}
   \else{$Z_\lambda$}
   \fi}
\newcommand{\rearth}{  % Earth radius
   \ifmmode{r_\oplus}
   \else{$r_\oplus$}
   \fi}
\newcommand{\scrL}{ % script L for path length
   \ifmmode{{\mathscr L}}
   \else{${\mathscr L}$\/}
   \fi}
\def\degree{\ifmmode{^\circ} \else{$^\circ$}\fi}
\def\nexpo#1#2{\ifmmode{#1 \times 10^{#2}}\else{$#1 \times 10^{#2}$}\fi}
\def\expo#1{\ifmmode{10^{#1}}\else{$10^{#1}$}\fi}
\def\msun{\ifmmode{\rm M_\odot}\else{M$_\odot$}\fi}
\def\esat{\ifmmode{\rm e_{sat}}\else{e$_{sat}$}\fi}
\def\etal           {{\rm et~al}.\ }
\def\eg             {{\rm e.g.}\/}
\def\ie             {{\rm i.e.}\/}
\def\uv{{\it uv}}
\def\cmm {\ifmmode{{\rm cm^{-3}}}\else{{${\rm cm^{-3}}$}}\fi}
\def\cmtwo {\ifmmode{{\rm cm^{-2}}}\else{{${\rm cm^{-2}}$}}\fi}
\def\invpc{\ifmmode{\,{\rm pc^{-1}}}\else{\thinspace {\rm pc$^{-1}$}}\fi}
\def\invsec{\ifmmode{\,{\rm s^{-1}}}\else{\thinspace {\rm s$^{-1}$}}\fi}
\def\invyr{\ifmmode{\,{\rm yr^{-1}}}\else{\thinspace {\rm yr$^{-1}$}}\fi}
\def\psec           {$.\negthinspace^{s}$}
\def\pmin           {$.\negthinspace^{m}$}
\def\pasec           {$.\negthinspace^{\prime\prime}$}
\def\pdeg           {$.\kern-.25em ^{^\circ}$}
\def\kms{\ifmmode{\,km\,s^{-1}}\else{\thinspace km\thinspace s$^{-1}$}\fi}
%
% Greater than or approximately equal to and less than or
% approximately equal to signs, using macros defined in PLAIN.TEX
%
\catcode`\@=11 \def\gapprox{\mathrel{\mathpalette\@versim>}}
\def\lapprox{\mathrel{\mathpalette\@versim<}}
\def\@versim#1#2{\lower2.9truept\vbox{\baselineskip0pt\lineskip0.5truept
    \ialign{$\m@th#1\hfil##\hfil$\crcr#2\crcr\sim\crcr}}}
\catcode`\@=12
% --------------------------------------------------------------------------
\begin{document}
\title{CARMA Lobe Rotator Software Design}
\author{Colby Kraybill}
\slugcomment{$Revision: 1.7 $}


\section{Introduction}
\subsection{Functionality and Scope}

The primary purpose of the Lobe Rotator subsystem is to control
and monitor the lobe rotator hardware.  This system is interconnected
with the interferometry engine and the correlator.

The Lobe Rotator Server (LRS) sends information to the Lobe Rotator
Boards (LRB) that in turn generate a 50MHz reference signal for the 1st
Local Oscillator phaselocks.  This base tone is modified to follow fringe
pattern changes due to the dynamically changing geometry between the
telescopes and the source being observed and other changes in the
electrical path length inbetween the telescopes and the correlator.

The LRS also loads phase shift patterns (Walsh sequences) to the
LRB and tells the LRB to use a particular sequence at the next integral second
frame.  

The interferometry subsystem (Interferometry Engine or IE) drives the LRS
by passing in phase and frequency information for computing the modified 50MHz
signal.

The walsh patterns are chosen well ahead of time and based on
individual array configurations.  Additional modifications to the walsh
sequence are made to handle sideband separation. 

I recommend having the following documents on hand while reading this one:
www.mmarray.org/project/WP/LobeRotator/hw/pdr/LobeRotatorCANbusAPI.pdf
www.mmarray.org/project/docs/SystemDesign/PhaseDelayUpdates.pdf

\subsection{Lobe Rotator Board}
The LRB mixes a 40MHz base tone with a ~10MHz one to generate
a dynamically modified 50MHz reference signal.  This reference
signal is then distributed to individual antennas.  

The ~10MHz tone is generated using Direct Digital Synthesis (DDS) chips.
Each DDS chip is told a particular phase and phase rate (frequency) for generating
a waveform by the LRS (see below).  There is one DDS chip per
antenna.  Each LRB has four DDS chips and there will be six boards
for a total of 24 DDS channels.

These waveforms are only valid for each second observing frame.
Each waveform is aligned to other subsystems by starting at a
one second timing heartbeat sent to each LRB over the CANbus cable.

It is the job of the LRS to load the pre-defined walsh sequences 
as a table into each LRB at LRS initialization (when first powered
up).  The tables take the form of a 4-bit pattern sequence, with
values of 0,1,2,3 corrisponding to phase shifts of 0, 90, 180 and 270 degrees.

The LRS then supplies information about which sequence
to use and when (via an MJD time-stamp).  The LRS will also
supply the phase and phase rate (frequency) loaded into each DDS chip.

Finer details of LRB hardware are discussed in the Lobe Rotator
CANBus API document (LobeRotatorCANbusAPI.pdf).

\subsection{Lobe Rotator Server}
The server consists of a standard cPCI crate with a single CPU board
running the CARMA standard Linux distribution, NFS mounted
disks and a CORBA DO (or LRS DO).  It has a CANBus card for
communicating with the LRB's.

CANbus communications will occur via the standard CARMA CANBus
classes (i.e. not directCAN).

The LRS DO is fed information by the IE to compute the proper phase and frequency each DDS chip
should be set to at the next absolute second.  The IE is continuously 
computing delay information every thirty seconds, based on phase and pointing centers
and source position and sky observing frequency (base frequency + doppler shift).  These
delays are passed to the LRS every 30 seconds, each update comes with a time stamp
for which the delays are applicable.  The LRS then uses the
delay (in nanoseconds) and frequency information to determine the proper
phase and frequency to be sent to the LRBs.  The LRS will keep track of the
30 second updates and interpolate a smooth changing phase and frequency 
for loading into the LRBs.

This system will re-use the CARMA::util package for interpolation routines.

See the PhaseDelayUpdates document for detailed information on how the 30 second update
was determined to fit accuracy requirements.  These requirements are primarily
driven by how fast the gemometry between source and instrument change.  These values
must be updates at least once every 41 seconds.  The IE design has been simplified
to update once every 30 seconds to stay well within the changing geometry.
(PhaseDelayUpdates.pdf)

After interpolation of the delay information supplied by the IE, proper phase and phase
rate (frequency) are computed once each second and loaded into the DDS chips
via the lobe rotator CANBus API, see LobeRotatorCANbusAPI.pdf).  These values will
be loaded 400-500ms before the next integral second.  The LRS will have to
monitor which DDS registers are free for the new values to be utilized.  This
information can be sniffed off the monitor stream on the next half second update frame.

The LRS DO is also fed information about which walsh sequences to
use by the Sub-Array Controller (SAC).  These sequences are pre-generated
patterns, stored in tables that are read from a main configuration
database and loaded into the LRB's (what about the correlator and
anything else that needs these tables to be loaded, where's the code
reuse?)  It is up to the SAC to keep track of which walsh sequence is being
used to modify signals from individual antennas.

\section{Code Testing}
The LRS core code will have have a small testing harness to
generate false monitor information and be excersized by a
sub-array controller that just loads configuration information
from a file and feeds it into the LRS.  This also applies
to inputs from the interferometry engine.  An idealized set of values
will be also read from a file for reporting the false monitor
information.  This will provide cross checking for monitor
correctness.  (Other testing modes?)

\clearpage

\section{Revised FTE Estimate and Milestones}
The original work package had many guesses for the schedule
based on incomplete information about hardware design.  A
revised estimate follows:

\begin{tabular}{lrcl}
\hline
Stage & & Months & \\
\hline
Design  & & 1.0 & \\
Implementation& & & \\
  & IDL interface & 0.125 & \\
  & Program interface & 0.125 & \\
  & Core LRS code & 0.25 & \\
  &       \cline{1-2}
  & Total             & 1.0 & \\
Integration & &  0.5 & \\
Testing & & 1.0 & \\
\hline
Package Total & & 3.5 & FTE months  \\
\end{tabular}

\bigskip
The schedule for completion is as follows:

\begin{tabular}{lc}
\hline
Task & When \\
\hline
API defined                          & Completed (PDR depending). \\
IDL interface coded                  & Dec 3, 2003 \\
Program interface completed          & Dec 15, 2003  \\
Core LRS code                        & Dec 30, 2003 \\
Unit Tests written                   & Dec 30, 2004 \\
Integration                          & Jan 15, 2004 (depends on Interf Engine)\\
\end{tabular}

\clearpage
\section{Monitor Points}
The lobe rotator engine has the following monitor points:

\begin{tabular}{lrcl}
\hline
Monitor Point & per each DDS channel or LRB & Type & Units \\
\hline

WalshColumn             & DDS & Integer & \\
ActiveFrequency         & DDS & double & Hz \\
LastFreqWrite           & DDS & Boolean & \\
LRB5VPS1                & LRB & Single & mV \\
LRB5VPS2                & LRB & Single & mV \\
LRB5VPS3                & LRB & Single & mV \\
LRB24VPS1               & LRB & Single & mV \\
BoardTemp               & LRB & Single & C \\
TimeStatus              & LRB & Integer & \\
TimeSync                & LRB & Integer & \\
ExternalClock           & LRB & Integer & \\
PhaseSwitching          & LRB & Boolean & \\
ModuleID                & LRB & Integer & \\
ModuleType              & LRB & Integer & \\
InitRequest             & LRB & Integer & \\
CANRxErrs               & LRB & Integer & \\
CANTxErrs               & LRB & Integer & \\
MemErrs                 & LRB & Integer & \\
SysErrs                 & LRB & Integer & \\
SchedulerOverflow       & LRB & Integer & \\
TimedSchedulerOverflow  & LRB & Integer & \\
SoftwareMajorVersion    & LRB & Integer & \\
SoftwareMinorVersion    & LRB & Integer & \\
SoftwareVersionTest     & LRB & Integer & \\
CommErrs                & LRB & Integer & \\
TimeErrs                & LRB & Integer & \\
SoftErrs                & LRB & Integer & \\
HardErrs                & LRB & Integer & \\
AverageTimeOffset       & LRB & Integer & ms \\
TimeStampInt            & LRB & Integer & ms \\
TimeDelta               & LRB & Integer & ms \\
\hline
\end{tabular}

The WalshColumn is the selected column for walsh values in the pre-supplied
table, for each individual DDS channel (Antenna).

ActiveFrequency is the current active frequency being used at the
integral one second.

LastFreqWrite will be true if the last frequency update was made
in time.

LRB5VPS1-3, LRB24VPS1 are power supply voltages for each board.

BoardTemp, temperatures of individual boards.

TimeStatus is a bit pattern for coarse notification of timer related
problems (such as time sync error/module time OK, time sync PK/module time error...).

TimeSync is a bit pattern for relaying information on what timing period
is in use (non, 100ms, 1 s...).

ExternalClock is a bit pattern for information on the phase switching clock
(none, 1ms, 6.25ms...).

PhaseSwitching is a coarse indicator of whether or not phase switching
has been turned on.

ModuleID is the serial number for each LRB.

ModuleType is the module type for each LRB.

InitRequest the LRB is requesting to be initialized.

CANRxErrs number of Rx errors on the CANBus for this LRB.

CANTxErrs number of Tx errors on the CANBus for this LRB.

MemoryErrs number of memory errors on this LRB.

SysErrs number of system errors on this LRB.

SchedulerOverflow 0x0000 - 0xFFFF

TimedSchedulerOverflow 0x0000 - 0xFFFF

SoftwareMajorVersion, major version.

SoftwareMinorVersion, minor version.

SoftwareTestVersion, ??

CommErrs overall communication error count.

TimeErrs overall timer error count.

SoftErrs overall software error count.

HardErrs overall hardware error count.

AverageTimeOffset (<t{internal} - t{system}>).

TimeStampInt is the interval between receiving time stamps.

TimeDelta is the instantaneous difference, (t{internal} - t{system})

\section{Questions/Open issues}

\begin{enumerate}
\item Should the Sub-Array Controller handle the downloading of walsh
tables via the LRS DO and the correlator DO's?

\item Note engineering vs normal operations methods in API.  Should
a method be created to download new program code to the LRBs?

\item Walsh tables will primary come from flat files until
we decide who/what controls the walsh table info

\item How to best synchronize phase switching tables?  What about
walsh rows that include variable length sequences? (determine
it dynamically via computing length of walsh sequence...)

\item How best to test without hardware?

\end{enumerate}

\clearpage

\section{Lobe Rotator API}

The following is the raw IDL of the LRS DO.  It can be found in
\begin{verbatim}
$CARMA/carma/loberotator/LobeRotatorControl.idl
\end{verbatim}

\verbatiminput{LobeRotatorControl.idl}

\end{document}


